%% LaTeX Curriculum Vitae Template
%%
%% Copyright (C) 2004-2009 Jason Blevins <jrblevin@sdf.lonestar.org>
%% http://jblevins.org/projects/cv-template/
%%
%% You may use use this document as a template to create your own CV
%% and you may redistribute the source code freely. No attribution is
%% required in any resulting documents. I do ask that you please leave
%% this notice and the above URL in the source code if you choose to
%% redistribute this file.
%
%\documentclass[letterpaper,12pt]{article}
%
%\usepackage{hyperref}
%\usepackage{geometry}
%
%
%% Comment the following lines to use the default Computer Modern font
%% instead of the Palatino font provided by the mathpazo package.
%% Remove the 'osf' bit if you don't like the old style figures.
%\usepackage[T1]{fontenc}
%\usepackage[utf8]{inputenc}
%\usepackage[sc]{mathpazo}
%\usepackage[croatian]{babel}
%% Set your name here
%\def\name{Jelena Luetić}
%
%% Replace this with a link to your CV if you like, or set it empty
%% (as in \def\footerlink{}) to remove the link in the footer:
%\def\footerlink{}
%
%% The following metadata will show up in the PDF properties
%\hypersetup{
%  colorlinks = true,
%  urlcolor = black,
%  pdfauthor = {\name},
%  pdfkeywords = {physics, statistics, mathematics},
%  pdftitle = {\name: Curriculum Vitae},
%  pdfsubject = {Curriculum Vitae},
%  pdfpagemode = UseNone
%}
%
%\geometry{
%  body={6.5in, 8.5in},
%  left=1.0in,
%  top=1.25in
%}
%
%% Customize page headers
%\pagestyle{myheadings}
%\markright{\name}
%\thispagestyle{empty}
%
%% Custom section fonts
%\usepackage{sectsty}
%\sectionfont{\rmfamily\mdseries\Large}
%\subsectionfont{\rmfamily\mdseries\itshape\large}
%
%% Other possible font commands include:
%% \ttfamily for teletype,
%% \sffamily for sans serif,
%% \bfseries for bold,
%% \scshape for small caps,
%% \normalsize, \large, \Large, \LARGE sizes.
%
%% Don't indent paragraphs.
%\setlength\parindent{0em}
%
%% Make lists without bullets
%%\renewenvironment{itemize}{
%%  \begin{list}{}{
%%    \setlength{\leftmargin}{1.5em}
%%  }
%%}{
%%  \end{list}
%%}
%
%\begin{document}
%
%% Place name at left
%{\huge \name}
%
%% Alternatively, print name centered and bold:
%%\centerline{\huge \bf \name}
%
%\vspace{0.25in}
%
%%\begin{minipage}{0.45\linewidth}
%%  \href{http://www.irb.hr/}{Ruđer Bošković Institute} \\
%%  Department of experimental physics \\
%%  Bijenička 54, 10000 Zagreb
%%\end{minipage}

\chapter*{Curriculum vitae}
\fancyhead[LE,RO]{\emph{Curriculum vitae}}
\section*{Personal}


  \begin{tabular}{r|l}
    Name & Jelena Luetić \\
	Address & Supilova 7, 10000 Zagreb, Croatia \\   
    Date of Birth & April 5, 1987 \\
    Place of birth & Split, Croatia \\
    Citizenship & Croatian \\
    Phone & +385 91 1480103 \\
    Email & \href{mailto:jelena.luetic@cern.ch}{\tt jelena.luetic@cern.ch} \\
    Languages & English, basic Italian and French, Croatian - native \\
  \end{tabular}



\section*{Education}

\begin{table}[h!]
 \centering
\begin{tabular}{r | l}
2011 - & \textbf{PhD}, Faculty of Science, Physics department \\ & \textit{Title}: 
Measurement of the cross section for associated production of a W boson \\ & and two b quarks with the CMS detector at the Large Hadron Collider \\ &  \textit{Advisor:} Prof. Vuko Brigljević, thesis defence is scheduled for July, 15th 2015. \\[5pt] 
2005 - 2010 & \textbf{MSc}, Faculty of Science, Physics department \\  & \textit{Title}: Measurement of Z boson cross section in proton-proton collisions \\ & with CMS detector at Large Hadron Collider \\ &  \textit{Advisor:} Prof. Ivica Puljak, thesis defended on November, 30th 2010. \\ 
\end{tabular}
\end{table}


\section*{Professional experience}

Since 2009 I've been a part of the CMS group at Ruđer Bošković Institute, working on gauge boson produced in association with jets measurements and on technical aspects of CMS detector:
\begin{itemize}
\item One of the main contributors to Wbb cross section measurement at $\sqrt{s} =$ 8 TeV. Collaboration with University of Wisconsin-Madison, University of Trieste and CERN.
\item Responsible for Lorentz angle monitoring and calibration for CMS Pixel detector.
\item Responsible for development and maintenance of technical tools used in the offline analysis shared with other members of CMS Pixel group.
\item Participated in pixel detector recommissioning during the long shutdown. 
\item Paricipated in CMS Pixel Operations
\end{itemize}



\section*{Teaching}

\begin{tabular}{r | l}
  2011 - 2015 & Faculty of Science, Zagreb, Croatia \\ & Teaching assistant in 2 courses - \textit{Introductory laboratory exercises} (2011.- 2012.) \\ & and \textit{Programming in C} (2011.-2015.) \\[5pt]
  2010 - 2011 & Faculty of Electrical Engineering, Mechanical Engineering and Naval \\ & Architecture in Split, Croatia \\ & Teaching assistant - \textit{Laboratory exercises in Modern physics} 
\end{tabular}

\section*{Schools and Conferences}

\begin{tabular}{r | l}
  2009 & CERN Summer Student Programme, Geneva, Switzerland \\[5pt]
  2010 & CERN School of Computing, London, UK \\[5pt]
  2011 & CMSDAS - Data analysis school, Pisa, Italy \\[5pt]
  2012 & Silicon Detector Workshop, Split, Croatia \\
  	   & \textit{Talk:} Lorentz angle measurement in CMS Pixel detector\\[5pt] 
  2013 & EDIT - Excellence in Detectors, Tsukuba, Japan \\
  	   & \textit{Poster}: CMS Pixel detector and Lorentz angle determination \\[5pt]
  2014 & Fermilab - CERN Hadron Collider School, Fermilab, USA\\
\end{tabular}

\section*{Computer skills}
\begin{itemize}
\item Computer languages: C, C++, Fortran, Python 
\item Software: ROOT, Mathematica, LabView
\end{itemize}

\section*{Other}

Participation in various physics and science outreach programs:
\begin{itemize}
	\item Organization of International Masterclasses: lectures and laboratory exercises for high school students covering various topics in high energy physics.
	\item Participated in several science fairs for general public demonstrating experiments. 
\end{itemize}


\section*{Publications}

\setstretch{1.}
\begin{enumerate}

%%   LIST OF PAPERS
%%   Please comment out anything between here and the
%%   first \item
%%   Please send any updates or corrections to the list to
%%   admin@inspirehep.net
%\cite{Aad:2015zhl}
\item%{Aad:2015zhl}
{\bf ``Combined Measurement of the Higgs Boson Mass in $pp$ Collisions at $\sqrt{s}=7$ and 8 TeV with the ATLAS and CMS Experiments''}, 
G.~Aad {\it et al.}  [ATLAS and CMS Collaborations], 
  %arXiv:1503.07589 [hep-ex]
    %{}10.1103/PhysRevLett.114.191803
Phys.\ Rev.\ Lett.\  {\bf 114}191803 (2015) %(Mar 262015)
%\href{http://inspirehep.net/record/1356276}{HEP entry}
%74 citations counted in INSPIRE as of 02 Jul 2015


%\cite{Khachatryan:2015axa}
\item%{Khachatryan:2015axa}
{\bf ``Search for vector-like T quarks decaying to top quarks and Higgs bosons in the all-hadronic channel using jet substructure''}, 
  V.~Khachatryan {\it et al.}  [CMS Collaboration], 
  %arXiv:1503.01952 [hep-ex]
    %{}10.1007/JHEP06(2015)080
JHEP {\bf 1506}080 (2015) %(Mar 62015)
%\href{http://inspirehep.net/record/1348542}{HEP entry}
%5 citations counted in INSPIRE as of 02 Jul 2015


%\cite{Khachatryan:2015rja}
\item%{Khachatryan:2015rja}
{\bf ``Study of final-state radiation in decays of Z bosons produced in $pp$ collisions at 7 TeV''}, 
  V.~Khachatryan {\it et al.}  [CMS Collaboration], 
  %arXiv:1502.07940 [hep-ex]
    %{}10.1103/PhysRevD.91.092012
Phys.\ Rev.\ D {\bf 91}no. 9092012 (2015) %(Feb 272015)
%\href{http://inspirehep.net/record/1346843}{HEP entry}



%\cite{Khachatryan:2015lwa}
\item%{Khachatryan:2015lwa}
{\bf ``Search for physics beyond the standard model in events with two leptonsjetsand missing transverse momentum in pp collisions at sqrt(s) = 8 TeV''}, 
  V.~Khachatryan {\it et al.}  [CMS Collaboration], 
  %arXiv:1502.06031 [hep-ex]
    %{}10.1007/JHEP04(2015)124
JHEP {\bf 1504}124 (2015) %(Feb 202015)
%\href{http://inspirehep.net/record/1345823}{HEP entry}
%21 citations counted in INSPIRE as of 02 Jul 2015


%\cite{Khachatryan:2015kea}
\item%{Khachatryan:2015kea}
{\bf ``Measurement of the Z$\gamma$ production cross section in pp collisions at 8 TeV and search for anomalous triple gauge boson couplings''}, 
  V.~Khachatryan {\it et al.}  [CMS Collaboration], 
  %arXiv:1502.05664 [hep-ex]
    %{}10.1007/JHEP04(2015)164
JHEP {\bf 1504}164 (2015) %(Feb 192015)
%\href{http://inspirehep.net/record/1345354}{HEP entry}
%2 citations counted in INSPIRE as of 02 Jul 2015


%\cite{Khachatryan:2015xaa}
\item%{Khachatryan:2015xaa}
{\bf ``Nuclear effects on the transverse momentum spectra of charged particles in pPb collisions at $\sqrt{s_{_\mathrm {NN}}} =5.02$ TeV''}, 
  V.~Khachatryan {\it et al.}  [CMS Collaboration], 
  %arXiv:1502.05387 [nucl-ex]
    %{}10.1140/epjc/s10052-015-3435-4
Eur.\ Phys.\ J.\ C {\bf 75}no. 5237 (2015) %(Feb 182015)
%\href{http://inspirehep.net/record/1345263}{HEP entry}
%4 citations counted in INSPIRE as of 02 Jul 2015


%\cite{Khachatryan:2015vra}
\item%{Khachatryan:2015vra}
{\bf ``Searches for supersymmetry using the M$_{T2}$ variable in hadronic events produced in pp collisions at 8 TeV''}, 
  V.~Khachatryan {\it et al.}  [CMS Collaboration], 
  %arXiv:1502.04358 [hep-ex]
    %{}10.1007/JHEP05(2015)078
JHEP {\bf 1505}078 (2015) %(Feb 152015)
%\href{http://inspirehep.net/record/1345027}{HEP entry}
%18 citations counted in INSPIRE as of 02 Jul 2015


%\cite{Khachatryan:2015rra}
\item%{Khachatryan:2015rra}
{\bf ``Measurement of J/$\psi$ and $\psi$(2S) Prompt Double-Differential Cross Sections in pp Collisions at $\sqrt{s}$=7 TeV''}, 
  V.~Khachatryan {\it et al.}  [CMS Collaboration], 
  %arXiv:1502.04155 [hep-ex]
    %{}10.1103/PhysRevLett.114.191802
Phys.\ Rev.\ Lett.\  {\bf 114}no. 19191802 (2015) %(Feb 132015)
%\href{http://inspirehep.net/record/1345023}{HEP entry}
%1 citations counted in INSPIRE as of 02 Jul 2015


%\cite{Khachatryan:2015hwa}
\item%{Khachatryan:2015hwa}
{\bf ``Performance of electron reconstruction and selection with the CMS detector in proton-proton collisions at $\sqrt{s}$ = 8  TeV''}, 
  V.~Khachatryan {\it et al.}  [CMS Collaboration], 
  %arXiv:1502.02701 [physics.ins-det]
    %{}10.1088/1748-0221/10/06/P06005
JINST {\bf 10}no. 06P06005 (2015) %(Feb 92015)
%\href{http://inspirehep.net/record/1343791}{HEP entry}
%25 citations counted in INSPIRE as of 02 Jul 2015


%\cite{Khachatryan:2015ila}
\item%{Khachatryan:2015ila}
{\bf ``Search for a standard model Higgs boson produced in association with a top-quark pair and decaying to bottom quarks using a matrix element method''}, 
  V.~Khachatryan {\it et al.}  [CMS Collaboration], 
  %arXiv:1502.02485 [hep-ex]
    %{}10.1140/epjc/s10052-015-3454-1
Eur.\ Phys.\ J.\ C {\bf 75}no. 6251 (2015) %(Feb 92015)
%\href{http://inspirehep.net/record/1343506}{HEP entry}
%10 citations counted in INSPIRE as of 02 Jul 2015


%\cite{Khachatryan:2015pwa}
\item%{Khachatryan:2015pwa}
{\bf ``Search for supersymmetry using razor variables in events with $b$-tagged jets in $pp$ collisions at $\sqrt{s} =$ 8 TeV''}, 
  V.~Khachatryan {\it et al.}  [CMS Collaboration], 
  %arXiv:1502.00300 [hep-ex]
    %{}10.1103/PhysRevD.91.052018
Phys.\ Rev.\ D {\bf 91}052018 (2015) %(Feb 12015)
%\href{http://inspirehep.net/record/1342447}{HEP entry}
%13 citations counted in INSPIRE as of 02 Jul 2015


%\cite{Khachatryan:2015jha}
\item%{Khachatryan:2015jha}
  {\bf ``Search for decays of stopped long-lived particles produced in proton–proton collisions at $\sqrt{s}= 8\,\text {TeV} $''}, 
  V.~Khachatryan {\it et al.}  [CMS Collaboration], 
  %arXiv:1501.05603 [hep-ex]
    %{}10.1140/epjc/s10052-015-3367-z
Eur.\ Phys.\ J.\ C {\bf 75}no. 4151 (2015) %(Jan 222015)
%\href{http://inspirehep.net/record/1340699}{HEP entry}
%8 citations counted in INSPIRE as of 02 Jul 2015


%\cite{Khachatryan:2015sja}
\item%{Khachatryan:2015sja}
{\bf ``Search for resonances and quantum black holes using dijet mass spectra in proton-proton collisions at $\sqrt{s} =$ 8 TeV''}, 
  V.~Khachatryan {\it et al.}  [CMS Collaboration], 
  %arXiv:1501.04198 [hep-ex]
    %{}10.1103/PhysRevD.91.052009
Phys.\ Rev.\ D {\bf 91}no. 5052009 (2015) %(Jan 172015)
%\href{http://inspirehep.net/record/1340084}{HEP entry}
%27 citations counted in INSPIRE as of 02 Jul 2015


%\cite{Khachatryan:2014jba}
\item%{Khachatryan:2014jba}
{\bf ``Precise determination of the mass of the Higgs boson and tests of compatibility of its couplings with the standard model predictions using proton collisions at 7 and 8 $\,\text {TeV}$''}, 
  V.~Khachatryan {\it et al.}  [CMS Collaboration], 
  %arXiv:1412.8662 [hep-ex]
    %{}10.1140/epjc/s10052-015-3351-7
Eur.\ Phys.\ J.\ C {\bf 75}no. 5212 (2015) %(Dec 302014)
%\href{http://inspirehep.net/record/1335811}{HEP entry}
%124 citations counted in INSPIRE as of 02 Jul 2015


%\cite{Khachatryan:2014lpa}
\item%{Khachatryan:2014lpa}
{\bf ``Search for pair-produced resonances decaying to jet pairs in proton-proton collisions at sqrt(s) = 8 TeV''}, 
  V.~Khachatryan {\it et al.}  [CMS Collaboration], 
  %arXiv:1412.7706 [hep-ex]
    %{}10.1016/j.physletb.2015.04.045
Phys.\ Lett.\ B {\bf 747}98 (2015) %(Dec 242014)
%\href{http://inspirehep.net/record/1335501}{HEP entry}
%14 citations counted in INSPIRE as of 02 Jul 2015


%\cite{Khachatryan:2014fba}
\item%{Khachatryan:2014fba}
{\bf ``Search for physics beyond the standard model in dilepton mass spectra in proton-proton collisions at $ \sqrt{s}=8 $ TeV''}, 
  V.~Khachatryan {\it et al.}  [CMS Collaboration], 
  %arXiv:1412.6302 [hep-ex]
    %{}10.1007/JHEP04(2015)025
JHEP {\bf 1504}025 (2015) %(Dec 192014)
%\href{http://inspirehep.net/record/1335131}{HEP entry}
%25 citations counted in INSPIRE as of 02 Jul 2015


%\cite{CMS:2014dpa}
\item%{CMS:2014dpa}
{\bf ``Searches for supersymmetry based on events with b jets and four W bosons in pp collisions at 8 TeV''}, 
  V.~Khachatryan {\it et al.}  [CMS Collaboration], 
  %arXiv:1412.4109 [hep-ex]
    %{}10.1016/j.physletb.2015.04.002
Phys.\ Lett.\ B {\bf 745}5 (2015) %(Dec 122014)
%\href{http://inspirehep.net/record/1334141}{HEP entry}
%5 citations counted in INSPIRE as of 02 Jul 2015


%\cite{CMS:2014mna}
\item%{CMS:2014mna}
{\bf ``Measurement of the inclusive 3-jet production differential cross section in proton–proton collisions at 7 TeV and determination of the strong coupling constant in the TeV range''}, 
  V.~Khachatryan {\it et al.}  [CMS Collaboration], 
  %arXiv:1412.1633 [hep-ex]
    %{}10.1140/epjc/s10052-015-3376-y
Eur.\ Phys.\ J.\ C {\bf 75}no. 5186 (2015) %(Dec 42014)
%\href{http://inspirehep.net/record/1332746}{HEP entry}
%6 citations counted in INSPIRE as of 02 Jul 2015


%\cite{CMS:2014jea}
\item%{CMS:2014jea}
{\bf ``Measurements of differential and double-differential Drell-Yan cross sections in proton-proton collisions at 8 TeV''}, 
  V.~Khachatryan {\it et al.}  [CMS Collaboration], 
  %arXiv:1412.1115 [hep-ex]
    %{}10.1140/epjc/s10052-015-3364-2
Eur.\ Phys.\ J.\ C {\bf 75}no. 4147 (2015) %(Dec 22014)
%\href{http://inspirehep.net/record/1332509}{HEP entry}
%4 citations counted in INSPIRE as of 02 Jul 2015


%\cite{CMS:2014exa}
\item%{CMS:2014exa}
{\bf ``Search for stealth supersymmetry in events with jetseither photons or leptonsand low missing transverse momentum in pp collisions at 8 TeV''}, 
  V.~Khachatryan {\it et al.}  [CMS Collaboration], 
  %arXiv:1411.7255 [hep-ex]
    %{}10.1016/j.physletb.2015.03.017
Phys.\ Lett.\ B {\bf 743}503 (2015) %(Nov 262014)
%\href{http://inspirehep.net/record/1330294}{HEP entry}
%6 citations counted in INSPIRE as of 02 Jul 2015


%\cite{CMS:2014hka}
\item%{CMS:2014hka}
{\bf ``Search for long-lived particles that decay into final states containing two electrons or two muons in proton-proton collisions at $\sqrt{s} =$ 8 TeV''}, 
  V.~Khachatryan {\it et al.}  [CMS Collaboration], 
  %arXiv:1411.6977 [hep-ex]
    %{}10.1103/PhysRevD.91.052012
Phys.\ Rev.\ D {\bf 91}no. 5052012 (2015) %(Nov 252014)
%\href{http://inspirehep.net/record/1329959}{HEP entry}
%15 citations counted in INSPIRE as of 02 Jul 2015


%\cite{CMS:2014wda}
\item%{CMS:2014wda}
{\bf ``Search for long-lived neutral particles decaying to quark-antiquark pairs in proton-proton collisions at $\sqrt{s} =$ 8 TeV''}, 
  V.~Khachatryan {\it et al.}  [CMS Collaboration], 
  %arXiv:1411.6530 [hep-ex]
    %{}10.1103/PhysRevD.91.012007
Phys.\ Rev.\ D {\bf 91}no. 1012007 (2015) %(Nov 242014)
%\href{http://inspirehep.net/record/1329792}{HEP entry}
%22 citations counted in INSPIRE as of 02 Jul 2015


%\cite{CMS:2014gxa}
\item%{CMS:2014gxa}
{\bf ``Search for disappearing tracks in proton-proton collisions at $ \sqrt{s}=8 $ TeV''}, 
  V.~Khachatryan {\it et al.}  [CMS Collaboration], 
  %arXiv:1411.6006 [hep-ex]
    %{}10.1007/JHEP01(2015)096
JHEP {\bf 1501}096 (2015) %(Nov 212014)
%\href{http://inspirehep.net/record/1329620}{HEP entry}
%20 citations counted in INSPIRE as of 02 Jul 2015


%\cite{CMS:2014yxa}
\item%{CMS:2014yxa}
{\bf ``Measurement of the cross section ratio $\sigma_{ttb\bar{b}}/\sigma_{ttjj}$ in $pp$ collisions at $\sqrt{s} =$ 8 TeV''}, 
  V.~Khachatryan {\it et al.}  [CMS Collaboration], 
  %arXiv:1411.5621 [hep-ex]
    %{}10.1016/j.physletb.2015.04.060
Phys.\ Lett.\ B {\bf 746}132 (2015) %(Nov 202014)
%\href{http://inspirehep.net/record/1328962}{HEP entry}
%1 citations counted in INSPIRE as of 02 Jul 2015


%\cite{CMS:2014xfa}
\item%{CMS:2014xfa}
{\bf ``Observation of the rare $B^0_s\to\mu^+\mu^-$ decay from the combined analysis of CMS and LHCb data''}, 
  V.~Khachatryan {\it et al.}  [CMS and LHCb Collaborations], 
  %arXiv:1411.4413 [hep-ex]
    %{}10.1038/nature14474
Nature {\bf 522}68 (2015) %(Nov 172014)
%\href{http://inspirehep.net/record/1328493}{HEP entry}
%50 citations counted in INSPIRE as of 02 Jul 2015


%\cite{Khachatryan:2014cja}
\item%{Khachatryan:2014cja}
{\bf ``Search for quark contact interactions and extra spatial dimensions using dijet angular distributions in proton–proton collisions at $\sqrt{s} =$ 8 TeV''}, 
  V.~Khachatryan {\it et al.}  [CMS Collaboration], 
  %arXiv:1411.2646 [hep-ex]
    %{}10.1016/j.physletb.2015.04.042
Phys.\ Lett.\ B {\bf 746}79 (2015) %(Nov 102014)
%\href{http://inspirehep.net/record/1327224}{HEP entry}
%9 citations counted in INSPIRE as of 02 Jul 2015


%\cite{Khachatryan:2014gga}
\item%{Khachatryan:2014gga}
{\bf ``Performance of the CMS missing transverse momentum reconstruction in pp data at $\sqrt{s}$ = 8 TeV''}, 
  V.~Khachatryan {\it et al.}  [CMS Collaboration], 
  %arXiv:1411.0511 [physics.ins-det]
    %{}10.1088/1748-0221/10/02/P02006
JINST {\bf 10}no. 02P02006 (2015) %(Nov 32014)
%\href{http://inspirehep.net/record/1325798}{HEP entry}
%5 citations counted in INSPIRE as of 02 Jul 2015


%\cite{Khachatryan:2014waa}
\item%{Khachatryan:2014waa}
{\bf ``Constraints on parton distribution functions and extraction of the strong coupling constant from the inclusive jet cross section in pp collisions at $\sqrt{s} = 7$ $\,\text {TeV}$''}, 
  V.~Khachatryan {\it et al.}  [CMS Collaboration], 
  %arXiv:1410.6765 [hep-ex]
    %{}10.1140/epjc/s10052-015-3499-1
Eur.\ Phys.\ J.\ C {\bf 75}no. 6288 (2015) %(Oct 242014)
%\href{http://inspirehep.net/record/1323630}{HEP entry}
%12 citations counted in INSPIRE as of 02 Jul 2015


%\cite{Khachatryan:2014aep}
\item%{Khachatryan:2014aep}
{\bf ``Search for a standard model$-$like Higgs boson in the $\mu ^{+}\mu ^{-}$ and $e^{+}e^{-}$ decay channels at the LHC''}, 
  V.~Khachatryan {\it et al.}  [CMS Collaboration], 
  %arXiv:1410.6679 [hep-ex]
    %{}10.1016/j.physletb.2015.03.048
Phys.\ Lett.\ B {\bf 744}184 (2015) %(Oct 242014)
%\href{http://inspirehep.net/record/1323624}{HEP entry}
%21 citations counted in INSPIRE as of 02 Jul 2015


%\cite{Khachatryan:2014sta}
\item%{Khachatryan:2014sta}
{\bf ``Study of vector boson scattering and search for new physics in events with two same-sign leptons and two jets''}, 
  V.~Khachatryan {\it et al.}  [CMS Collaboration], 
  %arXiv:1410.6315 [hep-ex]
    %{}10.1103/PhysRevLett.114.051801
Phys.\ Rev.\ Lett.\  {\bf 114}no. 5051801 (2015) %(Oct 232014)
%\href{http://inspirehep.net/record/1323444}{HEP entry}
%10 citations counted in INSPIRE as of 02 Jul 2015


%\cite{Khachatryan:2014nfa}
\item%{Khachatryan:2014nfa}
{\bf ``Measurement of the ratio of the production cross sections times branching fractions of $B_{c}^{\pm} \to J/\psi \pi^{\pm}$ and $B^{\pm} \to J/\psi K^{\pm}$ and $\mathcal{B}(B_{c}^{\pm} \to J/\psi \pi^{\pm}\pi^{\pm}\pi^{\mp})/\mathcal{B}(B_{c}^{\pm} \to J/\psi \pi^{\pm})$ in pp collisions at $\sqrt{s} =$ 7 TeV''}, 
  V.~Khachatryan {\it et al.}  [CMS Collaboration], 
  %arXiv:1410.5729 [hep-ex]
    %{}10.1007/JHEP01(2015)063
JHEP {\bf 1501}063 (2015) %(Oct 212014)
%\href{http://inspirehep.net/record/1323075}{HEP entry}
%2 citations counted in INSPIRE as of 02 Jul 2015


%\cite{Chatrchyan:2014csa}
\item%{Chatrchyan:2014csa}
{\bf ``Study of Z production in PbPb and pp collisions at $ \sqrt{s_{\mathrm{NN}}}=2.76 $ TeV in the dimuon and dielectron decay channels''}, 
  S.~Chatrchyan {\it et al.}  [CMS Collaboration], 
  %arXiv:1410.4825 [nucl-ex]
    %{}10.1007/JHEP03(2015)022
JHEP {\bf 1503}022 (2015) %(Oct 172014)
%\href{http://inspirehep.net/record/1322726}{HEP entry}
%6 citations counted in INSPIRE as of 02 Jul 2015


%\cite{Khachatryan:2014vla}
\item%{Khachatryan:2014vla}
{\bf ``Identification techniques for highly boosted W bosons that decay into hadrons''}, 
  V.~Khachatryan {\it et al.}  [CMS Collaboration], 
  %arXiv:1410.4227 [hep-ex]
    %{}10.1007/JHEP12(2014)017
JHEP {\bf 1412}017 (2014) %(Oct 152014)
%\href{http://inspirehep.net/record/1322563}{HEP entry}
%12 citations counted in INSPIRE as of 02 Jul 2015


%\cite{Khachatryan:2014dea}
\item%{Khachatryan:2014dea}
{\bf ``Measurement of electroweak production of two jets in association with a Z boson in proton-proton collisions at $\sqrt{s}=8\,\text {TeV}$''}, 
  V.~Khachatryan {\it et al.}  [CMS Collaboration], 
  %arXiv:1410.3153 [hep-ex]
    %{}10.1140/epjc/s10052-014-3232-5
Eur.\ Phys.\ J.\ C {\bf 75}no. 266 (2015) %(Oct 122014)
%\href{http://inspirehep.net/record/1321687}{HEP entry}
%7 citations counted in INSPIRE as of 02 Jul 2015


%\cite{Khachatryan:2014jya}
\item%{Khachatryan:2014jya}
{\bf ``Searches for heavy Higgs bosons in two-Higgs-doublet models and for $t\rightarrow ch$ decay using multilepton and diphoton final states in $pp$ collisions at 8 TeV''}, 
  V.~Khachatryan {\it et al.}  [CMS Collaboration], 
  %arXiv:1410.2751 [hep-ex]
    %{}10.1103/PhysRevD.90.112013
Phys.\ Rev.\ D {\bf 90}112013 (2014) %(Oct 102014)
%\href{http://inspirehep.net/record/1321537}{HEP entry}
%18 citations counted in INSPIRE as of 02 Jul 2015


%\cite{Khachatryan:2014bva}
\item%{Khachatryan:2014bva}
{\bf ``Measurement of Prompt $\psi(2S) \to J/\psi$ Yield Ratios in Pb-Pb and $p-p$ Collisions at $\sqrt {s_{NN}}=$ 2.76 TeV''}, 
  V.~Khachatryan {\it et al.}  [CMS Collaboration], 
  %arXiv:1410.1804 [nucl-ex]
    %{}10.1103/PhysRevLett.113.262301
Phys.\ Rev.\ Lett.\  {\bf 113}no. 26262301 (2014) %(Oct 72014)
%\href{http://inspirehep.net/record/1320775}{HEP entry}
%7 citations counted in INSPIRE as of 02 Jul 2015


%\cite{Khachatryan:2014vma}
\item%{Khachatryan:2014vma}
{\bf ``Measurement of the W boson helicity in events with a single reconstructed top quark in pp collisions at $ \sqrt{s}=8 $ TeV''}, 
  V.~Khachatryan {\it et al.}  [CMS Collaboration], 
  %arXiv:1410.1154 [hep-ex]
    %{}10.1007/JHEP01(2015)053
JHEP {\bf 1501}053 (2015) %(Oct 52014)
%\href{http://inspirehep.net/record/1320561}{HEP entry}
%6 citations counted in INSPIRE as of 02 Jul 2015


%\cite{Khachatryan:2014uma}
\item%{Khachatryan:2014uma}
{\bf ``Search for Monotop Signatures in Proton-Proton Collisions at $\sqrt s =$ 8 TeV''}, 
  V.~Khachatryan {\it et al.}  [CMS Collaboration], 
  %arXiv:1410.1149 [hep-ex]
    %{}10.1103/PhysRevLett.114.101801
Phys.\ Rev.\ Lett.\  {\bf 114}no. 10101801 (2015) %(Oct 52014)
%\href{http://inspirehep.net/record/1320560}{HEP entry}
%12 citations counted in INSPIRE as of 02 Jul 2015


%\cite{Khachatryan:2014sca}
\item%{Khachatryan:2014sca}
{\bf ``Search for standard model production of four top quarks in the lepton + jets channel in pp collisions at $\sqrt{s}$ = 8 TeV''}, 
  V.~Khachatryan {\it et al.}  [CMS Collaboration], 
  %arXiv:1409.7339 [hep-ex]
    %{}10.1007/JHEP11(2014)154
JHEP {\bf 1411}154 (2014) %(Sep 252014)
%\href{http://inspirehep.net/record/1318946}{HEP entry}
%10 citations counted in INSPIRE as of 02 Jul 2015


%\cite{Khachatryan:2014ofa}
\item%{Khachatryan:2014ofa}
{\bf ``Measurement of the production cross section ratio sigma(chi[b2](1P)) / sigma(chi[b1](1P)) in pp collisions at sqrt(s) = 8 TeV''}, 
  V.~Khachatryan {\it et al.}  [CMS Collaboration], 
  %arXiv:1409.5761 [hep-ex]
    %{}10.1016/j.physletb.2015.02.048
Phys.\ Lett.\ B {\bf 743}383 (2015) %(Sep 192014)
%\href{http://inspirehep.net/record/1318344}{HEP entry}
%3 citations counted in INSPIRE as of 02 Jul 2015


%\cite{Khachatryan:2014mea}
\item%{Khachatryan:2014mea}
{\bf ``Search for Displaced Supersymmetry in events with an electron and a muon with large impact parameters''}, 
  V.~Khachatryan {\it et al.}  [CMS Collaboration], 
  %arXiv:1409.4789 [hep-ex]
    %{}10.1103/PhysRevLett.114.061801
Phys.\ Rev.\ Lett.\  {\bf 114}no. 6061801 (2015) %(Sep 162014)
%\href{http://inspirehep.net/record/1317640}{HEP entry}
%14 citations counted in INSPIRE as of 02 Jul 2015


%\cite{Khachatryan:2014jra}
\item%{Khachatryan:2014jra}
{\bf ``Long-range two-particle correlations of strange hadrons with charged particles in pPb and PbPb collisions at LHC energies''}, 
  V.~Khachatryan {\it et al.}  [CMS Collaboration], 
  %arXiv:1409.3392 [nucl-ex]
    %{}10.1016/j.physletb.2015.01.034
Phys.\ Lett.\ B {\bf 742}200 (2015) %(Sep 112014)
%\href{http://inspirehep.net/record/1315947}{HEP entry}
%12 citations counted in INSPIRE as of 02 Jul 2015


%\cite{Khachatryan:2014mma}
\item%{Khachatryan:2014mma}
{\bf ``Searches for electroweak neutralino and chargino production in channels with HiggsZand W bosons in pp collisions at 8 TeV''}, 
  V.~Khachatryan {\it et al.}  [CMS Collaboration], 
  %arXiv:1409.3168 [hep-ex]
    %{}10.1103/PhysRevD.90.092007
Phys.\ Rev.\ D {\bf 90}no. 9092007 (2014) %(Sep 102014)
%\href{http://inspirehep.net/record/1315820}{HEP entry}
%29 citations counted in INSPIRE as of 02 Jul 2015


%\cite{Khachatryan:2014rra}
\item%{Khachatryan:2014rra}
{\bf ``Search for dark matterextra dimensionsand unparticles in monojet events in proton–proton collisions at $\sqrt{s} = 8$ TeV''}, 
  V.~Khachatryan {\it et al.}  [CMS Collaboration], 
  %arXiv:1408.3583 [hep-ex]
    %{}10.1140/epjc/s10052-015-3451-4
Eur.\ Phys.\ J.\ C {\bf 75}no. 5235 (2015) %(Aug 152014)
%\href{http://inspirehep.net/record/1311223}{HEP entry}
%82 citations counted in INSPIRE as of 02 Jul 2015


%\cite{Khachatryan:2014wca}
\item%{Khachatryan:2014wca}
{\bf ``Search for neutral MSSM Higgs bosons decaying to a pair of tau leptons in pp collisions''}, 
  V.~Khachatryan {\it et al.}  [CMS Collaboration], 
  %arXiv:1408.3316 [hep-ex]
    %{}10.1007/JHEP10(2014)160
JHEP {\bf 1410}160 (2014) %(Aug 142014)
%\href{http://inspirehep.net/record/1310838}{HEP entry}
%65 citations counted in INSPIRE as of 02 Jul 2015


%\cite{Khachatryan:2014zya}
\item%{Khachatryan:2014zya}
{\bf ``Measurements of jet multiplicity and differential production cross sections of $Z +$ jets events in proton-proton collisions at $\sqrt{s} =$ 7 TeV''}, 
  V.~Khachatryan {\it et al.}  [CMS Collaboration], 
  %arXiv:1408.3104 [hep-ex]
    %{}10.1103/PhysRevD.91.052008
Phys.\ Rev.\ D {\bf 91}no. 5052008 (2015) %(Aug 132014)
%\href{http://inspirehep.net/record/1310737}{HEP entry}
%6 citations counted in INSPIRE as of 02 Jul 2015


%\cite{Khachatryan:2014tva}
\item%{Khachatryan:2014tva}
{\bf ``Search for physics beyond the standard model in final states with a lepton and missing transverse energy in proton-proton collisions at sqrt(s) = 8 TeV''}, 
  V.~Khachatryan {\it et al.}  [CMS Collaboration], 
  %arXiv:1408.2745 [hep-ex]
    %{}10.1103/PhysRevD.91.092005
Phys.\ Rev.\ D {\bf 91}no. 9092005 (2015) %(Aug 122014)
%\href{http://inspirehep.net/record/1310653}{HEP entry}
%40 citations counted in INSPIRE as of 02 Jul 2015


%\cite{Khachatryan:2014qaa}
\item%{Khachatryan:2014qaa}
{\bf ``Search for the associated production of the Higgs boson with a top-quark pair''}, 
  V.~Khachatryan {\it et al.}  [CMS Collaboration], 
  %arXiv:1408.1682 [hep-ex]
    %{}10.1007/JHEP09(2014)08710.1007/JHEP10(2014)106
JHEP {\bf 1409}087 (2014)[JHEP {\bf 1410}106 (2014)] %(Aug 72014)
%\href{http://inspirehep.net/record/1310104}{HEP entry}
%47 citations counted in INSPIRE as of 02 Jul 2015


%\cite{Khachatryan:2014ura}
\item%{Khachatryan:2014ura}
{\bf ``Search for pair production of third-generation scalar leptoquarks and top squarks in proton-proton collisions at sqrt(s) = 8 TeV''}, 
  V.~Khachatryan {\it et al.}  [CMS Collaboration], 
  %arXiv:1408.0806 [hep-ex]
    %{}10.1016/j.physletb.2014.10.063
Phys.\ Lett.\ B {\bf 739}229 (2014) %(Aug 42014)
%\href{http://inspirehep.net/record/1309874}{HEP entry}
%25 citations counted in INSPIRE as of 02 Jul 2015


%\cite{Khachatryan:2014loa}
\item%{Khachatryan:2014loa}
{\bf ``Measurement of the $t \bar t$ production cross section in $pp$ collisions at $\sqrt s = 8$ TeV in dilepton final states containing one $\tau$ lepton''}, 
  V.~Khachatryan {\it et al.}  [CMS Collaboration], 
  %arXiv:1407.6643 [hep-ex]
    %{}10.1016/j.physletb.2014.10.032
Phys.\ Lett.\ B {\bf 739}23 (2014) %(Jul 242014)
%\href{http://inspirehep.net/record/1307759}{HEP entry}
%6 citations counted in INSPIRE as of 02 Jul 2015


%\cite{Khachatryan:2014dka}
\item%{Khachatryan:2014dka}
{\bf ``Search for heavy neutrinos and $\mathrm {W}$ bosons with right-handed couplings in proton-proton collisions at $\sqrt{s} = 8\,\text {TeV} $''}, 
  V.~Khachatryan {\it et al.}  [CMS Collaboration], 
  %arXiv:1407.3683 [hep-ex]
    %{}10.1140/epjc/s10052-014-3149-z
Eur.\ Phys.\ J.\ C {\bf 74}no. 113149 (2014) %(Jul 142014)
%\href{http://inspirehep.net/record/1306295}{HEP entry}
%58 citations counted in INSPIRE as of 02 Jul 2015


%\cite{Khachatryan:2014xja}
\item%{Khachatryan:2014xja}
{\bf ``Search for new resonances decaying via WZ to leptons in proton-proton collisions at $\sqrt s =$ 8 TeV''}, 
  V.~Khachatryan {\it et al.}  [CMS Collaboration], 
  %arXiv:1407.3476 [hep-ex]
    %{}10.1016/j.physletb.2014.11.026
Phys.\ Lett.\ B {\bf 740}83 (2015) %(Jul 132014)
%\href{http://inspirehep.net/record/1306289}{HEP entry}
%22 citations counted in INSPIRE as of 02 Jul 2015


%\cite{Khachatryan:2014ika}
\item%{Khachatryan:2014ika}
{\bf ``Study of hadronic event-shape variables in multijet final states in pp collisions at sqrt(s) = 7 TeV''}, 
  V.~Khachatryan {\it et al.}  [CMS Collaboration], 
  %arXiv:1407.2856 [hep-ex]
    %{}10.1007/JHEP10(2014)087
JHEP {\bf 1410}87 (2014) %(Jul 102014)
%\href{http://inspirehep.net/record/1305624}{HEP entry}
%4 citations counted in INSPIRE as of 02 Jul 2015


%\cite{Khachatryan:2014ira}
\item%{Khachatryan:2014ira}
{\bf ``Observation of the diphoton decay of the Higgs boson and measurement of its properties''}, 
  V.~Khachatryan {\it et al.}  [CMS Collaboration], 
  %arXiv:1407.0558 [hep-ex]
    %{}10.1140/epjc/s10052-014-3076-z
Eur.\ Phys.\ J.\ C {\bf 74}no. 103076 (2014) %(Jul 22014)
%\href{http://inspirehep.net/record/1304454}{HEP entry}
%162 citations counted in INSPIRE as of 02 Jul 2015


%\cite{Khachatryan:2014ewa}
\item%{Khachatryan:2014ewa}
{\bf ``Measurement of top quark-antiquark pair production in association with a W or Z boson in pp collisions at $\sqrt{s} = 8$ $\,\text {TeV}$''}, 
  V.~Khachatryan {\it et al.}  [CMS Collaboration], 
  %arXiv:1406.7830 [hep-ex]
    %{}10.1140/epjc/s10052-014-3060-7
Eur.\ Phys.\ J.\ C {\bf 74}no. 93060 (2014) %(Jun 302014)
%\href{http://inspirehep.net/record/1303904}{HEP entry}
%22 citations counted in INSPIRE as of 02 Jul 2015


%\cite{Khachatryan:2014uva}
\item%{Khachatryan:2014uva}
{\bf ``Differential cross section measurements for the production of a W boson in association with jets in proton–proton collisions at $\sqrt s=7$ TeV''}, 
  V.~Khachatryan {\it et al.}  [CMS Collaboration], 
  %arXiv:1406.7533 [hep-ex]
    %{}10.1016/j.physletb.2014.12.003
Phys.\ Lett.\ B {\bf 741}12 (2015) %(Jun 292014)
%\href{http://inspirehep.net/record/1303894}{HEP entry}
%16 citations counted in INSPIRE as of 02 Jul 2015


%\cite{Khachatryan:2014aka}
\item%{Khachatryan:2014aka}
{\bf ``Search for excited quarks in the $\gamma +$jet final state in proton–proton collisions at $\sqrt s=8$ TeV''}, 
  V.~Khachatryan {\it et al.}  [CMS Collaboration], 
  %arXiv:1406.5171 [hep-ex]
    %{}10.1016/j.physletb.2014.09.048
Phys.\ Lett.\ B {\bf 738}274 (2014) %(Jun 192014)
%\href{http://inspirehep.net/record/1301560}{HEP entry}
%9 citations counted in INSPIRE as of 02 Jul 2015


%\cite{Chatrchyan:2014ava}
\item%{Chatrchyan:2014ava}
{\bf ``Measurement of jet fragmentation in PbPb and pp collisions at $\sqrt{s_{NN}}=2.76$ TeV''}, 
  S.~Chatrchyan {\it et al.}  [CMS Collaboration], 
  %arXiv:1406.0932 [nucl-ex]
    %{}10.1103/PhysRevC.90.024908
Phys.\ Rev.\ C {\bf 90}no. 2024908 (2014) %(Jun 32014)
%\href{http://inspirehep.net/record/1299142}{HEP entry}
%24 citations counted in INSPIRE as of 02 Jul 2015


%\cite{Khachatryan:2014iia}
\item%{Khachatryan:2014iia}
{\bf ``Measurement of prompt $J/\psi$ pair production in pp collisions at $ \sqrt{s} $ = 7 Tev''}, 
  V.~Khachatryan {\it et al.}  [CMS Collaboration], 
  %arXiv:1406.0484 [hep-ex]
    %{}10.1007/JHEP09(2014)094
JHEP {\bf 1409}094 (2014) %(Jun 22014)
%\href{http://inspirehep.net/record/1298812}{HEP entry}
%9 citations counted in INSPIRE as of 02 Jul 2015


%\cite{Chatrchyan:2014gia}
\item%{Chatrchyan:2014gia}
{\bf ``Measurement of the ratio of inclusive jet cross sections using the anti-$k_T$ algorithm with radius parameters R=0.5 and 0.7 in pp collisions at $\sqrt{s}=7$ TeV''}, 
  S.~Chatrchyan {\it et al.}  [CMS Collaboration], 
  %arXiv:1406.0324 [hep-ex]
    %{}10.1103/PhysRevD.90.072006
Phys.\ Rev.\ D {\bf 90}no. 7072006 (2014) %(Jun 22014)
%\href{http://inspirehep.net/record/1298810}{HEP entry}
%6 citations counted in INSPIRE as of 02 Jul 2015


%\cite{CMS:2014xja}
\item%{CMS:2014xja}
{\bf ``Measurement of the $pp \to ZZ$ production cross section and constraints on anomalous triple gauge couplings in four-lepton final states at $\sqrt s=$8 TeV''}, 
  V.~Khachatryan {\it et al.}  [CMS Collaboration], 
  %arXiv:1406.0113 [hep-ex]
    %{}10.1016/j.physletb.2014.11.059
Phys.\ Lett.\ B {\bf 740}250 (2015) %(May 312014)
%\href{http://inspirehep.net/record/1298807}{HEP entry}
%19 citations counted in INSPIRE as of 02 Jul 2015


%\cite{Khachatryan:2014uwa}
\item%{Khachatryan:2014uwa}
{\bf ``Search for jet extinction in the inclusive jet-$p_t$ spectrum from proton-proton collisions at $\sqrt s =$ 8 TeV''}, 
  V.~Khachatryan {\it et al.}  [CMS Collaboration], 
  %arXiv:1405.7653 [hep-ex]
    %{}10.1103/PhysRevD.90.032005 
Phys.\ Rev.\ D {\bf 90}no. 3032005 (2014) %(May 292014)
%\href{http://inspirehep.net/record/1298512}{HEP entry}
%2 citations counted in INSPIRE as of 02 Jul 2015


%\cite{Khachatryan:2014qwa}
\item%{Khachatryan:2014qwa}
{\bf ``Searches for electroweak production of charginos, neutralinos and sleptons decaying to leptons and WZ and Higgs bosons in pp collisions at 8 TeV''}, 
  V.~Khachatryan {\it et al.}  [CMS Collaboration], 
  %arXiv:1405.7570 [hep-ex]
    %{}10.1140/epjc/s10052-014-3036-7
Eur.\ Phys.\ J.\ C {\bf 74}no. 93036 (2014) %(May 292014)
%\href{http://inspirehep.net/record/1298508}{HEP entry}
%98 citations counted in INSPIRE as of 02 Jul 2015


%\cite{Chatrchyan:2014fsa}
\item%{Chatrchyan:2014fsa}
{\bf ``Measurement of differential cross sections for the production of a pair of isolated photons in pp collisions at $\sqrt{s}=7\,\text {TeV} $''}, 
  S.~Chatrchyan {\it et al.}  [CMS Collaboration], 
  %arXiv:1405.7225 [hep-ex]
    %{}10.1140/epjc/s10052-014-3129-3
Eur.\ Phys.\ J.\ C {\bf 74}no. 113129 (2014) %(May 282014)
%\href{http://inspirehep.net/record/1298393}{HEP entry}
%10 citations counted in INSPIRE as of 02 Jul 2015


%\cite{Chatrchyan:2014fea}
\item%{Chatrchyan:2014fea}
{\bf ``Description and performance of track and primary-vertex reconstruction with the CMS tracker''}, 
  S.~Chatrchyan {\it et al.}  [CMS Collaboration], 
  %arXiv:1405.6569 [physics.ins-det]
    %{}10.1088/1748-0221/9/10/P10009
JINST {\bf 9}no. 10P10009 (2014) %(May 262014)
%\href{http://inspirehep.net/record/1298029}{HEP entry}
%56 citations counted in INSPIRE as of 02 Jul 2015


%\cite{Chatrchyan:2014goa}
\item%{Chatrchyan:2014goa}
{\bf ``Search for supersymmetry with razor variables in pp collisions at $\sqrt{s}$=7 TeV''}, 
  S.~Chatrchyan {\it et al.}  [CMS Collaboration], 
  %arXiv:1405.3961 [hep-ex]
    %{}10.1103/PhysRevD.90.112001
Phys.\ Rev.\ D {\bf 90}no. 11112001 (2014) %(May 152014)
%\href{http://inspirehep.net/record/1296262}{HEP entry}
%10 citations counted in INSPIRE as of 02 Jul 2015


%\cite{Khachatryan:2014doa}
\item%{Khachatryan:2014doa}
{\bf ``Search for top-squark pairs decaying into Higgs or Z bosons in pp collisions at $\sqrt{s}$=8 TeV''}, 
  V.~Khachatryan {\it et al.}  [CMS Collaboration], 
  %arXiv:1405.3886 [hep-ex]
    %{}10.1016/j.physletb.2014.07.053
Phys.\ Lett.\ B {\bf 736}371 (2014) %(May 152014)
%\href{http://inspirehep.net/record/1296259}{HEP entry}
%34 citations counted in INSPIRE as of 02 Jul 2015


%\cite{Khachatryan:2014iha}
\item%{Khachatryan:2014iha}
{\bf ``Constraints on the Higgs boson width from off-shell production and decay to Z-boson pairs''}, 
  V.~Khachatryan {\it et al.}  [CMS Collaboration], 
  %arXiv:1405.3455 [hep-ex]
    %{}10.1016/j.physletb.2014.06.077
Phys.\ Lett.\ B {\bf 736}64 (2014) %(May 142014)
%\href{http://inspirehep.net/record/1296082}{HEP entry}
%72 citations counted in INSPIRE as of 02 Jul 2015


%\cite{Khachatryan:2014gha}
\item%{Khachatryan:2014gha}
{\bf ``Search for massive resonances decaying into pairs of boosted bosons in semi-leptonic final states at $\sqrt{s} =$ 8 TeV''}, 
  V.~Khachatryan {\it et al.}  [CMS Collaboration], 
  %arXiv:1405.3447 [hep-ex]
    %{}10.1007/JHEP08(2014)174
JHEP {\bf 1408}174 (2014) %(May 142014)
%\href{http://inspirehep.net/record/1296080}{HEP entry}
%41 citations counted in INSPIRE as of 02 Jul 2015


%\cite{Khachatryan:2014hpa}
\item%{Khachatryan:2014hpa}
{\bf ``Search for massive resonances in dijet systems containing jets tagged as W or Z boson decays in pp collisions at $ \sqrt{s} $ = 8 TeV''}, 
  V.~Khachatryan {\it et al.}  [CMS Collaboration], 
  %arXiv:1405.1994 [hep-ex]
    %{}10.1007/JHEP08(2014)173
JHEP {\bf 1408}173 (2014) %(May 82014)
%\href{http://inspirehep.net/record/1294937}{HEP entry}
%41 citations counted in INSPIRE as of 02 Jul 2015


%\cite{Chatrchyan:2014qka}
\item%{Chatrchyan:2014qka}
{\bf ``Measurement of pseudorapidity distributions of charged particles in proton-proton collisions at $\sqrt{s}$ = 8 TeV by the CMS and TOTEM experiments''}, 
  S.~Chatrchyan {\it et al.}  [CMS and TOTEM Collaborations], 
  %arXiv:1405.0722 [hep-ex]
    %{}10.1140/epjc/s10052-014-3053-6
Eur.\ Phys.\ J.\ C {\bf 74}no. 103053 (2014) %(May 42014)
%\href{http://inspirehep.net/record/1294140}{HEP entry}
%9 citations counted in INSPIRE as of 02 Jul 2015


%\cite{Chatrchyan:2014aea}
\item%{Chatrchyan:2014aea}
{\bf ``Search for anomalous production of events with three or more leptons in $pp$ collisions at $\sqrt(s) =$ 8 TeV''}, 
  S.~Chatrchyan {\it et al.}  [CMS Collaboration], 
  %arXiv:1404.5801 [hep-ex]
    %{}10.1103/PhysRevD.90.032006
Phys.\ Rev.\ D {\bf 90}032006 (2014) %(Apr 232014)
%\href{http://inspirehep.net/record/1291940}{HEP entry}
%45 citations counted in INSPIRE as of 02 Jul 2015


%\cite{Chatrchyan:2014bza}
\item%{Chatrchyan:2014bza}
{\bf ``Search for $WW \gamma$ and $WZ \gamma$ production and constraints on anomalous quartic gauge couplings in $pp$ collisions at $\sqrt s =$ 8 TeV''}, 
  S.~Chatrchyan {\it et al.}  [CMS Collaboration], 
  %arXiv:1404.4619 [hep-ex]
    %{}10.1103/PhysRevD.90.032008
Phys.\ Rev.\ D {\bf 90}no. 3032008 (2014) %(Apr 172014)
%\href{http://inspirehep.net/record/1291135}{HEP entry}
%18 citations counted in INSPIRE as of 02 Jul 2015


%\cite{Chatrchyan:2014gma}
\item%{Chatrchyan:2014gma}
{\bf ``Measurement of jet multiplicity distributions in $\mathrm {t}\overline{\mathrm {t}}$ production in pp collisions at $\sqrt{s} = 7\,\text {TeV} $''}, 
  S.~Chatrchyan {\it et al.}  [CMS Collaboration], 
  %arXiv:1404.3171 [hep-ex]
    %{}10.1140/epjc/s10052-014-3014-010.1140/epjc/s10052-015-3437-2
Eur.\ Phys.\ J.\ C {\bf 74}3014 (2015)[Eur.\ Phys.\ J.\ C {\bf 75}no. 5216 (2015)] %(Apr 112014)
%\href{http://inspirehep.net/record/1290126}{HEP entry}
%12 citations counted in INSPIRE as of 02 Jul 2015


%\cite{Khachatryan:2014nda}
\item%{Khachatryan:2014nda}
{\bf ``Measurement of the ratio $B(t \to Wb)/B(t \to Wq)$ in pp collisions at $\sqrt{s}$ = 8 TeV''}, 
  V.~Khachatryan {\it et al.}  [CMS Collaboration], 
  %arXiv:1404.2292 [hep-ex]
    %{}10.1016/j.physletb.2014.06.076
Phys.\ Lett.\ B {\bf 736}33 (2014) %(Apr 82014)
%\href{http://inspirehep.net/record/1289223}{HEP entry}
%15 citations counted in INSPIRE as of 02 Jul 2015


%\cite{Chatrchyan:2014tja}
\item%{Chatrchyan:2014tja}
{\bf ``Search for invisible decays of Higgs bosons in the vector boson fusion and associated ZH production modes''}, 
  S.~Chatrchyan {\it et al.}  [CMS Collaboration], 
  %arXiv:1404.1344 [hep-ex]
    %{}10.1140/epjc/s10052-014-2980-6
Eur.\ Phys.\ J.\ C {\bf 74}2980 (2014) %(Apr 42014)
%\href{http://inspirehep.net/record/1288709}{HEP entry}
%94 citations counted in INSPIRE as of 02 Jul 2015


%\cite{Khachatryan:2014iya}
\item%{Khachatryan:2014iya}
{\bf ``Measurement of the t-channel single-top-quark production cross section and of the $\mid V_{tb} \mid$ CKM matrix element in pp collisions at $\sqrt{s}$= 8 TeV''}, 
  V.~Khachatryan {\it et al.}  [CMS Collaboration], 
  %arXiv:1403.7366 [hep-ex]
    %{}10.1007/JHEP06(2014)090
JHEP {\bf 1406}090 (2014) %(Mar 282014)
%\href{http://inspirehep.net/record/1287736}{HEP entry}
%42 citations counted in INSPIRE as of 02 Jul 2015


%\cite{Chatrchyan:2014aqa}
\item%{Chatrchyan:2014aqa}
{\bf ``Measurement of WZ and ZZ production in pp collisions at $\sqrt{s} = 8\,\text {TeV} $ in final states with b-tagged jets''}, 
  S.~Chatrchyan {\it et al.}  [CMS Collaboration], 
  %arXiv:1403.3047 [hep-ex]
    %{}10.1140/epjc/s10052-014-2973-5
Eur.\ Phys.\ J.\ C {\bf 74}no. 82973 (2014) %(Mar 122014)
%\href{http://inspirehep.net/record/1285492}{HEP entry}
%16 citations counted in INSPIRE as of 02 Jul 2015


%\cite{Chatrchyan:2014wfa}
\item%{Chatrchyan:2014wfa}
{\bf ``Alignment of the CMS tracker with LHC and cosmic ray data''}, 
  S.~Chatrchyan {\it et al.}  [CMS Collaboration], 
  %arXiv:1403.2286 [physics.ins-det]
    %{}10.1088/1748-0221/9/06/P06009
JINST {\bf 9}P06009 (2014) %(Mar 102014)
%\href{http://inspirehep.net/record/1285228}{HEP entry}
%4 citations counted in INSPIRE as of 02 Jul 2015


%\cite{Chatrchyan:2014lfa}
\item%{Chatrchyan:2014lfa}
{\bf ``Search for new physics in the multijet and missing transverse momentum final state in proton-proton collisions at $\sqrt{s}$= 8 TeV''}, 
  S.~Chatrchyan {\it et al.}  [CMS Collaboration], 
  %arXiv:1402.4770 [hep-ex]
    %{}10.1007/JHEP06(2014)055
JHEP {\bf 1406}055 (2014) %(Feb 192014)
%\href{http://inspirehep.net/record/1281837}{HEP entry}
%140 citations counted in INSPIRE as of 02 Jul 2015


%\cite{Chatrchyan:2014yta}
\item%{Chatrchyan:2014yta}
{\bf ``Measurements of the $t\bar{t}$ charge asymmetry using the dilepton decay channel in pp collisions at $\sqrt{s} =$ 7 TeV''}, 
  S.~Chatrchyan {\it et al.}  [CMS Collaboration], 
  %arXiv:1402.3803 [hep-ex]
    %{}10.1007/JHEP04(2014)191
JHEP {\bf 1404}191 (2014) %(Feb 162014)
%\href{http://inspirehep.net/record/1281538}{HEP entry}
%23 citations counted in INSPIRE as of 02 Jul 2015


%\cite{Chatrchyan:2014koa}
\item%{Chatrchyan:2014koa}
{\bf ``Search for W' $\to $ tb decays in the lepton + jets final state in pp collisions at $\sqrt{s}$ = 8 TeV''}, 
  S.~Chatrchyan {\it et al.}  [CMS Collaboration], 
  %arXiv:1402.2176 [hep-ex]
    %{}10.1007/JHEP05(2014)108
JHEP {\bf 1405}108 (2014) %(Feb 102014)
%\href{http://inspirehep.net/record/1280718}{HEP entry}
%25 citations counted in INSPIRE as of 02 Jul 2015


%\cite{Chatrchyan:2014dha}
\item%{Chatrchyan:2014dha}
{\bf ``Measurement of the production cross sections for a Z boson and one or more b jets in pp collisions at sqrt(s) = 7 TeV''}, 
  S.~Chatrchyan {\it et al.}  [CMS Collaboration], 
  %arXiv:1402.1521 [hep-ex]
    %{}10.1007/JHEP06(2014)120
JHEP {\bf 1406}120 (2014) %(Feb 62014)
%\href{http://inspirehep.net/record/1280529}{HEP entry}
%21 citations counted in INSPIRE as of 02 Jul 2015


%\cite{Chatrchyan:2014mua}
\item%{Chatrchyan:2014mua}
{\bf ``Measurement of inclusive W and Z boson production cross sections in pp collisions at $\sqrt{s}$ = 8 TeV''}, 
  S.~Chatrchyan {\it et al.}  [CMS Collaboration], 
  %arXiv:1402.0923 [hep-ex]
    %{}10.1103/PhysRevLett.112.191802
Phys.\ Rev.\ Lett.\  {\bf 112}191802 (2014) %(Feb 42014)
%\href{http://inspirehep.net/record/1280200}{HEP entry}
%23 citations counted in INSPIRE as of 02 Jul 2015


%\cite{Chatrchyan:2014vua}
\item%{Chatrchyan:2014vua}
{\bf ``Evidence for the direct decay of the 125 GeV Higgs boson to fermions''}, 
  S.~Chatrchyan {\it et al.}  [CMS Collaboration], 
  %arXiv:1401.6527 [hep-ex]
    %{}10.1038/nphys3005
Nature Phys.\  {\bf 10}557 (2014) %(Jan 252014)
%\href{http://inspirehep.net/record/1278857}{HEP entry}
%81 citations counted in INSPIRE as of 02 Jul 2015


%\cite{Chatrchyan:2014nva}
\item%{Chatrchyan:2014nva}
{\bf ``Evidence for the 125 GeV Higgs boson decaying to a pair of $\tau$ leptons''}, 
  S.~Chatrchyan {\it et al.}  [CMS Collaboration], 
  %arXiv:1401.5041 [hep-ex]
    %{}10.1007/JHEP05(2014)104
JHEP {\bf 1405}104 (2014) %(Jan 202014)
%\href{http://inspirehep.net/record/1278199}{HEP entry}
%141 citations counted in INSPIRE as of 02 Jul 2015


%\cite{Chatrchyan:2014hqa}
\item%{Chatrchyan:2014hqa}
{\bf ``Studies of dijet transverse momentum balance and pseudorapidity distributions in pPb collisions at $\sqrt{s_{\mathrm{NN}}} = 5.02$ $\,\text {TeV}$''}, 
  S.~Chatrchyan {\it et al.}  [CMS Collaboration], 
  %arXiv:1401.4433 [nucl-ex]
    %{}10.1140/epjc/s10052-014-2951-y
Eur.\ Phys.\ J.\ C {\bf 74}no. 72951 (2014) %(Jan 172014)
%\href{http://inspirehep.net/record/1278063}{HEP entry}
%35 citations counted in INSPIRE as of 02 Jul 2015


%\cite{Chatrchyan:2014tua}
\item%{Chatrchyan:2014tua}
{\bf ``Observation of the associated production of a single top quark and a $W$ boson in $pp$ collisions at $\sqrt s = $8 TeV''}, 
  S.~Chatrchyan {\it et al.}  [CMS Collaboration], 
  %arXiv:1401.2942 [hep-ex]
    %{}10.1103/PhysRevLett.112.231802
Phys.\ Rev.\ Lett.\  {\bf 112}no. 23231802 (2014) %(Jan 132014)
%\href{http://inspirehep.net/record/1276827}{HEP entry}
%49 citations counted in INSPIRE as of 02 Jul 2015


%\cite{Chatrchyan:2013faa}
\item%{Chatrchyan:2013faa}
{\bf ``Measurement of the $t \bar{t}$ production cross section in the dilepton channel in pp collisions at $\sqrt{s}$ = 8 TeV''}, 
  S.~Chatrchyan {\it et al.}  [CMS Collaboration], 
  %arXiv:1312.7582 [hep-ex]arXiv:1312.7582
    %{}10.1007/JHEP02(2014)02410.1007/JHEP02(2014)102
JHEP {\bf 1402}024 (2014)[JHEP {\bf 1402}102 (2014)] %(Dec 292013)
%\href{http://inspirehep.net/record/1275617}{HEP entry}
%55 citations counted in INSPIRE as of 02 Jul 2015


%\cite{Chatrchyan:2013uza}
\item%{Chatrchyan:2013uza}
{\bf ``Measurement of the production cross section for a W boson and two b jets in pp collisions at $\sqrt{s}$=7 TeV''}, 
  S.~Chatrchyan {\it et al.}  [CMS Collaboration], 
  %arXiv:1312.6608 [hep-ex]
    %{}10.1016/j.physletb.2014.06.041
Phys.\ Lett.\ B {\bf 735}204 (2014) %(Dec 232013)
%\href{http://inspirehep.net/record/1273578}{HEP entry}
%15 citations counted in INSPIRE as of 02 Jul 2015


%\cite{Chatrchyan:2013qza}
\item%{Chatrchyan:2013qza}
{\bf ``Measurement of four-jet production in proton-proton collisions at $\sqrt{s}=7$ TeV''}, 
  S.~Chatrchyan {\it et al.}  [CMS Collaboration], 
  %arXiv:1312.6440 [hep-ex]
    %{}10.1103/PhysRevD.89.092010
Phys.\ Rev.\ D {\bf 89}no. 9092010 (2014) %(Dec 222013)
%\href{http://inspirehep.net/record/1273574}{HEP entry}
%17 citations counted in INSPIRE as of 02 Jul 2015


%\cite{Chatrchyan:2013nza}
\item%{Chatrchyan:2013nza}
{\bf ``Event activity dependence of Y(nS) production in $\sqrt{s_{NN}}$=5.02 TeV pPb and $\sqrt{s}$=2.76 TeV pp collisions''}, 
  S.~Chatrchyan {\it et al.}  [CMS Collaboration], 
  %arXiv:1312.6300 [nucl-ex]
    %{}10.1007/JHEP04(2014)103
JHEP {\bf 1404}103 (2014) %(Dec 212013)
%\href{http://inspirehep.net/record/1273571}{HEP entry}
%23 citations counted in INSPIRE as of 02 Jul 2015


%\cite{Chatrchyan:2013mza}
\item%{Chatrchyan:2013mza}
{\bf ``Measurement of the muon charge asymmetry in inclusive $pp \to W+X$ production at $\sqrt s =$ 7 TeV and an improved determination of light parton distribution functions''}, 
  S.~Chatrchyan {\it et al.}  [CMS Collaboration], 
  %arXiv:1312.6283 [hep-ex]
    %{}10.1103/PhysRevD.90.032004
Phys.\ Rev.\ D {\bf 90}no. 3032004 (2014) %(Dec 212013)
%\href{http://inspirehep.net/record/1273570}{HEP entry}
%38 citations counted in INSPIRE as of 02 Jul 2015


%\cite{Chatrchyan:2013xxa}
\item%{Chatrchyan:2013xxa}
{\bf ``Study of double parton scattering using W + 2-jet events in proton-proton collisions at $\sqrt{s}$ = 7 TeV''}, 
  S.~Chatrchyan {\it et al.}  [CMS Collaboration], 
  %arXiv:1312.5729 [hep-ex]
    %{}10.1007/JHEP03(2014)032
JHEP {\bf 1403}032 (2014) %(Dec 192013)
%\href{http://inspirehep.net/record/1272853}{HEP entry}
%33 citations counted in INSPIRE as of 02 Jul 2015


%\cite{Chatrchyan:2013mxa}
\item%{Chatrchyan:2013mxa}
{\bf ``Measurement of the properties of a Higgs boson in the four-lepton final state''}, 
  S.~Chatrchyan {\it et al.}  [CMS Collaboration], 
  %arXiv:1312.5353 [hep-ex]
    %{}10.1103/PhysRevD.89.092007
Phys.\ Rev.\ D {\bf 89}no. 9092007 (2014) %(Dec 182013)
%\href{http://inspirehep.net/record/1272842}{HEP entry}
%262 citations counted in INSPIRE as of 02 Jul 2015


%\cite{Chatrchyan:2013exa}
\item%{Chatrchyan:2013exa}
{\bf ``Evidence of b-Jet Quenching in PbPb Collisions at $\sqrt{s_{NN}}=2.76$ TeV''}, 
  S.~Chatrchyan {\it et al.}  [CMS Collaboration], 
  %arXiv:1312.4198 [nucl-ex]
    %{}10.1103/PhysRevLett.113.132301
Phys.\ Rev.\ Lett.\  {\bf 113}no. 13132301 (2014) %(Dec 152013)
%\href{http://inspirehep.net/record/1269454}{HEP entry}
%28 citations counted in INSPIRE as of 02 Jul 2015


%\cite{Chatrchyan:2013nwa}
\item%{Chatrchyan:2013nwa}
{\bf ``Search for Flavor-Changing Neutral Currents in Top-Quark Decays $t \to Zq$ in $pp$ Collisions at $\sqrt{s}=8$ TeV''}, 
  S.~Chatrchyan {\it et al.}  [CMS Collaboration], 
  %arXiv:1312.4194 [hep-ex]
    %{}10.1103/PhysRevLett.112.171802
Phys.\ Rev.\ Lett.\  {\bf 112}no. 17171802 (2014) %(Dec 152013)
%\href{http://inspirehep.net/record/1269437}{HEP entry}
%29 citations counted in INSPIRE as of 02 Jul 2015


%\cite{Chatrchyan:2013mya}
\item%{Chatrchyan:2013mya}
{\bf ``Search for top squark and higgsino production using diphoton Higgs boson decays''}, 
  S.~Chatrchyan {\it et al.}  [CMS Collaboration], 
  %arXiv:1312.3310 [hep-ex]
    %{}10.1103/PhysRevLett.112.161802
Phys.\ Rev.\ Lett.\  {\bf 112}161802 (2014) %(Dec 112013)
%\href{http://inspirehep.net/record/1268812}{HEP entry}
%37 citations counted in INSPIRE as of 02 Jul 2015


%\cite{Chatrchyan:2013wfa}
\item%{Chatrchyan:2013wfa}
{\bf ``Search for top-quark partners with charge 5/3 in the same-sign dilepton final state''}, 
  S.~Chatrchyan {\it et al.}  [CMS Collaboration], 
  %arXiv:1312.2391 [hep-ex]
    %{}10.1103/PhysRevLett.112.171801
Phys.\ Rev.\ Lett.\  {\bf 112}no. 17171801 (2014) %(Dec 92013)
%\href{http://inspirehep.net/record/1268328}{HEP entry}
%45 citations counted in INSPIRE as of 02 Jul 2015


%\cite{CMS:2013bza}
\item%{CMS:2013bza}
{\bf ``Studies of azimuthal dihadron correlations in ultra-central PbPb collisions at $\sqrt{s_{NN}} =$ 2.76 TeV''}, 
  S.~Chatrchyan {\it et al.}  [CMS Collaboration], 
  %arXiv:1312.1845 [nucl-ex]
    %{}10.1007/JHEP02(2014)088
JHEP {\bf 1402}088 (2014) %(Dec 62013)
%\href{http://inspirehep.net/record/1268151}{HEP entry}
%34 citations counted in INSPIRE as of 02 Jul 2015


%\cite{Chatrchyan:2013iaa}
\item%{Chatrchyan:2013iaa}
{\bf ``Measurement of Higgs boson production and properties in the WW decay channel with leptonic final states''}, 
  S.~Chatrchyan {\it et al.}  [CMS Collaboration], 
  %arXiv:1312.1129 [hep-ex]
    %{}10.1007/JHEP01(2014)096
JHEP {\bf 1401}096 (2014) %(Dec 42013)
%\href{http://inspirehep.net/record/1267508}{HEP entry}
%166 citations counted in INSPIRE as of 02 Jul 2015


%\cite{Chatrchyan:2013uxa}
\item%{Chatrchyan:2013uxa}
{\bf ``Inclusive search for a vector-like T quark with charge $\frac{2}{3}$ in pp collisions at $\sqrt{s}$ = 8 TeV''}, 
  S.~Chatrchyan {\it et al.}  [CMS Collaboration], 
  %arXiv:1311.7667 [hep-ex]
    %{}10.1016/j.physletb.2014.01.006
Phys.\ Lett.\ B {\bf 729}149 (2014) %(Nov 292013)
%\href{http://inspirehep.net/record/1266766}{HEP entry}
%96 citations counted in INSPIRE as of 02 Jul 2015


%\cite{Chatrchyan:2013fea}
\item%{Chatrchyan:2013fea}
{\bf ``Search for new physics in events with same-sign dileptons and jets in pp collisions at $\sqrt{s}$ = 8 TeV''}, 
  S.~Chatrchyan {\it et al.}  [CMS Collaboration], 
  %arXiv:1311.6736arXiv:1311.6736 [hep-ex]
    %{}10.1007/JHEP01(2015)01410.1007/JHEP01(2014)163
JHEP {\bf 1401}163 (2014)[JHEP {\bf 1501}014 (2015)] %(Nov 262013)
%\href{http://inspirehep.net/record/1266257}{HEP entry}
%69 citations counted in INSPIRE as of 02 Jul 2015


%\cite{Chatrchyan:2013mwa}
\item%{Chatrchyan:2013mwa}
{\bf ``Measurement of the triple-differential cross section for photon+jets production in proton-proton collisions at $\sqrt{s}$=7 TeV''}, 
  S.~Chatrchyan {\it et al.}  [CMS Collaboration], 
  %arXiv:1311.6141 [hep-ex]
    %{}10.1007/JHEP06(2014)009
JHEP {\bf 1406}009 (2014) %(Nov 242013)
%\href{http://inspirehep.net/record/1266056}{HEP entry}
%14 citations counted in INSPIRE as of 02 Jul 2015


%\cite{Chatrchyan:2013fha}
\item%{Chatrchyan:2013fha}
{\bf ``Probing color coherence effects in pp collisions at $\sqrt{s}=7\,\text {TeV} $''}, 
  S.~Chatrchyan {\it et al.}  [CMS Collaboration], 
  %arXiv:1311.5815 [hep-ex]
    %{}10.1140/epjc/s10052-014-2901-8
Eur.\ Phys.\ J.\ C {\bf 74}no. 62901 (2014) %(Nov 222013)
%\href{http://inspirehep.net/record/1265659}{HEP entry}
%8 citations counted in INSPIRE as of 02 Jul 2015


%\cite{Chatrchyan:2013oba}
\item%{Chatrchyan:2013oba}
{\bf ``Search for pair production of excited top quarks in the lepton + jets final state''}, 
  S.~Chatrchyan {\it et al.}  [CMS Collaboration], 
  %arXiv:1311.5357 [hep-ex]
    %{}10.1007/JHEP06(2014)125
JHEP {\bf 1406}125 (2014) %(Nov 212013)
%\href{http://inspirehep.net/record/1265512}{HEP entry}
%3 citations counted in INSPIRE as of 02 Jul 2015


%\cite{Chatrchyan:2013iqa}
\item%{Chatrchyan:2013iqa}
{\bf ``Search for supersymmetry in pp collisions at $\sqrt{s}$=8 TeV in events with a single leptonlarge jet multiplicityand multiple b jets''}, 
  S.~Chatrchyan {\it et al.}  [CMS Collaboration], 
  %arXiv:1311.4937 [hep-ex]
    %{}10.1016/j.physletb.2014.04.023
Phys.\ Lett.\ B {\bf 733}328 (2014) %(Nov 192013)
%\href{http://inspirehep.net/record/1265220}{HEP entry}
%54 citations counted in INSPIRE as of 02 Jul 2015


%\cite{Chatrchyan:2013wua}
\item%{Chatrchyan:2013wua}
{\bf ``Measurements of $t\bar{t}$ spin correlations and top-quark polarization using dilepton final states in $pp$ collisions at $\sqrt{s}$ = 7 TeV''}, 
  S.~Chatrchyan {\it et al.}  [CMS Collaboration], 
  %arXiv:1311.3924 [hep-ex]
    %{}10.1103/PhysRevLett.112.182001
Phys.\ Rev.\ Lett.\  {\bf 112}no. 18182001 (2014) %(Nov 152013)
%\href{http://inspirehep.net/record/1264662}{HEP entry}
%35 citations counted in INSPIRE as of 02 Jul 2015


%\cite{Chatrchyan:2013gia}
\item%{Chatrchyan:2013gia}
{\bf ``Searches for light- and heavy-flavour three-jet resonances in pp collisions at $\sqrt{s} = 8$ TeV''}, 
  S.~Chatrchyan {\it et al.}  [CMS Collaboration], 
  %arXiv:1311.1799 [hep-ex]
    %{}10.1016/j.physletb.2014.01.049
Phys.\ Lett.\ B {\bf 730}193 (2014) %(Nov 72013)
%\href{http://inspirehep.net/record/1263658}{HEP entry}
%23 citations counted in INSPIRE as of 02 Jul 2015


%\cite{Chatrchyan:2013kba}
\item%{Chatrchyan:2013kba}
{\bf ``Measurement of higher-order harmonic azimuthal anisotropy in PbPb collisions at $\sqrt{s_{NN}}$ = 2.76 TeV''}, 
  S.~Chatrchyan {\it et al.}  [CMS Collaboration], 
  %arXiv:1310.8651 [nucl-ex]
    %{}10.1103/PhysRevC.89.044906
Phys.\ Rev.\ C {\bf 89}no. 4044906 (2014) %(Oct 312013)
%\href{http://inspirehep.net/record/1262804}{HEP entry}
%41 citations counted in INSPIRE as of 02 Jul 2015


%\cite{Chatrchyan:2013tia}
\item%{Chatrchyan:2013tia}
{\bf ``Measurement of the differential and double-differential Drell-Yan cross sections in proton-proton collisions at $\sqrt{s} =$ 7 TeV''}, 
  S.~Chatrchyan {\it et al.}  [CMS Collaboration], 
  %arXiv:1310.7291 [hep-ex]
    %{}10.1007/JHEP12(2013)030
JHEP {\bf 1312}030 (2013) %(Oct 272013)
%\href{http://inspirehep.net/record/1262319}{HEP entry}
%36 citations counted in INSPIRE as of 02 Jul 2015


%\cite{Chatrchyan:2013ala}
\item%{Chatrchyan:2013ala}
{\bf ``Jet and underlying event properties as a function of charged-particle multiplicity in proton–proton collisions at $\sqrt{s}$ = 7 TeV''}, 
  S.~Chatrchyan {\it et al.}  [CMS Collaboration], 
  %arXiv:1310.4554 [hep-ex]
    %{}10.1140/epjc/s10052-013-2674-5
Eur.\ Phys.\ J.\ C {\bf 73}no. 122674 (2013) %(Oct 162013)
%\href{http://inspirehep.net/record/1261026}{HEP entry}
%20 citations counted in INSPIRE as of 02 Jul 2015


%\cite{Chatrchyan:2013zna}
\item%{Chatrchyan:2013zna}
{\bf ``Search for the standard model Higgs boson produced in association with a W or a Z boson and decaying to bottom quarks''}, 
  S.~Chatrchyan {\it et al.}  [CMS Collaboration], 
  %arXiv:1310.3687 [hep-ex]
    %{}10.1103/PhysRevD.89.012003
Phys.\ Rev.\ D {\bf 89}no. 1012003 (2014) %(Oct 142013)
%\href{http://inspirehep.net/record/1258399}{HEP entry}
%153 citations counted in INSPIRE as of 02 Jul 2015


%\cite{Chatrchyan:2013oda}
\item%{Chatrchyan:2013oda}
{\bf ``Rapidity distributions in exclusive Z + jet and $\gamma$ + jet events in $pp$ collisions at $\sqrt{s}$ = 7 TeV''}, 
  S.~Chatrchyan {\it et al.}  [CMS Collaboration], 
  %arXiv:1310.3082 [hep-ex]
    %{}10.1103/PhysRevD.88.112009
Phys.\ Rev.\ D {\bf 88}no. 11112009 (2013) %(Oct 112013)
%\href{http://inspirehep.net/record/1258128}{HEP entry}
%16 citations counted in INSPIRE as of 02 Jul 2015


%\cite{Chatrchyan:2013bba}
\item%{Chatrchyan:2013bba}
{\bf ``Search for baryon number violation in top-quark decays''}, 
  S.~Chatrchyan {\it et al.}  [CMS Collaboration], 
  %arXiv:1310.1618 [hep-ex]
    %{}10.1016/j.physletb.2014.02.033
Phys.\ Lett.\ B {\bf 731}173 (2014) %(Oct 62013)
%\href{http://inspirehep.net/record/1257387}{HEP entry}
%2 citations counted in INSPIRE as of 02 Jul 2015


%\cite{Chatrchyan:2013zja}
\item%{Chatrchyan:2013zja}
{\bf ``Measurement of the cross section and angular correlations for associated production of a Z boson with b hadrons in pp collisions at $\sqrt{s} =$ 7 TeV''}, 
  S.~Chatrchyan {\it et al.}  [CMS Collaboration], 
  %arXiv:1310.1349 [hep-ex]
    %{}10.1007/JHEP12(2013)039
JHEP {\bf 1312}039 (2013) %(Oct 42013)
%\href{http://inspirehep.net/record/1256943}{HEP entry}
%18 citations counted in INSPIRE as of 02 Jul 2015


%\cite{Chatrchyan:2013uja}
\item%{Chatrchyan:2013uja}
{\bf ``Measurement of associated W + charm production in pp collisions at $\sqrt{s}$ = 7 TeV''}, 
  S.~Chatrchyan {\it et al.}  [CMS Collaboration], 
  %arXiv:1310.1138 [hep-ex]
    %{}10.1007/JHEP02(2014)013
JHEP {\bf 1402}013 (2014) %(Oct 32013)
%\href{http://inspirehep.net/record/1256938}{HEP entry}
%35 citations counted in INSPIRE as of 02 Jul 2015


%\cite{Chatrchyan:2013kwa}
\item%{Chatrchyan:2013kwa}
{\bf ``Modification of jet shapes in PbPb collisions at $\sqrt {s_{NN}} = 2.76$ TeV''}, 
  S.~Chatrchyan {\it et al.}  [CMS Collaboration], 
  %arXiv:1310.0878 [nucl-ex]
    %{}10.1016/j.physletb.2014.01.042
Phys.\ Lett.\ B {\bf 730}243 (2014) %(Oct 22013)
%\href{http://inspirehep.net/record/1256590}{HEP entry}
%45 citations counted in INSPIRE as of 02 Jul 2015


%\cite{Chatrchyan:2013dma}
\item%{Chatrchyan:2013dma}
{\bf ``Observation of a peaking structure in the $J/\psi \phi$ mass spectrum from $B^{\pm} \to J/\psi \phi K^{\pm}$ decays''}, 
  S.~Chatrchyan {\it et al.}  [CMS Collaboration], 
  %arXiv:1309.6920 [hep-ex]
    %{}10.1016/j.physletb.2014.05.055
Phys.\ Lett.\ B {\bf 734}261 (2014) %(Sep 262013)
%\href{http://inspirehep.net/record/1255647}{HEP entry}
%36 citations counted in INSPIRE as of 02 Jul 2015


%\cite{Chatrchyan:2013lca}
\item%{Chatrchyan:2013lca}
{\bf ``Searches for new physics using the $t\bar{t}$ invariant mass distribution in pp collisions at $\sqrt{s}$=8 TeV''}, 
  S.~Chatrchyan {\it et al.}  [CMS Collaboration], 
  %arXiv:1309.2030 [hep-ex]
    %{}10.1103/PhysRevLett.111.21180410.1103/PhysRevLett.112.119903
Phys.\ Rev.\ Lett.\  {\bf 111}no. 21211804 (2013)[Phys.\ Rev.\ Lett.\  {\bf 112}no. 11119903 (2014)] %(Sep 82013)
%\href{http://inspirehep.net/record/1253367}{HEP entry}
%57 citations counted in INSPIRE as of 02 Jul 2015


%\cite{Chatrchyan:2013nda}
\item%{Chatrchyan:2013nda}
{\bf ``Measurement of the production cross section for $Z\gamma \to \nu\bar{\nu}\gamma$ in pp collisions at $\sqrt{s} =$ 7 TeV and limits on $ZZ\gamma$ and $Z\gamma\gamma$ triple gauge boson couplings''}, 
  S.~Chatrchyan {\it et al.}  [CMS Collaboration], 
  %arXiv:1309.1117 [hep-ex]
    %{}10.1007/JHEP10(2013)164
JHEP {\bf 1310}164 (2013) %(Sep 42013)
%\href{http://inspirehep.net/record/1252719}{HEP entry}
%13 citations counted in INSPIRE as of 02 Jul 2015


%\cite{Chatrchyan:2013mea}
\item%{Chatrchyan:2013mea}
{\bf ``Search for a new bottomonium state decaying to $\Upsilon(1S)\pi^+\pi^-$ in pp collisions at $\sqrt{s}$ = 8 TeV''}, 
  S.~Chatrchyan {\it et al.}  [CMS Collaboration], 
  %arXiv:1309.0250 [hep-ex]
    %{}10.1016/j.physletb.2013.10.016
Phys.\ Lett.\ B {\bf 727}57 (2013) %(Sep 12013)
%\href{http://inspirehep.net/record/1252068}{HEP entry}
%24 citations counted in INSPIRE as of 02 Jul 2015


%\cite{Chatrchyan:2013fya}
\item%{Chatrchyan:2013fya}
{\bf ``Measurement of the $W\gamma$ and $Z\gamma$ inclusive cross sections in $pp$ collisions at $\sqrt s=7$ TeV and limits on anomalous triple gauge boson couplings''}, 
  S.~Chatrchyan {\it et al.}  [CMS Collaboration], 
  %arXiv:1308.6832 [hep-ex]
    %{}10.1103/PhysRevD.89.092005
Phys.\ Rev.\ D {\bf 89}no. 9092005 (2014) %(Aug 302013)
%\href{http://inspirehep.net/record/1251905}{HEP entry}
%31 citations counted in INSPIRE as of 02 Jul 2015


%\cite{Chatrchyan:2013jna}
\item%{Chatrchyan:2013jna}
{\bf ``Measurement of the W-boson helicity in top-quark decays from $t\bar{t}$ production in lepton+jets events in pp collisions at $\sqrt{s} =$ 7 TeV''}, 
  S.~Chatrchyan {\it et al.}  [CMS Collaboration], 
  %arXiv:1308.3879 [hep-ex]
    %{}10.1007/JHEP10(2013)167
JHEP {\bf 1310}167 (2013) %(Aug 182013)
%\href{http://inspirehep.net/record/1249595}{HEP entry}
%24 citations counted in INSPIRE as of 02 Jul 2015


%\cite{Chatrchyan:2013cda}
\item%{Chatrchyan:2013cda}
{\bf ``Angular analysis and branching fraction measurement of the decay $B^0 \to K^{*0} \mu^+\mu^-$''}, 
  S.~Chatrchyan {\it et al.}  [CMS Collaboration], 
  %arXiv:1308.3409 [hep-ex]
    %{}10.1016/j.physletb.2013.10.017
Phys.\ Lett.\ B {\bf 727}77 (2013) %(Aug 152013)
%\href{http://inspirehep.net/record/1247976}{HEP entry}
%34 citations counted in INSPIRE as of 02 Jul 2015


%\cite{Chatrchyan:2013xna}
\item%{Chatrchyan:2013xna}
{\bf ``Search for top-squark pair production in the single-lepton final state in pp collisions at $\sqrt{s}$ = 8 TeV''}, 
  S.~Chatrchyan {\it et al.}  [CMS Collaboration], 
  %arXiv:1308.1586 [hep-ex]
    %{}10.1140/epjc/s10052-013-2677-2
Eur.\ Phys.\ J.\ C {\bf 73}no. 122677 (2013) %(Aug 72013)
%\href{http://inspirehep.net/record/1246905}{HEP entry}
%166 citations counted in INSPIRE as of 02 Jul 2015


%\cite{Chatrchyan:2013cla}
\item%{Chatrchyan:2013cla}
{\bf ``Measurement of the prompt $J/\psi$ and $\psi$(2S) polarizations in $pp$ collisions at $\sqrt{s}$ = 7 TeV''}, 
  S.~Chatrchyan {\it et al.}  [CMS Collaboration], 
  %arXiv:1307.6070 [hep-ex]
    %{}10.1016/j.physletb.2013.10.055
Phys.\ Lett.\ B {\bf 727}381 (2013) %(Jul 232013)
%\href{http://inspirehep.net/record/1244128}{HEP entry}
%56 citations counted in INSPIRE as of 02 Jul 2015


%\cite{Chatrchyan:2013vaa}
\item%{Chatrchyan:2013vaa}
{\bf ``Search for a Higgs boson decaying into a Z and a photon in pp collisions at $\sqrt{s}$ = 7 and 8 TeV''}, 
  S.~Chatrchyan {\it et al.}  [CMS Collaboration], 
  %arXiv:1307.5515 [hep-ex]
    %{}10.1016/j.physletb.2013.09.057
Phys.\ Lett.\ B {\bf 726}587 (2013) %(Jul 212013)
%\href{http://inspirehep.net/record/1243861}{HEP entry}
%83 citations counted in INSPIRE as of 02 Jul 2015


%\cite{Chatrchyan:2013bka}
\item%{Chatrchyan:2013bka}
{\bf ``Measurement of the B(s) to mu+ mu- branching fraction and search for B0 to mu+ mu- with the CMS Experiment''}, 
  S.~Chatrchyan {\it et al.}  [CMS Collaboration], 
  %arXiv:1307.5025 [hep-ex]
    %{}10.1103/PhysRevLett.111.101804
Phys.\ Rev.\ Lett.\  {\bf 111}101804 (2013) %(Jul 182013)
%\href{http://inspirehep.net/record/1243425}{HEP entry}
%205 citations counted in INSPIRE as of 02 Jul 2015


%\cite{Chatrchyan:2013xza}
\item%{Chatrchyan:2013xza}
{\bf ``Measurement of the top-quark mass in all-jets $t\bar{t}$ events in pp collisions at $\sqrt{s}$=7 TeV''}, 
  S.~Chatrchyan {\it et al.}  [CMS Collaboration], 
  %arXiv:1307.4617 [hep-ex]
    %{}10.1140/epjc/s10052-014-2758-x
Eur.\ Phys.\ J.\ C {\bf 74}no. 42758 (2014) %(Jul 172013)
%\href{http://inspirehep.net/record/1243161}{HEP entry}
%47 citations counted in INSPIRE as of 02 Jul 2015


%\cite{Chatrchyan:2013eya}
\item%{Chatrchyan:2013eya}
{\bf ``Study of the production of charged pions, kaons and protons in pPb collisions at $\sqrt{s_{NN}} =\  $ 5.02 $\,\text {TeV}$''}, 
  S.~Chatrchyan {\it et al.}  [CMS Collaboration], 
  %arXiv:1307.3442 [hep-ex]
    %{}10.1140/epjc/s10052-014-2847-x
Eur.\ Phys.\ J.\ C {\bf 74}no. 62847 (2014) %(Jul 122013)
%\href{http://inspirehep.net/record/1242440}{HEP entry}
%57 citations counted in INSPIRE as of 02 Jul 2015


%\cite{Chatrchyan:2013haa}
\item%{Chatrchyan:2013haa}
{\bf ``Determination of the top-quark pole mass and strong coupling constant from the t t-bar production cross section in pp collisions at $\sqrt{s}$ = 7 TeV''}, 
  S.~Chatrchyan {\it et al.}  [CMS Collaboration], 
  %arXiv:1307.1907 [hep-ex]
    %{}10.1016/j.physletb.2014.08.04010.1016/j.physletb.2013.12.009
Phys.\ Lett.\ B {\bf 728}496 (2014)[Phys.\ Lett.\ B {\bf 728}526 (2014)] %(Aug 212014)
%\href{http://inspirehep.net/record/1241819}{HEP entry}
%70 citations counted in INSPIRE as of 02 Jul 2015


%\cite{Chatrchyan:2013sba}
\item%{Chatrchyan:2013sba}
{\bf ``The performance of the CMS muon detector in proton-proton collisions at sqrt(s) = 7 TeV at the LHC''}, 
  S.~Chatrchyan {\it et al.}  [CMS Collaboration], 
  %arXiv:1306.6905 [physics.ins-det]
    %{}10.1088/1748-0221/8/11/P11002
JINST {\bf 8}P11002 (2013) %(Jun 282013)
%\href{http://inspirehep.net/record/1240504}{HEP entry}
%9 citations counted in INSPIRE as of 02 Jul 2015


%\cite{Chatrchyan:2013xsw}
\item%{Chatrchyan:2013xsw}
{\bf ``Search for top squarks in $R$-parity-violating supersymmetry using three or more leptons and b-tagged jets''}, 
  S.~Chatrchyan {\it et al.}  [CMS Collaboration], 
  %arXiv:1306.6643 [hep-ex]
    %{}10.1103/PhysRevLett.111.221801
Phys.\ Rev.\ Lett.\  {\bf 111}no. 22221801 (2013) %(Jun 272013)
%\href{http://inspirehep.net/record/1240497}{HEP entry}
%36 citations counted in INSPIRE as of 02 Jul 2015


%\cite{Chatrchyan:2013dga}
\item%{Chatrchyan:2013dga}
{\bf ``Energy Calibration and Resolution of the CMS Electromagnetic Calorimeter in $pp$ Collisions at $\sqrt{s} = 7$ TeV''}, 
  S.~Chatrchyan {\it et al.}  [CMS Collaboration], 
  %arXiv:1306.2016 [hep-ex]
    %{}10.1088/1748-0221/8/09/P09009
JINST {\bf 8}P09009 (2013) %(Jun 92013)
%\href{http://inspirehep.net/record/1237915}{HEP entry}
%97 citations counted in INSPIRE as of 02 Jul 2015


%\cite{Chatrchyan:2013yaa}
\item%{Chatrchyan:2013yaa}
{\bf ``Measurement of the $W^+W^-$ Cross section in $pp$ Collisions at $\sqrt{s} = 7$ TeV and Limits on Anomalous $WW\gamma$ and $WWZ$ couplings''}, 
  S.~Chatrchyan {\it et al.}  [CMS Collaboration], 
  %arXiv:1306.1126 [hep-ex]
    %{}10.1140/epjc/s10052-013-2610-8
Eur.\ Phys.\ J.\ C {\bf 73}no. 102610 (2013) %(Jun 52013)
%\href{http://inspirehep.net/record/1237104}{HEP entry}
%63 citations counted in INSPIRE as of 02 Jul 2015


%\cite{Chatrchyan:2013jya}
\item%{Chatrchyan:2013jya}
{\bf ``Measurement of the hadronic activity in events with a Z and two jets and extraction of the cross section for the electroweak production of a Z with two jets in pp collisions at $\sqrt{s}$ = 7 TeV''}, 
  S.~Chatrchyan {\it et al.}  [CMS Collaboration], 
  %arXiv:1305.7389 [hep-ex]
    %{}10.1007/JHEP10(2013)062
JHEP {\bf 1310}062 (2013) %(May 312013)
%\href{http://inspirehep.net/record/1236361}{HEP entry}
%19 citations counted in INSPIRE as of 02 Jul 2015


%\cite{Chatrchyan:2013qsa}
\item%{Chatrchyan:2013qsa}
{\bf ``Measurement of neutral strange particle production in the underlying event in proton-proton collisions at sqrt(s) = 7 TeV''}, 
  S.~Chatrchyan {\it et al.}  [CMS Collaboration], 
  %arXiv:1305.6016 [hep-ex]
    %{}10.1103/PhysRevD.88.052001
Phys.\ Rev.\ D {\bf 88}052001 (2013) %(May 262013)
%\href{http://inspirehep.net/record/1235536}{HEP entry}
%6 citations counted in INSPIRE as of 02 Jul 2015


%\cite{Chatrchyan:2013foa}
\item%{Chatrchyan:2013foa}
{\bf ``Study of exclusive two-photon production of $W^+W^-$ in $pp$ collisions at $\sqrt{s} = 7$ TeV and constraints on anomalous quartic gauge couplings''}, 
  S.~Chatrchyan {\it et al.}  [CMS Collaboration], 
  %arXiv:1305.5596 [hep-ex]
    %{}10.1007/JHEP07(2013)116
JHEP {\bf 1307}116 (2013) %(May 232013)
%\href{http://inspirehep.net/record/1235421}{HEP entry}
%70 citations counted in INSPIRE as of 02 Jul 2015


%\cite{Chatrchyan:2013wxa}
\item%{Chatrchyan:2013wxa}
{\bf ``Search for gluino mediated bottom- and top-squark production in multijet final states in pp collisions at 8 TeV''}, 
  S.~Chatrchyan {\it et al.}  [CMS Collaboration], 
  %arXiv:1305.2390 [hep-ex]
    %{}10.1016/j.physletb.2013.06.058
Phys.\ Lett.\ B {\bf 725}243 (2013) %(May 102013)
%\href{http://inspirehep.net/record/1232968}{HEP entry}
%83 citations counted in INSPIRE as of 02 Jul 2015


%\cite{Chatrchyan:2013nka}
\item%{Chatrchyan:2013nka}
{\bf ``Multiplicity and transverse momentum dependence of two- and four-particle correlations in pPb and PbPb collisions''}, 
  S.~Chatrchyan {\it et al.}  [CMS Collaboration], 
  %arXiv:1305.0609 [nucl-ex]
    %{}10.1016/j.physletb.2013.06.028
Phys.\ Lett.\ B {\bf 724}213 (2013) %(May 22013)
%\href{http://inspirehep.net/record/1231945}{HEP entry}
%157 citations counted in INSPIRE as of 02 Jul 2015


%\cite{Chatrchyan:2013oca}
\item%{Chatrchyan:2013oca}
{\bf ``Searches for long-lived charged particles in pp collisions at $\sqrt{s}$=7 and 8 TeV''}, 
  S.~Chatrchyan {\it et al.}  [CMS Collaboration], 
  %arXiv:1305.0491 [hep-ex]
    %{}10.1007/JHEP07(2013)122
JHEP {\bf 1307}122 (2013) %(May 22013)
%\href{http://inspirehep.net/record/1231738}{HEP entry}
%94 citations counted in INSPIRE as of 02 Jul 2015


%\cite{Chatrchyan:2013txa}
\item%{Chatrchyan:2013txa}
{\bf ``Measurement of the ratio of the inclusive 3-jet cross section to the inclusive 2-jet cross section in pp collisions at $\sqrt{s}$ = 7 TeV and first determination of the strong coupling constant in the TeV range''}, 
  S.~Chatrchyan {\it et al.}  [CMS Collaboration], 
  %arXiv:1304.7498 [hep-ex]
    %{}10.1140/epjc/s10052-013-2604-6
Eur.\ Phys.\ J.\ C {\bf 73}no. 102604 (2013) %(Apr 282013)
%\href{http://inspirehep.net/record/1230937}{HEP entry}
%45 citations counted in INSPIRE as of 02 Jul 2015


%\cite{Chatrchyan:2013sxa}
\item%{Chatrchyan:2013sxa}
{\bf ``Measurement of the $\Lambda_{b}^0$ lifetime in pp collisions at $\sqrt{s} = 7$ TeV''}, 
  S.~Chatrchyan {\it et al.}  [CMS Collaboration], 
  %arXiv:1304.7495 [hep-ex]
    %{}10.1007/JHEP07(2013)163
JHEP {\bf 1307}163 (2013) %(Apr 282013)
%\href{http://inspirehep.net/record/1230936}{HEP entry}
%25 citations counted in INSPIRE as of 02 Jul 2015


%\cite{Chatrchyan:2013boa}
\item%{Chatrchyan:2013boa}
{\bf ``Measurement of masses in the $t \bar{t}$ system by kinematic endpoints in pp collisions at $\sqrt{s}$ = 7 TeV''}, 
  S.~Chatrchyan {\it et al.}  [CMS Collaboration], 
  %arXiv:1304.5783 [hep-ex]
    %{}10.1140/epjc/s10052-013-2494-7
Eur.\ Phys.\ J.\ C {\bf 73}2494 (2013) %(Apr 212013)
%\href{http://inspirehep.net/record/1229333}{HEP entry}
%49 citations counted in INSPIRE as of 02 Jul 2015


%\cite{Chatrchyan:2013yoa}
\item%{Chatrchyan:2013yoa}
{\bf ``Search for a standard-model-like Higgs boson with a mass in the range 145 to 1000 GeV at the LHC''}, 
  S.~Chatrchyan {\it et al.}  [CMS Collaboration], 
  %arXiv:1304.0213 [hep-ex]
    %{}10.1140/epjc/s10052-013-2469-8
Eur.\ Phys.\ J.\ C {\bf 73}2469 (2013) %(Mar 312013)
%\href{http://inspirehep.net/record/1225976}{HEP entry}
%75 citations counted in INSPIRE as of 02 Jul 2015


%\cite{Chatrchyan:2013yna}
\item%{Chatrchyan:2013yna}
{\bf ``Measurement of the $\Upsilon(1S)\Upsilon(2S)$and $\Upsilon(3S)$ cross sections in pp collisions at $\sqrt{s}$ = 7 TeV''}, 
  S.~Chatrchyan {\it et al.}  [CMS Collaboration], 
  %arXiv:1303.5900 [hep-ex]
    %{}10.1016/j.physletb.2013.10.033
Phys.\ Lett.\ B {\bf 727}101 (2013) %(Mar 232013)
%\href{http://inspirehep.net/record/1225274}{HEP entry}
%34 citations counted in INSPIRE as of 02 Jul 2015


%\cite{Chatrchyan:2013xva}
\item%{Chatrchyan:2013xva}
{\bf ``Search for microscopic black holes in pp collisions at sqrt(s) = 8 TeV''}, 
  S.~Chatrchyan {\it et al.}  [CMS Collaboration], 
  %arXiv:1303.5338 [hep-ex]
    %{}10.1007/JHEP07(2013)178
JHEP {\bf 1307}178 (2013) %(Mar 212013)
%\href{http://inspirehep.net/record/1224805}{HEP entry}
%43 citations counted in INSPIRE as of 02 Jul 2015


%\cite{Chatrchyan:2013rla}
\item%{Chatrchyan:2013rla}
{\bf ``Studies of jet mass in dijet and W/Z + jet events''}, 
  S.~Chatrchyan {\it et al.}  [CMS Collaboration], 
  %arXiv:1303.4811 [hep-ex]
    %{}10.1007/JHEP05(2013)090
JHEP {\bf 1305}090 (2013) %(Mar 192013)
%\href{http://inspirehep.net/record/1224539}{HEP entry}
%43 citations counted in INSPIRE as of 02 Jul 2015


%\cite{Chatrchyan:2013lba}
\item%{Chatrchyan:2013lba}
{\bf ``Observation of a new boson with mass near 125 GeV in pp collisions at $\sqrt{s}$ = 7 and 8 TeV''}, 
  S.~Chatrchyan {\it et al.}  [CMS Collaboration], 
  %arXiv:1303.4571 [hep-ex]
    %{}10.1007/JHEP06(2013)081
JHEP {\bf 1306}081 (2013) %(Mar 192013)
%\href{http://inspirehep.net/record/1224273}{HEP entry}
%319 citations counted in INSPIRE as of 02 Jul 2015


%\cite{CMS:2012nga}
\item%{CMS:2012nga}
{\bf ``A New Boson with a Mass of 125 GeV Observed with the CMS Experiment at the Large Hadron Collider''}, 
  S.~Chatrchyan {\it et al.}  [CMS Collaboration], 
    %{}10.1126/science.1230816
Science {\bf 338}1569 (2012). %(2012)
%\href{http://inspirehep.net/record/1223729}{HEP entry}
%42 citations counted in INSPIRE as of 02 Jul 2015


%\cite{Chatrchyan:2013qca}
\item%{Chatrchyan:2013qca}
{\bf ``Measurement of associated production of vector bosons and top quark-antiquark pairs at sqrt(s) = 7 TeV''}, 
  S.~Chatrchyan {\it et al.}  [CMS Collaboration], 
  %arXiv:1303.3239 [hep-ex]
    %{}10.1103/PhysRevLett.110.172002
Phys.\ Rev.\ Lett.\  {\bf 110}172002 (2013) %(Mar 132013)
%\href{http://inspirehep.net/record/1223628}{HEP entry}
%48 citations counted in INSPIRE as of 02 Jul 2015


%\cite{Chatrchyan:2013lya}
\item%{Chatrchyan:2013lya}
{\bf ``Search for supersymmetry in hadronic final states with missing transverse energy using the variables $\alpha_T$ and b-quark multiplicity in pp collisions at $\sqrt s=8$ TeV''}, 
  S.~Chatrchyan {\it et al.}  [CMS Collaboration], 
  %arXiv:1303.2985 [hep-ex]
    %{}10.1140/epjc/s10052-013-2568-6
Eur.\ Phys.\ J.\ C {\bf 73}no. 92568 (2013) %(Mar 122013)
%\href{http://inspirehep.net/record/1223519}{HEP entry}
%161 citations counted in INSPIRE as of 02 Jul 2015


%\cite{Chatrchyan:2013yea}
\item%{Chatrchyan:2013yea}
{\bf ``Search for the standard model Higgs boson produced in association with a top-quark pair in pp collisions at the LHC''}, 
  S.~Chatrchyan {\it et al.}  [CMS Collaboration], 
  %arXiv:1303.0763 [hep-ex]
    %{}10.1007/JHEP05(2013)145
JHEP {\bf 1305}145 (2013) %(Mar 42013)
%\href{http://inspirehep.net/record/1222336}{HEP entry}
%61 citations counted in INSPIRE as of 02 Jul 2015


%\cite{Chatrchyan:2013qha}
\item%{Chatrchyan:2013qha}
{\bf ``Search for narrow resonances using the dijet mass spectrum in pp collisions at $\sqrt{s}$=8 TeV''}, 
  S.~Chatrchyan {\it et al.}  [CMS Collaboration], 
  %arXiv:1302.4794 [hep-ex]
    %{}10.1103/PhysRevD.87.114015
Phys.\ Rev.\ D {\bf 87}no. 11114015 (2013) %(Feb 192013)
%\href{http://inspirehep.net/record/1220378}{HEP entry}
%94 citations counted in INSPIRE as of 02 Jul 2015


%\cite{Chatrchyan:2013cld}
\item%{Chatrchyan:2013cld}
{\bf ``Measurement of the X(3872) production cross section via decays to J/psi pi pi in pp collisions at sqrt(s) = 7 TeV''}, 
  S.~Chatrchyan {\it et al.}  [CMS Collaboration], 
  %arXiv:1302.3968 [hep-ex]
    %{}10.1007/JHEP04(2013)154
JHEP {\bf 1304}154 (2013) %(Feb 162013)
%\href{http://inspirehep.net/record/1219950}{HEP entry}
%55 citations counted in INSPIRE as of 02 Jul 2015


%\cite{Chatrchyan:2013qga}
\item%{Chatrchyan:2013qga}
{\bf ``Search for a Higgs boson decaying into a b-quark pair and produced in association with b quarks in proton-proton collisions at 7 TeV''}, 
  S.~Chatrchyan {\it et al.}  [CMS Collaboration], 
  %arXiv:1302.2892 [hep-ex]
    %{}10.1016/j.physletb.2013.04.017
Phys.\ Lett.\ B {\bf 722}207 (2013) %(Feb 122013)
%\href{http://inspirehep.net/record/1219000}{HEP entry}
%50 citations counted in INSPIRE as of 02 Jul 2015


%\cite{Chatrchyan:2013lga}
\item%{Chatrchyan:2013lga}
{\bf ``Search for new physics in final states with a lepton and missing transverse energy in pp collisions at the LHC''}, 
  S.~Chatrchyan {\it et al.}  [CMS Collaboration], 
  %arXiv:1302.2812 [hep-ex]
    %{}10.1103/PhysRevD.87.072005
Phys.\ Rev.\ D {\bf 87}no. 7072005 (2013) %(Feb 122013)
%\href{http://inspirehep.net/record/1218995}{HEP entry}
%34 citations counted in INSPIRE as of 02 Jul 2015


%\cite{Chatrchyan:2013gfi}
\item%{Chatrchyan:2013gfi}
{\bf ``Study of the underlying event at forward rapidity in pp collisions at $\sqrt{s}$ = 0.9, 2.76 and 7 TeV''}, 
  S.~Chatrchyan {\it et al.}  [CMS Collaboration], 
  %arXiv:1302.2394 [hep-ex]
    %{}10.1007/JHEP04(2013)072
JHEP {\bf 1304}072 (2013) %(Feb 102013)
%\href{http://inspirehep.net/record/1218372}{HEP entry}
%31 citations counted in INSPIRE as of 02 Jul 2015


%\cite{Chatrchyan:2013sfs}
\item%{Chatrchyan:2013sfs}
{\bf ``Searches for Higgs bosons in pp collisions at sqrt(s) = 7 and 8 TeV in the context of four-generation and fermiophobic models''}, 
  S.~Chatrchyan {\it et al.}  [CMS Collaboration], 
  %arXiv:1302.1764 [hep-ex]
    %{}10.1016/j.physletb.2013.06.043
Phys.\ Lett.\ B {\bf 725}36 (2013) %(Feb 72013)
%\href{http://inspirehep.net/record/1218021}{HEP entry}
%21 citations counted in INSPIRE as of 02 Jul 2015


%\cite{Chatrchyan:2013izb}
\item%{Chatrchyan:2013izb}
{\bf ``Search for pair-produced dijet resonances in four-jet final states in pp collisions at $\sqrt{s}$=7 TeV''}, 
  S.~Chatrchyan {\it et al.}  [CMS Collaboration], 
  %arXiv:1302.0531 [hep-ex]
    %{}10.1103/PhysRevLett.110.141802
Phys.\ Rev.\ Lett.\  {\bf 110}no. 14141802 (2013) %(Feb 32013)
%\href{http://inspirehep.net/record/1217552}{HEP entry}
%41 citations counted in INSPIRE as of 02 Jul 2015


%\cite{Chatrchyan:2013ual}
\item%{Chatrchyan:2013ual}
{\bf ``Measurement of the $t\bar{t}$ production cross section in the all-jet final state in pp collisions at $\sqrt{s}$ = 7 TeV''}, 
  S.~Chatrchyan {\it et al.}  [CMS Collaboration], 
  %arXiv:1302.0508 [hep-ex]
    %{}10.1007/JHEP05(2013)065
JHEP {\bf 1305}065 (2013) %(Feb 32013)
%\href{http://inspirehep.net/record/1217551}{HEP entry}
%29 citations counted in INSPIRE as of 02 Jul 2015


%\cite{Chatrchyan:2013kff}
\item%{Chatrchyan:2013kff}
{\bf ``Measurement of the top-antitop production cross section in the tau+jets channel in pp collisions at sqrt(s) = 7 TeV''}, 
  S.~Chatrchyan {\it et al.}  [CMS Collaboration], 
  %arXiv:1301.5755 [hep-ex]
    %{}10.1140/epjc/s10052-013-2386-x
Eur.\ Phys.\ J.\ C {\bf 73}no. 42386 (2013) %(Jan 242013)
%\href{http://inspirehep.net/record/1216035}{HEP entry}
%27 citations counted in INSPIRE as of 02 Jul 2015


%\cite{Chatrchyan:2013muj}
\item%{Chatrchyan:2013muj}
{\bf ``Search for contact interactions using the inclusive jet $p_T$ spectrum in $pp$ collisions at $\sqrt{s} = 7$ TeV''}, 
  S.~Chatrchyan {\it et al.}  [CMS Collaboration], 
  %arXiv:1301.5023 [hep-ex]
    %{}10.1103/PhysRevD.87.052017
Phys.\ Rev.\ D {\bf 87}no. 5052017 (2013) %(Jan 212013)
%\href{http://inspirehep.net/record/1215599}{HEP entry}
%24 citations counted in INSPIRE as of 02 Jul 2015


%\cite{Chatrchyan:2013oev}
\item%{Chatrchyan:2013oev}
{\bf ``Measurement of W+W- and ZZ production cross sections in pp collisions at sqrt(s) = 8 TeV''}, 
  S.~Chatrchyan {\it et al.}  [CMS Collaboration], 
  %arXiv:1301.4698 [hep-ex]
    %{}10.1016/j.physletb.2013.03.027
Phys.\ Lett.\ B {\bf 721}190 (2013) %(Jan 202013)
%\href{http://inspirehep.net/record/1215317}{HEP entry}
%107 citations counted in INSPIRE as of 02 Jul 2015


%\cite{Chatrchyan:2013dsa}
\item%{Chatrchyan:2013dsa}
{\bf ``Search for physics beyond the standard model in events with $\tau$ leptonsjetsand large transverse momentum imbalance in pp collisions at $\sqrt{s}$ = 7 TeV''}, 
  S.~Chatrchyan {\it et al.}  [CMS Collaboration], 
  %arXiv:1301.3792 [hep-ex]
    %{}10.1140/epjc/s10052-013-2493-8
Eur.\ Phys.\ J.\ C {\bf 73}2493 (2013) %(Jan 2013)
%\href{http://inspirehep.net/record/1211187}{HEP entry}
%13 citations counted in INSPIRE as of 02 Jul 2015


%\cite{Chatrchyan:2013sza}
\item%{Chatrchyan:2013sza}
{\bf ``Interpretation of Searches for Supersymmetry with simplified Models''}, 
  S.~Chatrchyan {\it et al.}  [CMS Collaboration], 
  %arXiv:1301.2175 [hep-ex]
    %{}10.1103/PhysRevD.88.052017
Phys.\ Rev.\ D {\bf 88}no. 5052017 (2013) %(Jan 2013)
%\href{http://inspirehep.net/record/1210032}{HEP entry}
%72 citations counted in INSPIRE as of 02 Jul 2015


%\cite{Chatrchyan:2013tna}
\item%{Chatrchyan:2013tna}
{\bf ``Event shapes and azimuthal correlations in $Z$ + jets events in $pp$ collisions at $\sqrt{s}=7$ TeV''}, 
  S.~Chatrchyan {\it et al.}  [CMS Collaboration], 
  %arXiv:1301.1646 [hep-ex]
    %{}10.1016/j.physletb.2013.04.025
Phys.\ Lett.\ B {\bf 722}238 (2013) %(Jan 2013)
%\href{http://inspirehep.net/record/1209721}{HEP entry}
%32 citations counted in INSPIRE as of 02 Jul 2015


%\cite{Chatrchyan:2013kha}
\item%{Chatrchyan:2013kha}
{\bf ``Search for supersymmetry in events with opposite-sign dileptons and missing transverse energy using an artificial neural network''}, 
  S.~Chatrchyan {\it et al.}  [CMS Collaboration], 
  %arXiv:1301.0916 [hep-ex]
    %{}10.1103/PhysRevD.87.072001
Phys.\ Rev.\ D {\bf 87}no. 7072001 (2013) %(Jan 2013)
%\href{http://inspirehep.net/record/1209556}{HEP entry}
%11 citations counted in INSPIRE as of 02 Jul 2015


%\cite{Chatrchyan:2012ola}
\item%{Chatrchyan:2012ola}
{\bf ``Search for supersymmetry in $pp$ collisions at $\sqrt{s}=7$ TeV in events with a single leptonjetsand missing transverse momentum''}, 
  S.~Chatrchyan {\it et al.}  [CMS Collaboration], 
  %arXiv:1212.6428 [hep-ex]
    %{}10.1140/epjc/s10052-013-2404-z
Eur.\ Phys.\ J.\ C {\bf 73}2404 (2013) %(Dec 2012)
%\href{http://inspirehep.net/record/1208988}{HEP entry}
%21 citations counted in INSPIRE as of 02 Jul 2015


%\cite{Chatrchyan:2012jja}
\item%{Chatrchyan:2012jja}
{\bf ``Study of the Mass and Spin-Parity of the Higgs Boson Candidate Via Its Decays to Z Boson Pairs''}, 
  S.~Chatrchyan {\it et al.}  [CMS Collaboration], 
  %arXiv:1212.6639 [hep-ex]
    %{}10.1103/PhysRevLett.110.081803
Phys.\ Rev.\ Lett.\  {\bf 110}no. 8081803 (2013) %(Dec 2012)
%\href{http://inspirehep.net/record/1208931}{HEP entry}
%237 citations counted in INSPIRE as of 02 Jul 2015


%\cite{Chatrchyan:2012bja}
\item%{Chatrchyan:2012bja}
{\bf ``Measurements of differential jet cross sections in proton-proton collisions at $\sqrt{s}=7$ TeV with the CMS detector''}, 
  S.~Chatrchyan {\it et al.}  [CMS Collaboration], 
  %arXiv:1212.6660 [hep-ex]
    %{}10.1103/PhysRevD.87.11200210.1103/PhysRevD.87.119902
Phys.\ Rev.\ D {\bf 87}no. 11112002 (2013)[Phys.\ Rev.\ D {\bf 87}no. 11119902 (2013)] %(Dec 2012)
%\href{http://inspirehep.net/record/1208923}{HEP entry}
%90 citations counted in INSPIRE as of 02 Jul 2015


%\cite{Chatrchyan:2012ria}
\item%{Chatrchyan:2012ria}
{\bf ``Measurement of the $t\bar{t}$ production cross section in $pp$ collisions at $\sqrt{s}=7$ TeV with lepton + jets final states''}, 
  S.~Chatrchyan {\it et al.}  [CMS Collaboration], 
  %arXiv:1212.6682
    %{}10.1016/j.physletb.2013.02.021
Phys.\ Lett.\ B {\bf 720}83 (2013) %(Dec 2012)
%\href{http://inspirehep.net/record/1208913}{HEP entry}
%51 citations counted in INSPIRE as of 02 Jul 2015


%\cite{Chatrchyan:2012uea}
\item%{Chatrchyan:2012uea}
{\bf ``Inclusive search for supersymmetry using the razor variables in $pp$ collisions at $\sqrt{s}=7$ TeV''}, 
  S.~Chatrchyan {\it et al.}  [CMS Collaboration], 
  %arXiv:1212.6961 [hep-ex]
    %{}10.1103/PhysRevLett.111.081802
Phys.\ Rev.\ Lett.\  {\bf 111}no. 8081802 (2013) %(Dec 2012)
%\href{http://inspirehep.net/record/1208812}{HEP entry}
%57 citations counted in INSPIRE as of 02 Jul 2015


%\cite{Chatrchyan:2012paa}
\item%{Chatrchyan:2012paa}
{\bf ``Search for new physics in events with same-sign dileptons and $b$ jets in $pp$ collisions at $\sqrt{s}=8$ TeV''}, 
  S.~Chatrchyan {\it et al.}  [CMS Collaboration], 
  %arXiv:1212.6194 [hep-ex]
    %{}10.1007/JHEP03(2013)03710.1007/JHEP07(2013)041
JHEP {\bf 1303}037 (2013)[JHEP {\bf 1307}041 (2013)] %(Dec 2012)
%\href{http://inspirehep.net/record/1208703}{HEP entry}
%72 citations counted in INSPIRE as of 02 Jul 2015


%\cite{Chatrchyan:2012oaa}
\item%{Chatrchyan:2012oaa}
{\bf ``Search for heavy narrow dilepton resonances in $pp$ collisions at $\sqrt{s}=7$ TeV and $\sqrt{s}=8$ TeV''}, 
  S.~Chatrchyan {\it et al.}  [CMS Collaboration], 
  %arXiv:1212.6175 [hep-ex]
    %{}10.1016/j.physletb.2013.02.003
Phys.\ Lett.\ B {\bf 720}63 (2013) %(Dec 2012)
%\href{http://inspirehep.net/record/1208702}{HEP entry}
%86 citations counted in INSPIRE as of 02 Jul 2015


%\cite{Chatrchyan:2012hda}
\item%{Chatrchyan:2012hda}
{\bf ``Search for contact interactions in $\mu^+\mu^-$ events in $pp$ collisions at $\sqrt{s}=7$ TeV''}, 
  S.~Chatrchyan {\it et al.}  [CMS Collaboration], 
  %arXiv:1212.4563 [hep-ex]
    %{}10.1103/PhysRevD.87.032001
Phys.\ Rev.\ D {\bf 87}no. 3032001 (2013) %(Dec 2012)
%\href{http://inspirehep.net/record/1208097}{HEP entry}
%15 citations counted in INSPIRE as of 02 Jul 2015


%\cite{Chatrchyan:2012ypy}
\item%{Chatrchyan:2012ypy}
{\bf ``Search for heavy resonances in the W/Z-tagged dijet mass spectrum in pp collisions at 7 TeV''}, 
  S.~Chatrchyan {\it et al.}  [CMS Collaboration], 
  %arXiv:1212.1910 [hep-ex]
    %{}10.1016/j.physletb.2013.05.040
Phys.\ Lett.\ B {\bf 723}280 (2013) %(Dec 2012)
%\href{http://inspirehep.net/record/1206606}{HEP entry}
%36 citations counted in INSPIRE as of 02 Jul 2015


%\cite{Chatrchyan:2012jwg}
\item%{Chatrchyan:2012jwg}
{\bf ``Search for long-lived particles decaying to photons and missing energy in proton-proton collisions at $\sqrt{s}=7$ TeV''}, 
  S.~Chatrchyan {\it et al.}  [CMS Collaboration], 
  %arXiv:1212.1838 [hep-ex]
    %{}10.1016/j.physletb.2013.04.027
Phys.\ Lett.\ B {\bf 722}273 (2013) %(Dec 2012)
%\href{http://inspirehep.net/record/1206603}{HEP entry}
%25 citations counted in INSPIRE as of 02 Jul 2015


%\cite{Chatrchyan:2012rva}
\item%{Chatrchyan:2012rva}
{\bf ``Search for exotic resonances decaying into $WZ/ZZ$ in $pp$ collisions at $\sqrt{s}=7$ TeV''}, 
  S.~Chatrchyan {\it et al.}  [CMS Collaboration], 
  %arXiv:1211.5779 [hep-ex]
    %{}10.1007/JHEP02(2013)036
JHEP {\bf 1302}036 (2013) %(Nov 2012)
%\href{http://inspirehep.net/record/1203843}{HEP entry}
%21 citations counted in INSPIRE as of 02 Jul 2015


%\cite{Chatrchyan:2012sga}
\item%{Chatrchyan:2012sga}
{\bf ``Measurement of the $ZZ$ production cross section and search for anomalous couplings in 2 l2l ' final states in $pp$ collisions at $\sqrt{s}=7$ TeV''}, 
  S.~Chatrchyan {\it et al.}  [CMS Collaboration], 
  %arXiv:1211.4890 [hep-ex]
    %{}10.1007/JHEP01(2013)063
JHEP {\bf 1301}063 (2013) %(Nov 2012)
%\href{http://inspirehep.net/record/1203454}{HEP entry}
%57 citations counted in INSPIRE as of 02 Jul 2015


%\cite{Chatrchyan:2012bba}
\item%{Chatrchyan:2012bba}
{\bf ``Search for new physics in events with photonsjetsand missing transverse energy in $pp$ collisions at $\sqrt{s}=7$ TeV''}, 
  S.~Chatrchyan {\it et al.}  [CMS Collaboration], 
  %arXiv:1211.4784 [hep-ex]
    %{}10.1007/JHEP03(2013)111
JHEP {\bf 1303}111 (2013) %(Nov 2012)
%\href{http://inspirehep.net/record/1203307}{HEP entry}
%36 citations counted in INSPIRE as of 02 Jul 2015


%\cite{Chatrchyan:2012jua}
\item%{Chatrchyan:2012jua}
{\bf ``Identification of b-quark jets with the CMS experiment''}, 
  S.~Chatrchyan {\it et al.}  [CMS Collaboration], 
  %arXiv:1211.4462 [hep-ex]
    %{}10.1088/1748-0221/8/04/P04013
JINST {\bf 8}P04013 (2013) %(Nov 2012)
%\href{http://inspirehep.net/record/1203133}{HEP entry}
%288 citations counted in INSPIRE as of 02 Jul 2015


%\cite{Chatrchyan:2012yca}
\item%{Chatrchyan:2012yca}
{\bf ``Search for $Z$ ' resonances decaying to $t\bar{t}$ in dilepton+jets final states in $pp$ collisions at $\sqrt{s}=7$ TeV''}, 
  S.~Chatrchyan {\it et al.}  [CMS Collaboration], 
  %arXiv:1211.3338 [hep-ex]
    %{}10.1103/PhysRevD.87.072002
Phys.\ Rev.\ D {\bf 87}no. 7072002 (2013) %(Nov 2012)
%\href{http://inspirehep.net/record/1202680}{HEP entry}
%43 citations counted in INSPIRE as of 02 Jul 2015


%\cite{Chatrchyan:2012sca}
\item%{Chatrchyan:2012sca}
{\bf ``Search for supersymmetry in final states with a single lepton$b$-quark jetsand missing transverse energy in proton-proton collisions at $\sqrt{s}=7$ TeV''}, 
  S.~Chatrchyan {\it et al.}  [CMS Collaboration], 
  %arXiv:1211.3143 [hep-ex]
    %{}10.1103/PhysRevD.87.052006
Phys.\ Rev.\ D {\bf 87}no. 5052006 (2013) %(Nov 2012)
%\href{http://inspirehep.net/record/1202674}{HEP entry}
%22 citations counted in INSPIRE as of 02 Jul 2015


%\cite{Chatrchyan:2012jna}
\item%{Chatrchyan:2012jna}
{\bf ``Search in leptonic channels for heavy resonances decaying to long-lived neutral particles''}, 
  S.~Chatrchyan {\it et al.}  [CMS Collaboration], 
  %arXiv:1211.2472 [hep-ex]
    %{}10.1007/JHEP02(2013)085
JHEP {\bf 1302}085 (2013) %(Nov 2012)
%\href{http://inspirehep.net/record/1202275}{HEP entry}
%37 citations counted in INSPIRE as of 02 Jul 2015


%\cite{Chatrchyan:2012saa}
\item%{Chatrchyan:2012saa}
{\bf ``Measurement of differential top-quark pair production cross sections in $pp$ colisions at $\sqrt{s}=7$ TeV''}, 
  S.~Chatrchyan {\it et al.}  [CMS Collaboration], 
  %arXiv:1211.2220 [hep-ex]
    %{}10.1140/epjc/s10052-013-2339-4
Eur.\ Phys.\ J.\ C {\bf 73}no. 32339 (2013) %(Nov 2012)
%\href{http://inspirehep.net/record/1201946}{HEP entry}
%116 citations counted in INSPIRE as of 02 Jul 2015


%\cite{Chatrchyan:2012wa}
\item%{Chatrchyan:2012wa}
{\bf ``Search for supersymmetry in final states with missing transverse energy and 012or at least 3 b-quark jets in 7 TeV pp collisions using the variable alphaT''}, 
  S.~Chatrchyan {\it et al.}  [CMS Collaboration], 
  %arXiv:1210.8115 [hep-ex]
    %{}10.1007/JHEP01(2013)077
JHEP {\bf 1301}077 (2013) %(Oct 2012)
%\href{http://inspirehep.net/record/1194120}{HEP entry}
%58 citations counted in INSPIRE as of 02 Jul 2015


%\cite{Chatrchyan:2012cg}
\item%{Chatrchyan:2012cg}
{\bf ``Search for a non-standard-model Higgs boson decaying to a pair of new light bosons in four-muon final states''}, 
  S.~Chatrchyan {\it et al.}  [CMS Collaboration], 
  %arXiv:1210.7619 [hep-ex]
    %{}10.1016/j.physletb.2013.09.009
Phys.\ Lett.\ B {\bf 726}564 (2013) %(Oct 2012)
%\href{http://inspirehep.net/record/1193937}{HEP entry}
%35 citations counted in INSPIRE as of 02 Jul 2015


%\cite{Chatrchyan:2012bd}
\item%{Chatrchyan:2012bd}
{\bf ``Measurement of the sum of $W W$ and $WZ$ production with $W+$dijet events in $pp$ collisions at $\sqrt{s}=7$ TeV''}, 
  S.~Chatrchyan {\it et al.}  [CMS Collaboration], 
  %arXiv:1210.7544 [hep-ex]
    %{}10.1140/epjc/s10052-013-2283-3
Eur.\ Phys.\ J.\ C {\bf 73}no. 22283 (2013) %(Oct 2012)
%\href{http://inspirehep.net/record/1193935}{HEP entry}
%52 citations counted in INSPIRE as of 02 Jul 2015


%\cite{Chatrchyan:2012af}
\item%{Chatrchyan:2012af}
{\bf ``Search for heavy quarks decaying into a top quark and a $W$ or $Z$ boson using lepton + jets events in $pp$ collisions at $\sqrt{s}$ = 7 TeV''}, 
  S.~Chatrchyan {\it et al.}  [CMS Collaboration], 
  %arXiv:1210.7471 [hep-ex]
    %{}10.1007/JHEP01(2013)154
JHEP {\bf 1301}154 (2013) %(Oct 2012)
%\href{http://inspirehep.net/record/1193934}{HEP entry}
%38 citations counted in INSPIRE as of 02 Jul 2015


%\cite{Chatrchyan:2012nj}
\item%{Chatrchyan:2012nj}
{\bf ``Measurement of the inelastic proton-proton cross section at $\sqrt{s}=7$ TeV''}, 
  S.~Chatrchyan {\it et al.}  [CMS Collaboration], 
  %arXiv:1210.6718 [hep-ex]
    %{}10.1016/j.physletb.2013.03.024
Phys.\ Lett.\ B {\bf 722}5 (2013) %(Oct 2012)
%\href{http://inspirehep.net/record/1193338}{HEP entry}
%62 citations counted in INSPIRE as of 02 Jul 2015


%\cite{Chatrchyan:2012sv}
\item%{Chatrchyan:2012sv}
{\bf ``Search for pair production of third-generation leptoquarks and top squarks in $pp$ collisions at $\sqrt{s}=7$ TeV''}, 
  S.~Chatrchyan {\it et al.}  [CMS Collaboration], 
  %arXiv:1210.5629 [hep-ex]
    %{}10.1103/PhysRevLett.110.081801
Phys.\ Rev.\ Lett.\  {\bf 110}no. 8081801 (2013) %(Oct 2012)
%\href{http://inspirehep.net/record/1192034}{HEP entry}
%42 citations counted in INSPIRE as of 02 Jul 2015


%\cite{Chatrchyan:2012st}
\item%{Chatrchyan:2012st}
{\bf ``Search for third-generation leptoquarks and scalar bottom quarks in $pp$ collisions at $\sqrt{s}=7$ TeV''}, 
  S.~Chatrchyan {\it et al.}  [CMS Collaboration], 
  %arXiv:1210.5627 [hep-ex]
    %{}10.1007/JHEP12(2012)055
JHEP {\bf 1212}055 (2012) %(Oct 2012)
%\href{http://inspirehep.net/record/1192033}{HEP entry}
%25 citations counted in INSPIRE as of 02 Jul 2015


%\cite{CMS:2012qk}
\item%{CMS:2012qk}
{\bf ``Observation of long-range near-side angular correlations in proton-lead collisions at the LHC''}, 
  S.~Chatrchyan {\it et al.}  [CMS Collaboration], 
  %arXiv:1210.5482 [nucl-ex]
    %{}10.1016/j.physletb.2012.11.025
Phys.\ Lett.\ B {\bf 718}795 (2013) %(Oct 2012)
%\href{http://inspirehep.net/record/1191899}{HEP entry}
%218 citations counted in INSPIRE as of 02 Jul 2015


%\cite{CMS:2012bw}
\item%{CMS:2012bw}
{\bf ``Observation of Z decays to four leptons with the CMS detector at the LHC''}, 
  S.~Chatrchyan {\it et al.}  [CMS Collaboration], 
  %arXiv:1210.3844 [hep-ex]
    %{}10.1007/JHEP12(2012)034
JHEP {\bf 1212}034 (2012) %(Oct 2012)
%\href{http://inspirehep.net/record/1190671}{HEP entry}
%31 citations counted in INSPIRE as of 02 Jul 2015


%\cite{CMS:2012ad}
\item%{CMS:2012ad}
{\bf ``Search for excited leptons in $pp$ collisions at $\sqrt{s}=7$ TeV''}, 
  S.~Chatrchyan {\it et al.}  [CMS Collaboration], 
  %arXiv:1210.2422 [hep-ex]
    %{}10.1016/j.physletb.2013.02.031
Phys.\ Lett.\ B {\bf 720}309 (2013) %(Oct 2012)
%\href{http://inspirehep.net/record/1189987}{HEP entry}
%12 citations counted in INSPIRE as of 02 Jul 2015


%\cite{CMS:2012zv}
\item%{CMS:2012zv}
{\bf ``Search for heavy neutrinos and W[R] bosons with right-handed couplings in a left-right symmetric model in pp collisions at sqrt(s) = 7 TeV''}, 
  S.~Chatrchyan {\it et al.}  [CMS Collaboration], 
  %arXiv:1210.2402 [hep-ex]
    %{}10.1103/PhysRevLett.109.261802
Phys.\ Rev.\ Lett.\  {\bf 109}261802 (2012) %(Oct 2012)
%\href{http://inspirehep.net/record/1189986}{HEP entry}
%27 citations counted in INSPIRE as of 02 Jul 2015


%\cite{CMS:2012yf}
\item%{CMS:2012yf}
{\bf ``Search for narrow resonances and quantum black holes in inclusive and $b$-tagged dijet mass spectra from $pp$ collisions at $\sqrt{s}=7$ TeV''}, 
  S.~Chatrchyan {\it et al.}  [CMS Collaboration], 
  %arXiv:1210.2387 [hep-ex]
    %{}10.1007/JHEP01(2013)013
JHEP {\bf 1301}013 (2013) %(Oct 2012)
%\href{http://inspirehep.net/record/1189823}{HEP entry}
%44 citations counted in INSPIRE as of 02 Jul 2015


%\cite{CMS:2012xi}
\item%{CMS:2012xi}
{\bf ``Search for fractionally charged particles in $pp$ collisions at $\sqrt{s}=7$ TeV''}, 
  S.~Chatrchyan {\it et al.}  [CMS Collaboration], 
  %arXiv:1210.2311 [hep-ex]
    %{}10.1103/PhysRevD.87.092008
Phys.\ Rev.\ D {\bf 87}no. 9092008 (2013) %(Oct 2012)
%\href{http://inspirehep.net/record/1189819}{HEP entry}
%11 citations counted in INSPIRE as of 02 Jul 2015


%\cite{CMS:2012un}
\item%{CMS:2012un}
{\bf ``Search for supersymmetry in events with photons and low missing transverse energy in $pp$ collisions at $\sqrt{s}=7$ TeV''}, 
  S.~Chatrchyan {\it et al.}  [CMS Collaboration], 
  %arXiv:1210.2052 [hep-ex]
    %{}10.1016/j.physletb.2012.12.055
Phys.\ Lett.\ B {\bf 719}42 (2013) %(Oct 2012)
%\href{http://inspirehep.net/record/1189815}{HEP entry}
%26 citations counted in INSPIRE as of 02 Jul 2015


%\cite{CMS:2012ra}
\item%{CMS:2012ra}
{\bf ``Search for heavy lepton partners of neutrinos in proton-proton collisions in the context of the type III seesaw mechanism''}, 
  S.~Chatrchyan {\it et al.}  [CMS Collaboration], 
  %arXiv:1210.1797 [hep-ex]
    %{}10.1016/j.physletb.2012.10.070
Phys.\ Lett.\ B {\bf 718}348 (2012) %(Oct 2012)
%\href{http://inspirehep.net/record/1189663}{HEP entry}
%28 citations counted in INSPIRE as of 02 Jul 2015


%\cite{Chatrchyan:2012ub}
\item%{Chatrchyan:2012ub}
{\bf ``Measurement of the relative prompt production rate of chi(c2) and chi(c1) in $pp$ collisions at $\sqrt{s}=7$ TeV''}, 
  S.~Chatrchyan {\it et al.}  [CMS Collaboration], 
  %arXiv:1210.0875 [hep-ex]
    %{}10.1140/epjc/s10052-012-2251-3
Eur.\ Phys.\ J.\ C {\bf 72}2251 (2012) %(Oct 2012)
%\href{http://inspirehep.net/record/1189050}{HEP entry}
%35 citations counted in INSPIRE as of 02 Jul 2015


%\cite{Chatrchyan:2012tw}
\item%{Chatrchyan:2012tw}
{\bf ``Search for anomalous production of highly boosted $Z$ bosons decaying to dimuons in $pp$ collisions at $\sqrt{s}=7$ TeV''}, 
  S.~Chatrchyan {\it et al.}  [CMS Collaboration], 
  %arXiv:1210.0867 [hep-ex]
    %{}10.1016/j.physletb.2013.03.037
Phys.\ Lett.\ B {\bf 722}28 (2013) %(Oct 2012)
%\href{http://inspirehep.net/record/1189049}{HEP entry}
%6 citations counted in INSPIRE as of 02 Jul 2015


%\cite{Chatrchyan:2012pka}
\item%{Chatrchyan:2012pka}
{\bf ``Search for electroweak production of charginos and neutralinos using leptonic final states in $pp$ collisions at $\sqrt{s}=7$ TeV''}, 
  S.~Chatrchyan {\it et al.}  [CMS Collaboration], 
  %arXiv:1209.6620 [hep-ex]
    %{}10.1007/JHEP11(2012)147
JHEP {\bf 1211}147 (2012) %(Sep 2012)
%\href{http://inspirehep.net/record/1188683}{HEP entry}
%65 citations counted in INSPIRE as of 02 Jul 2015


%\cite{Chatrchyan:2012ep}
\item%{Chatrchyan:2012ep}
{\bf ``Measurement of the single-top-quark $t$-channel cross section in $pp$ collisions at $\sqrt{s}=7$ TeV''}, 
  S.~Chatrchyan {\it et al.}  [CMS Collaboration], 
  %arXiv:1209.4533 [hep-ex]
    %{}10.1007/JHEP12(2012)035
JHEP {\bf 1212}035 (2012) %(Sep 2012)
%\href{http://inspirehep.net/record/1186734}{HEP entry}
%123 citations counted in INSPIRE as of 02 Jul 2015


%\cite{Chatrchyan:2012cx}
\item%{Chatrchyan:2012cx}
{\bf ``Search for resonant $t\bar{t}$ production in lepton+jets events in $pp$ collisions at $\sqrt{s}=7$ TeV''}, 
  S.~Chatrchyan {\it et al.}  [CMS Collaboration], 
  %arXiv:1209.4397 [hep-ex]
    %{}10.1007/JHEP12(2012)015
JHEP {\bf 1212}015 (2012) %(Sep 2012)
%\href{http://inspirehep.net/record/1186730}{HEP entry}
%51 citations counted in INSPIRE as of 02 Jul 2015


%\cite{Chatrchyan:2012qr}
\item%{Chatrchyan:2012qr}
{\bf ``Search for the standard model Higgs boson produced in association with $W$ and $Z$ bosons in $pp$ collisions at $\sqrt{s}=7$ TeV''}, 
  S.~Chatrchyan {\it et al.}  [CMS Collaboration], 
  %arXiv:1209.3937 [hep-ex]
    %{}10.1007/JHEP11(2012)088
JHEP {\bf 1211}088 (2012) %(Sep 2012)
%\href{http://inspirehep.net/record/1186385}{HEP entry}
%10 citations counted in INSPIRE as of 02 Jul 2015


%\cite{Chatrchyan:2012baa}
\item%{Chatrchyan:2012baa}
{\bf ``Search for a narrow spin-2 resonance decaying to a pair of Z vector bosons in the semileptonic final state''}, 
  S.~Chatrchyan {\it et al.}  [CMS Collaboration], 
  %arXiv:1209.3807 [hep-ex]
    %{}10.1016/j.physletb.2012.11.063
Phys.\ Lett.\ B {\bf 718}1208 (2013) %(Sep 2012)
%\href{http://inspirehep.net/record/1186381}{HEP entry}
%16 citations counted in INSPIRE as of 02 Jul 2015


%\cite{Chatrchyan:2012zca}
\item%{Chatrchyan:2012zca}
{\bf ``Evidence for associated production of a single top quark and W boson in $pp$ collisions at $\sqrt{s}$ = 7 TeV''}, 
  S.~Chatrchyan {\it et al.}  [CMS Collaboration], 
  %arXiv:1209.3489 [hep-ex]
    %{}10.1103/PhysRevLett.110.022003
Phys.\ Rev.\ Lett.\  {\bf 110}022003 (2013) %(Sep 2012)
%\href{http://inspirehep.net/record/1185781}{HEP entry}
%86 citations counted in INSPIRE as of 02 Jul 2015


%\cite{Chatrchyan:2012woa}
\item%{Chatrchyan:2012woa}
{\bf ``Measurement of the $Y(1S)Y(2S)$ and $Y(3S)$ polarizations in $pp$ collisions at $\sqrt{s}=7$ TeV''}, 
  S.~Chatrchyan {\it et al.}  [CMS Collaboration], 
  %arXiv:1209.2922 [hep-ex]
    %{}10.1103/PhysRevLett.110.081802
Phys.\ Rev.\ Lett.\  {\bf 110}no. 8081802 (2013) %(Sep 2012)
%\href{http://inspirehep.net/record/1185414}{HEP entry}
%78 citations counted in INSPIRE as of 02 Jul 2015


%\cite{Chatrchyan:2012ea}
\item%{Chatrchyan:2012ea}
{\bf ``Measurement of the top-quark mass in $t\bar{t}$ events with dilepton final states in $pp$ collisions at $\sqrt{s}=7$ TeV''}, 
  S.~Chatrchyan {\it et al.}  [CMS Collaboration], 
  %arXiv:1209.2393 [hep-ex]
    %{}10.1140/epjc/s10052-012-2202-z
Eur.\ Phys.\ J.\ C {\bf 72}2202 (2012) %(Sep 2012)
%\href{http://inspirehep.net/record/1185104}{HEP entry}
%65 citations counted in INSPIRE as of 02 Jul 2015


%\cite{Chatrchyan:2012cz}
\item%{Chatrchyan:2012cz}
{\bf ``Measurement of the top-quark mass in $t\bar{t}$ events with lepton+jets final states in $pp$ collisions at $\sqrt{s}=7$ TeV''}, 
  S.~Chatrchyan {\it et al.}  [CMS Collaboration], 
  %arXiv:1209.2319 [hep-ex]
    %{}10.1007/JHEP12(2012)105
JHEP {\bf 1212}105 (2012) %(Sep 2012)
%\href{http://inspirehep.net/record/1185101}{HEP entry}
%81 citations counted in INSPIRE as of 02 Jul 2015


%\cite{Chatrchyan:2012vc}
\item%{Chatrchyan:2012vc}
{\bf ``Observation of a diffractive contribution to dijet production in proton-proton collisions at $\sqrt{s}=7$ TeV''}, 
  S.~Chatrchyan {\it et al.}  [CMS Collaboration], 
  %arXiv:1209.1805 [hep-ex]
    %{}10.1103/PhysRevD.87.012006
Phys.\ Rev.\ D {\bf 87}no. 1012006 (2013) %(Sep 2012)
%\href{http://inspirehep.net/record/1184941}{HEP entry}
%22 citations counted in INSPIRE as of 02 Jul 2015


%\cite{Chatrchyan:2012tv}
\item%{Chatrchyan:2012tv}
{\bf ``Search for exclusive or semi-exclusive photon pair production and observation of exclusive and semi-exclusive electron pair production in $pp$ collisions at $\sqrt{s}=7$ TeV''}, 
  S.~Chatrchyan {\it et al.}  [CMS Collaboration], 
  %arXiv:1209.1666 [hep-ex]
    %{}10.1007/JHEP11(2012)080
JHEP {\bf 1211}080 (2012) %(Sep 2012)
%\href{http://inspirehep.net/record/1184938}{HEP entry}
%41 citations counted in INSPIRE as of 02 Jul 2015


%\cite{Chatrchyan:2012fp}
\item%{Chatrchyan:2012fp}
{\bf ``Combined search for the quarks of a sequential fourth generation''}, 
  S.~Chatrchyan {\it et al.}  [CMS Collaboration], 
  %arXiv:1209.1062 [hep-ex]
    %{}10.1103/PhysRevD.86.112003
Phys.\ Rev.\ D {\bf 86}112003 (2012) %(Sep 2012)
%\href{http://inspirehep.net/record/1184487}{HEP entry}
%38 citations counted in INSPIRE as of 02 Jul 2015


%\cite{Chatrchyan:2012vu}
\item%{Chatrchyan:2012vu}
{\bf ``Search for pair produced fourth-generation up-type quarks in $pp$ collisions at $\sqrt{s}=7$ TeV with a lepton in the final state''}, 
  S.~Chatrchyan {\it et al.}  [CMS Collaboration], 
  %arXiv:1209.0471 [hep-ex]
    %{}10.1016/j.physletb.2012.10.038
Phys.\ Lett.\ B {\bf 718}307 (2012) %(Sep 2012)
%\href{http://inspirehep.net/record/1184341}{HEP entry}
%50 citations counted in INSPIRE as of 02 Jul 2015


%\cite{Chatrchyan:2012rg}
\item%{Chatrchyan:2012rg}
{\bf ``Search for supersymmetry in events with b-quark jets and missing transverse energy in pp collisions at 7 TeV''}, 
  S.~Chatrchyan {\it et al.}  [CMS Collaboration], 
  %arXiv:1208.4859 [hep-ex]
    %{}10.1103/PhysRevD.86.072010
Phys.\ Rev.\ D {\bf 86}072010 (2012) %(Aug 2012)
%\href{http://inspirehep.net/record/1181767}{HEP entry}
%34 citations counted in INSPIRE as of 02 Jul 2015


%\cite{Chatrchyan:2012jra}
\item%{Chatrchyan:2012jra}
{\bf ``Study of the dijet mass spectrum in $pp \to W +$ jets events at $\sqrt{s}=7$ TeV''}, 
  S.~Chatrchyan {\it et al.}  [CMS Collaboration], 
  %arXiv:1208.3477 [hep-ex]
    %{}10.1103/PhysRevLett.109.251801
Phys.\ Rev.\ Lett.\  {\bf 109}251801 (2012) %(Aug 2012)
%\href{http://inspirehep.net/record/1128019}{HEP entry}
%21 citations counted in INSPIRE as of 02 Jul 2015


%\cite{Chatrchyan:2012uxa}
\item%{Chatrchyan:2012uxa}
{\bf ``Search for three-jet resonances in $pp$ collisions at $\sqrt{s}=7$ TeV''}, 
  S.~Chatrchyan {\it et al.}  [CMS Collaboration], 
  %arXiv:1208.2931 [hep-ex]
    %{}10.1016/j.physletb.2012.10.048
Phys.\ Lett.\ B {\bf 718}329 (2012) %(Aug 2012)
%\href{http://inspirehep.net/record/1127510}{HEP entry}
%38 citations counted in INSPIRE as of 02 Jul 2015


%\cite{Chatrchyan:2012lxa}
\item%{Chatrchyan:2012lxa}
{\bf ``Observation of sequential Upsilon suppression in PbPb collisions''}, 
  S.~Chatrchyan {\it et al.}  [CMS Collaboration], 
  %arXiv:1208.2826 [nucl-ex]
    %{}10.1103/PhysRevLett.109.222301
Phys.\ Rev.\ Lett.\  {\bf 109}222301 (2012) %(Aug 2012)
%\href{http://inspirehep.net/record/1127501}{HEP entry}
%134 citations counted in INSPIRE as of 02 Jul 2015


%\cite{Chatrchyan:2012bra}
\item%{Chatrchyan:2012bra}
{\bf ``Measurement of the $t\bar{t}$ production cross section in the dilepton channel in $pp$ collisions at $\sqrt{s}=7$ TeV''}, 
  S.~Chatrchyan {\it et al.}  [CMS Collaboration], 
  %arXiv:1208.2671 [hep-ex]
    %{}10.1007/JHEP11(2012)067
JHEP {\bf 1211}067 (2012) %(Aug 2012)
%\href{http://inspirehep.net/record/1127335}{HEP entry}
%121 citations counted in INSPIRE as of 02 Jul 2015


%\cite{Chatrchyan:2012vqa}
\item%{Chatrchyan:2012vqa}
{\bf ``Measurement of the azimuthal anisotropy of neutral pions in PbPb collisions at $\sqrt{s_{NN}}=2.76$ TeV''}, 
  S.~Chatrchyan {\it et al.}  [CMS Collaboration], 
  %arXiv:1208.2470 [nucl-ex]
    %{}10.1103/PhysRevLett.110.042301
Phys.\ Rev.\ Lett.\  {\bf 110}no. 4042301 (2013) %(Aug 2012)
%\href{http://inspirehep.net/record/1127329}{HEP entry}
%18 citations counted in INSPIRE as of 02 Jul 2015


%\cite{Chatrchyan:2012hqa}
\item%{Chatrchyan:2012hqa}
{\bf ``Search for flavor changing neutral currents in top quark decays in pp collisions at 7 TeV''}, 
  S.~Chatrchyan {\it et al.}  [CMS Collaboration], 
  %arXiv:1208.0957 [hep-ex]
    %{}10.1016/j.physletb.2012.12.045
Phys.\ Lett.\ B {\bf 718}1252 (2013) %(Aug 2012)
%\href{http://inspirehep.net/record/1125963}{HEP entry}
%42 citations counted in INSPIRE as of 02 Jul 2015


%\cite{Chatrchyan:2012gqa}
\item%{Chatrchyan:2012gqa}
{\bf ``Search for a $W$ ' boson decaying to a bottom quark and a top quark in $pp$ collisions at $\sqrt{s}=7$ TeV''}, 
  S.~Chatrchyan {\it et al.}  [CMS Collaboration], 
  %arXiv:1208.0956 [hep-ex]
    %{}10.1016/j.physletb.2012.12.008
Phys.\ Lett.\ B {\bf 718}1229 (2013) %(Aug 2012)
%\href{http://inspirehep.net/record/1125962}{HEP entry}
%45 citations counted in INSPIRE as of 02 Jul 2015


%\cite{Chatrchyan:2012ufa}
\item%{Chatrchyan:2012ufa}
{\bf ``Observation of a new boson at a mass of 125 GeV with the CMS experiment at the LHC''}, 
  S.~Chatrchyan {\it et al.}  [CMS Collaboration], 
  %arXiv:1207.7235 [hep-ex]
    %{}10.1016/j.physletb.2012.08.021
Phys.\ Lett.\ B {\bf 716}30 (2012) %(Jul 2012)
%\href{http://inspirehep.net/record/1124338}{HEP entry}
%4498 citations counted in INSPIRE as of 02 Jul 2015


%\cite{Chatrchyan:2012fla}
\item%{Chatrchyan:2012fla}
{\bf ``Search for heavy Majorana neutrinos in $\mu^+\mu^+[\mu^-\mu^-]$ and $e^+e^+[e^-e^-]$ events in $pp$ collisions at $\sqrt{s} = 7$ TeV''}, 
  S.~Chatrchyan {\it et al.}  [CMS Collaboration], 
  %arXiv:1207.6079 [hep-ex]
    %{}10.1016/j.physletb.2012.09.012
Phys.\ Lett.\ B {\bf 717}109 (2012) %(Jul 2012)
%\href{http://inspirehep.net/record/1123803}{HEP entry}
%40 citations counted in INSPIRE as of 02 Jul 2015


%\cite{Chatrchyan:2012vza}
\item%{Chatrchyan:2012vza}
{\bf ``Search for pair production of first- and second-generation scalar leptoquarks in $pp$ collisions at $\sqrt{s}= 7$ TeV''}, 
  S.~Chatrchyan {\it et al.}  [CMS Collaboration], 
  %arXiv:1207.5406 [hep-ex]
    %{}10.1103/PhysRevD.86.052013
Phys.\ Rev.\ D {\bf 86}052013 (2012) %(Jul 2012)
%\href{http://inspirehep.net/record/1123507}{HEP entry}
%44 citations counted in INSPIRE as of 02 Jul 2015


%\cite{Chatrchyan:2012qb}
\item%{Chatrchyan:2012qb}
{\bf ``Study of the inclusive production of charged pionskaonsand protons in $pp$ collisions at $\sqrt{s}=0.9$2.76and 7 TeV''}, 
  S.~Chatrchyan {\it et al.}  [CMS Collaboration], 
  %arXiv:1207.4724 [hep-ex]
    %{}10.1140/epjc/s10052-012-2164-1
Eur.\ Phys.\ J.\ C {\bf 72}2164 (2012) %(Jul 2012)
%\href{http://inspirehep.net/record/1123117}{HEP entry}
%63 citations counted in INSPIRE as of 02 Jul 2015


%\cite{Chatrchyan:2012dc}
\item%{Chatrchyan:2012dc}
{\bf ``Forward-backward asymmetry of Drell-Yan lepton pairs in $pp$ collisions at $\sqrt{s} = 7$ TeV''}, 
  S.~Chatrchyan {\it et al.}  [CMS Collaboration], 
  %arXiv:1207.3973 [hep-ex]
    %{}10.1016/j.physletb.2012.10.082
Phys.\ Lett.\ B {\bf 718}752 (2013) %(Jul 2012)
%\href{http://inspirehep.net/record/1122847}{HEP entry}
%17 citations counted in INSPIRE as of 02 Jul 2015


%\cite{Chatrchyan:2012ya}
\item%{Chatrchyan:2012ya}
{\bf ``A search for a doubly-charged Higgs boson in $pp$ collisions at $\sqrt{s}=7$ TeV''}, 
  S.~Chatrchyan {\it et al.}  [CMS Collaboration], 
  %arXiv:1207.2666 [hep-ex]
    %{}10.1140/epjc/s10052-012-2189-5
Eur.\ Phys.\ J.\ C {\bf 72}2189 (2012) %(Jul 2012)
%\href{http://inspirehep.net/record/1122035}{HEP entry}
%106 citations counted in INSPIRE as of 02 Jul 2015


%\cite{Chatrchyan:2012tt}
\item%{Chatrchyan:2012tt}
{\bf ``Measurement of the underlying event activity in $pp$ collisions at $\sqrt{s} = 0.9$ and 7 TeV with the novel jet-area/median approach''}, 
  S.~Chatrchyan {\it et al.}  [CMS Collaboration], 
  %arXiv:1207.2392 [hep-ex]
    %{}10.1007/JHEP08(2012)130
JHEP {\bf 1208}130 (2012) %(Jul 2012)
%\href{http://inspirehep.net/record/1121876}{HEP entry}
%20 citations counted in INSPIRE as of 02 Jul 2015


%\cite{Chatrchyan:2012lia}
\item%{Chatrchyan:2012lia}
{\bf ``Search for new physics in the multijet and missing transverse momentum final state in proton-proton collisions at $\sqrt{s} = 7$ TeV''}, 
  S.~Chatrchyan {\it et al.}  [CMS Collaboration], 
  %arXiv:1207.1898 [hep-ex]
    %{}10.1103/PhysRevLett.109.171803
Phys.\ Rev.\ Lett.\  {\bf 109}171803 (2012) %(Jul 2012)
%\href{http://inspirehep.net/record/1121703}{HEP entry}
%155 citations counted in INSPIRE as of 02 Jul 2015


%\cite{Chatrchyan:2012jx}
\item%{Chatrchyan:2012jx}
{\bf ``Search for supersymmetry in hadronic final states using MT2 in $pp$ collisions at $\sqrt{s} = 7$ TeV''}, 
  S.~Chatrchyan {\it et al.}  [CMS Collaboration], 
  %arXiv:1207.1798 [hep-ex]
    %{}10.1007/JHEP10(2012)018
JHEP {\bf 1210}018 (2012) %(Jul 2012)
%\href{http://inspirehep.net/record/1121700}{HEP entry}
%134 citations counted in INSPIRE as of 02 Jul 2015


%\cite{Chatrchyan:2012vva}
\item%{Chatrchyan:2012vva}
{\bf ``Search for a fermiophobic Higgs boson in $pp$ collisions at $\sqrt{s}=7$ TeV''}, 
  S.~Chatrchyan {\it et al.}  [CMS Collaboration], 
  %arXiv:1207.1130 [hep-ex]
    %{}10.1007/JHEP09(2012)111
JHEP {\bf 1209}111 (2012) %(Jul 2012)
%\href{http://inspirehep.net/record/1121375}{HEP entry}
%22 citations counted in INSPIRE as of 02 Jul 2015


%\cite{Chatrchyan:2012ir}
\item%{Chatrchyan:2012ir}
{\bf ``Search for new physics with long-lived particles decaying to photons and missing energy in $pp$ collisions at $\sqrt{s}=7$ TeV''}, 
  S.~Chatrchyan {\it et al.}  [CMS Collaboration], 
  %arXiv:1207.0627 [hep-ex]
    %{}10.1007/JHEP11(2012)172
JHEP {\bf 1211}172 (2012) %(Jul 2012)
%\href{http://inspirehep.net/record/1120997}{HEP entry}
%12 citations counted in INSPIRE as of 02 Jul 2015


%\cite{Chatrchyan:2012dxa}
\item%{Chatrchyan:2012dxa}
{\bf ``Search for stopped long-lived particles produced in $pp$ collisions at $\sqrt{s}=7$ TeV''}, 
  S.~Chatrchyan {\it et al.}  [CMS Collaboration], 
  %arXiv:1207.0106 [hep-ex]
    %{}10.1007/JHEP08(2012)026
JHEP {\bf 1208}026 (2012) %(Jul 2012)
%\href{http://inspirehep.net/record/1120733}{HEP entry}
%26 citations counted in INSPIRE as of 02 Jul 2015


%\cite{Chatrchyan:2012cxa}
\item%{Chatrchyan:2012cxa}
{\bf ``Inclusive and differential measurements of the $t \bar{t}$ charge asymmetry in proton-proton collisions at 7 TeV''}, 
  S.~Chatrchyan {\it et al.}  [CMS Collaboration], 
  %arXiv:1207.0065 [hep-ex]
    %{}10.1016/j.physletb.2012.09.028
Phys.\ Lett.\ B {\bf 717}129 (2012) %(Jul 2012)
%\href{http://inspirehep.net/record/1120732}{HEP entry}
%90 citations counted in INSPIRE as of 02 Jul 2015


%\cite{Chatrchyan:2012am}
\item%{Chatrchyan:2012am}
{\bf ``Search for a light pseudoscalar Higgs boson in the dimuon decay channel in $pp$ collisions at $\sqrt{s}=7$ TeV''}, 
  S.~Chatrchyan {\it et al.}  [CMS Collaboration], 
  %arXiv:1206.6326 [hep-ex]
    %{}10.1103/PhysRevLett.109.121801
Phys.\ Rev.\ Lett.\  {\bf 109}121801 (2012) %(Jun 2012)
%\href{http://inspirehep.net/record/1120142}{HEP entry}
%39 citations counted in INSPIRE as of 02 Jul 2015


%\cite{Chatrchyan:2012me}
\item%{Chatrchyan:2012me}
{\bf ``Search for dark matter and large extra dimensions in monojet events in $pp$ collisions at $\sqrt{s}=7$ TeV''}, 
  S.~Chatrchyan {\it et al.}  [CMS Collaboration], 
  %arXiv:1206.5663 [hep-ex]
    %{}10.1007/JHEP09(2012)094
JHEP {\bf 1209}094 (2012) %(Jun 2012)
%\href{http://inspirehep.net/record/1119567}{HEP entry}
%201 citations counted in INSPIRE as of 02 Jul 2015


%\cite{Chatrchyan:2012xi}
\item%{Chatrchyan:2012xi}
{\bf ``Performance of CMS muon reconstruction in $pp$ collision events at $\sqrt{s}=7$ TeV''}, 
  S.~Chatrchyan {\it et al.}  [CMS Collaboration], 
  %arXiv:1206.4071 [physics.ins-det]
    %{}10.1088/1748-0221/7/10/P10002
JINST {\bf 7}P10002 (2012) %(Jun 2012)
%\href{http://inspirehep.net/record/1118729}{HEP entry}
%293 citations counted in INSPIRE as of 02 Jul 2015


%\cite{Chatrchyan:2012te}
\item%{Chatrchyan:2012te}
{\bf ``Search for new physics in events with opposite-sign leptonsjetsand missing transverse energy in $pp$ collisions at $\sqrt{s}=7$ TeV''}, 
  S.~Chatrchyan {\it et al.}  [CMS Collaboration], 
  %arXiv:1206.3949 [hep-ex]
    %{}10.1016/j.physletb.2012.11.036
Phys.\ Lett.\ B {\bf 718}815 (2013) %(Jun 2012)
%\href{http://inspirehep.net/record/1118578}{HEP entry}
%52 citations counted in INSPIRE as of 02 Jul 2015


%\cite{Chatrchyan:2012su}
\item%{Chatrchyan:2012su}
{\bf ``Search for charge-asymmetric production of $W$ ' bosons in top pair + jet events from $pp$ collisions at $\sqrt{s}=7$ TeV''}, 
  S.~Chatrchyan {\it et al.}  [CMS Collaboration], 
  %arXiv:1206.3921 [hep-ex]
    %{}10.1016/j.physletb.2012.09.048
Phys.\ Lett.\ B {\bf 717}351 (2012) %(Jun 2012)
%\href{http://inspirehep.net/record/1118577}{HEP entry}
%26 citations counted in INSPIRE as of 02 Jul 2015


%\cite{Chatrchyan:2012xt}
\item%{Chatrchyan:2012xt}
{\bf ``Measurement of the electron charge asymmetry in inclusive $W$ production in $pp$ collisions at $\sqrt{s}=7$ TeV''}, 
  S.~Chatrchyan {\it et al.}  [CMS Collaboration], 
  %arXiv:1206.2598 [hep-ex]
    %{}10.1103/PhysRevLett.109.111806
Phys.\ Rev.\ Lett.\  {\bf 109}111806 (2012) %(Jun 2012)
%\href{http://inspirehep.net/record/1118047}{HEP entry}
%49 citations counted in INSPIRE as of 02 Jul 2015


%\cite{Chatrchyan:2012it}
\item%{Chatrchyan:2012it}
{\bf ``Search for narrow resonances in dilepton mass spectra in $pp$ collisions at $\sqrt{s}=7$ TeV''}, 
  S.~Chatrchyan {\it et al.}  [CMS Collaboration], 
  %arXiv:1206.1849 [hep-ex]
    %{}10.1016/j.physletb.2012.06.051
Phys.\ Lett.\ B {\bf 714}158 (2012) %(Jun 2012)
%\href{http://inspirehep.net/record/1117706}{HEP entry}
%103 citations counted in INSPIRE as of 02 Jul 2015


%\cite{Chatrchyan:2012hd}
\item%{Chatrchyan:2012hd}
{\bf ``Search for high mass resonances decaying into $\tau^-$ lepton pairs in $pp$ collisions at $\sqrt{s}=7$ TeV''}, 
  S.~Chatrchyan {\it et al.}  [CMS Collaboration], 
  %arXiv:1206.1725 [hep-ex]
    %{}10.1016/j.physletb.2012.07.062
Phys.\ Lett.\ B {\bf 716}82 (2012) %(Jun 2012)
%\href{http://inspirehep.net/record/1117702}{HEP entry}
%39 citations counted in INSPIRE as of 02 Jul 2015


%\cite{Chatrchyan:2012kk}
\item%{Chatrchyan:2012kk}
{\bf ``Search for a $W^\prime$ or Techni-$\rho$ Decaying into $WZ$ in $pp$ Collisions at $\sqrt{s}=7$ TeV''}, 
  S.~Chatrchyan {\it et al.}  [CMS Collaboration], 
  %arXiv:1206.0433 [hep-ex]
    %{}10.1103/PhysRevLett.109.141801
Phys.\ Rev.\ Lett.\  {\bf 109}141801 (2012) %(Jun 2012)
%\href{http://inspirehep.net/record/1117012}{HEP entry}
%38 citations counted in INSPIRE as of 02 Jul 2015


%\cite{Chatrchyan:2012ira}
\item%{Chatrchyan:2012ira}
{\bf ``Search for new physics with same-sign isolated dilepton events with jets and missing transverse energy''}, 
  S.~Chatrchyan {\it et al.}  [CMS Collaboration], 
  %arXiv:1205.6615 [hep-ex]
    %{}10.1103/PhysRevLett.109.071803
Phys.\ Rev.\ Lett.\  {\bf 109}071803 (2012) %(May 2012)
%\href{http://inspirehep.net/record/1116526}{HEP entry}
%64 citations counted in INSPIRE as of 02 Jul 2015


%\cite{Chatrchyan:2012nt}
\item%{Chatrchyan:2012nt}
{\bf ``Study of $W$ boson production in PbPb and $pp$ collisions at $\sqrt{s_{NN}}=2.76$ TeV''}, 
  S.~Chatrchyan {\it et al.}  [CMS Collaboration], 
  %arXiv:1205.6334 [nucl-ex]
    %{}10.1016/j.physletb.2012.07.025
Phys.\ Lett.\ B {\bf 715}66 (2012) %(May 2012)
%\href{http://inspirehep.net/record/1116412}{HEP entry}
%61 citations counted in INSPIRE as of 02 Jul 2015


%\cite{Chatrchyan:2012gw}
\item%{Chatrchyan:2012gw}
{\bf ``Measurement of jet fragmentation into charged particles in $pp$ and PbPb collisions at $\sqrt{s_{NN}}=2.76$ TeV''}, 
  S.~Chatrchyan {\it et al.}  [CMS Collaboration], 
  %arXiv:1205.5872 [nucl-ex]
    %{}10.1007/JHEP10(2012)087
JHEP {\bf 1210}087 (2012) %(May 2012)
%\href{http://inspirehep.net/record/1116250}{HEP entry}
%81 citations counted in INSPIRE as of 02 Jul 2015


%\cite{Chatrchyan:2012vca}
\item%{Chatrchyan:2012vca}
{\bf ``Search for a light charged Higgs boson in top quark decays in $pp$ collisions at $\sqrt{s}=7$ TeV''}, 
  S.~Chatrchyan {\it et al.}  [CMS Collaboration], 
  %arXiv:1205.5736 [hep-ex]
    %{}10.1007/JHEP07(2012)143
JHEP {\bf 1207}143 (2012) %(May 2012)
%\href{http://inspirehep.net/record/1116149}{HEP entry}
%159 citations counted in INSPIRE as of 02 Jul 2015


%\cite{Chatrchyan:2012sa}
\item%{Chatrchyan:2012sa}
{\bf ``Search for new physics in events with same-sign dileptons and $b$-tagged jets in $pp$ collisions at $\sqrt{s}=7$ TeV''}, 
  S.~Chatrchyan {\it et al.}  [CMS Collaboration], 
  %arXiv:1205.3933 [hep-ex]
    %{}10.1007/JHEP08(2012)110
JHEP {\bf 1208}110 (2012) %(May 2012)
%\href{http://inspirehep.net/record/1115185}{HEP entry}
%83 citations counted in INSPIRE as of 02 Jul 2015


%\cite{Chatrchyan:2012mb}
\item%{Chatrchyan:2012mb}
{\bf ``Measurement of the pseudorapidity and centrality dependence of the transverse energy density in PbPb collisions at $\sqrt{s_{NN}}=2.76$ TeV''}, 
  S.~Chatrchyan {\it et al.}  [CMS Collaboration], 
  %arXiv:1205.2488 [nucl-ex]
    %{}10.1103/PhysRevLett.109.152303
Phys.\ Rev.\ Lett.\  {\bf 109}152303 (2012) %(May 2012)
%\href{http://inspirehep.net/record/1114315}{HEP entry}
%39 citations counted in INSPIRE as of 02 Jul 2015


%\cite{Chatrchyan:2012xg}
\item%{Chatrchyan:2012xg}
{\bf ``Measurement of the $Lambda_b$ cross section and the $\bar{\Lambda}$( $b^{)}$ to $Lambda_b$ ratio with $Lambda_b$ to J/Psi $\Lambda$ decays in $pp$ collisions at $\sqrt{s}=7$ TeV''}, 
  S.~Chatrchyan {\it et al.}  [CMS Collaboration], 
  %arXiv:1205.0594 [hep-ex]
    %{}10.1016/j.physletb.2012.05.063
Phys.\ Lett.\ B {\bf 714}136 (2012) %(May 2012)
%\href{http://inspirehep.net/record/1113442}{HEP entry}
%38 citations counted in INSPIRE as of 02 Jul 2015


%\cite{Chatrchyan:2012sp}
\item%{Chatrchyan:2012sp}
{\bf ``Search for heavy long-lived charged particles in $pp$ collisions at $\sqrt{s}=7$ TeV''}, 
  S.~Chatrchyan {\it et al.}  [CMS Collaboration], 
  %arXiv:1205.0272 [hep-ex]
    %{}10.1016/j.physletb.2012.06.023
Phys.\ Lett.\ B {\bf 713}408 (2012) %(May 2012)
%\href{http://inspirehep.net/record/1113310}{HEP entry}
%83 citations counted in INSPIRE as of 02 Jul 2015


%\cite{Chatrchyan:2012gt}
\item%{Chatrchyan:2012gt}
{\bf ``Studies of jet quenching using isolated-photon+jet correlations in PbPb and $pp$ collisions at $\sqrt{s_{NN}}=2.76$ TeV''}, 
  S.~Chatrchyan {\it et al.}  [CMS Collaboration], 
  %arXiv:1205.0206 [nucl-ex]
    %{}10.1016/j.physletb.2012.11.003
Phys.\ Lett.\ B {\bf 718}773 (2013) %(May 2012)
%\href{http://inspirehep.net/record/1112986}{HEP entry}
%108 citations counted in INSPIRE as of 02 Jul 2015


%\cite{Chatrchyan:2012ni}
\item%{Chatrchyan:2012ni}
{\bf ``Observation of a new Xi(b) baryon''}, 
  S.~Chatrchyan {\it et al.}  [CMS Collaboration], 
  %arXiv:1204.5955 [hep-ex]
    %{}10.1103/PhysRevLett.108.252002
Phys.\ Rev.\ Lett.\  {\bf 108}252002 (2012) %(Apr 2012)
%\href{http://inspirehep.net/record/1112562}{HEP entry}
%53 citations counted in INSPIRE as of 02 Jul 2015


%\cite{Chatrchyan:2012mea}
\item%{Chatrchyan:2012mea}
{\bf ``Search for anomalous production of multilepton events in $pp$ collisions at $\sqrt{s}=7$ TeV''}, 
  S.~Chatrchyan {\it et al.}  [CMS Collaboration], 
  %arXiv:1204.5341 [hep-ex]
    %{}10.1007/JHEP06(2012)169
JHEP {\bf 1206}169 (2012) %(Apr 2012)
%\href{http://inspirehep.net/record/1112169}{HEP entry}
%94 citations counted in INSPIRE as of 02 Jul 2015


%\cite{Chatrchyan:2012meb}
\item%{Chatrchyan:2012meb}
{\bf ``Search for leptonic decays of $W$ ' bosons in $pp$ collisions at $\sqrt{s}=7$ TeV''}, 
  S.~Chatrchyan {\it et al.}  [CMS Collaboration], 
  %arXiv:1204.4764 [hep-ex]
    %{}10.1007/JHEP08(2012)023
JHEP {\bf 1208}023 (2012) %(Apr 2012)
%\href{http://inspirehep.net/record/1111995}{HEP entry}
%72 citations counted in INSPIRE as of 02 Jul 2015


%\cite{Chatrchyan:2012qka}
\item%{Chatrchyan:2012qka}
{\bf ``Search for physics beyond the standard model in events with a $Z$ bosonjetsand missing transverse energy in $pp$ collisions at $\sqrt{s}=7$ TeV''}, 
  S.~Chatrchyan {\it et al.}  [CMS Collaboration], 
  %arXiv:1204.3774 [hep-ex]
    %{}10.1016/j.physletb.2012.08.026
Phys.\ Lett.\ B {\bf 716}260 (2012) %(Apr 2012)
%\href{http://inspirehep.net/record/1111141}{HEP entry}
%42 citations counted in INSPIRE as of 02 Jul 2015


%\cite{Chatrchyan:2012mec}
\item%{Chatrchyan:2012mec}
{\bf ``ShapeTransverse Sizeand Charged Hadron Multiplicity of Jets in pp Collisions at 7 TeV''}, 
  S.~Chatrchyan {\it et al.}  [CMS Collaboration], 
  %arXiv:1204.3170 [hep-ex]
    %{}10.1007/JHEP06(2012)160
JHEP {\bf 1206}160 (2012) %(Apr 2012)
%\href{http://inspirehep.net/record/1111014}{HEP entry}
%27 citations counted in INSPIRE as of 02 Jul 2015


%\cite{Chatrchyan:2012uba}
\item%{Chatrchyan:2012uba}
{\bf ``Measurement of the mass difference between top and antitop quarks''}, 
  S.~Chatrchyan {\it et al.}  [CMS Collaboration], 
  %arXiv:1204.2807 [hep-ex]
    %{}10.1007/JHEP06(2012)109
JHEP {\bf 1206}109 (2012) %(Apr 2012)
%\href{http://inspirehep.net/record/1110691}{HEP entry}
%28 citations counted in INSPIRE as of 02 Jul 2015


%\cite{Chatrchyan:2012ku}
\item%{Chatrchyan:2012ku}
{\bf ``Search for Anomalous $t\bar{t}$ Production in the Highly-Boosted All-Hadronic Final State''}, 
  S.~Chatrchyan {\it et al.}  [CMS Collaboration], 
  %arXiv:1204.2488 [hep-ex]
    %{}10.1007/JHEP09(2012)02910.1007/JHEP03(2014)132
JHEP {\bf 1209}029 (2012)[JHEP {\bf 1403}132 (2014)] %(Apr 2012)
%\href{http://inspirehep.net/record/1108144}{HEP entry}
%101 citations counted in INSPIRE as of 02 Jul 2015


%\cite{Chatrchyan:2012xq}
\item%{Chatrchyan:2012xq}
{\bf ``Azimuthal anisotropy of charged particles at high transverse momenta in PbPb collisions at $\sqrt{s_{NN}}=2.76$ TeV''}, 
  S.~Chatrchyan {\it et al.}  [CMS Collaboration], 
  %arXiv:1204.1850 [nucl-ex]
    %{}10.1103/PhysRevLett.109.022301
Phys.\ Rev.\ Lett.\  {\bf 109}022301 (2012) %(Apr 2012)
%\href{http://inspirehep.net/record/1107735}{HEP entry}
%80 citations counted in INSPIRE as of 02 Jul 2015


%\cite{Chatrchyan:2012vr}
\item%{Chatrchyan:2012vr}
{\bf ``Measurement of the Z/$\gamma$*+b-jet cross section in pp collisions at $\sqrt{s}$ = 7 TeV''}, 
  S.~Chatrchyan {\it et al.}  [CMS Collaboration], 
  %arXiv:1204.1643 [hep-ex]
    %{}10.1007/JHEP06(2012)126
JHEP {\bf 1206}126 (2012) %(Apr 2012)
%\href{http://inspirehep.net/record/1107730}{HEP entry}
%37 citations counted in INSPIRE as of 02 Jul 2015


%\cite{Chatrchyan:2012ta}
\item%{Chatrchyan:2012ta}
{\bf ``Measurement of the elliptic anisotropy of charged particles produced in PbPb collisions at $\sqrt{s}_{NN}$=2.76 TeV''}, 
  S.~Chatrchyan {\it et al.}  [CMS Collaboration], 
  %arXiv:1204.1409 [nucl-ex]
    %{}10.1103/PhysRevC.87.014902
Phys.\ Rev.\ C {\bf 87}no. 1014902 (2013) %(Apr 2012)
%\href{http://inspirehep.net/record/1107659}{HEP entry}
%125 citations counted in INSPIRE as of 02 Jul 2015


%\cite{Chatrchyan:2012tb}
\item%{Chatrchyan:2012tb}
{\bf ``Measurement of the underlying event in the Drell-Yan process in proton-proton collisions at $\sqrt{s}=7$ TeV''}, 
  S.~Chatrchyan {\it et al.}  [CMS Collaboration], 
  %arXiv:1204.1411 [hep-ex]
    %{}10.1140/epjc/s10052-012-2080-4
Eur.\ Phys.\ J.\ C {\bf 72}2080 (2012) %(Apr 2012)
%\href{http://inspirehep.net/record/1107658}{HEP entry}
%35 citations counted in INSPIRE as of 02 Jul 2015


%\cite{Chatrchyan:2012yea}
\item%{Chatrchyan:2012yea}
{\bf ``Search for heavy bottom-like quarks in 4.9 inverse femtobarns of $pp$ collisions at $\sqrt{s}=7$ TeV''}, 
  S.~Chatrchyan {\it et al.}  [CMS Collaboration], 
  %arXiv:1204.1088 [hep-ex]
    %{}10.1007/JHEP05(2012)123
JHEP {\bf 1205}123 (2012) %(Apr 2012)
%\href{http://inspirehep.net/record/1104744}{HEP entry}
%83 citations counted in INSPIRE as of 02 Jul 2015


%\cite{Chatrchyan:2012tea}
\item%{Chatrchyan:2012tea}
{\bf ``Search for Dark Matter and Large Extra Dimensions in pp Collisions Yielding a Photon and Missing Transverse Energy''}, 
  S.~Chatrchyan {\it et al.}  [CMS Collaboration], 
  %arXiv:1204.0821 [hep-ex]
    %{}10.1103/PhysRevLett.108.261803
Phys.\ Rev.\ Lett.\  {\bf 108}261803 (2012) %(Apr 2012)
%\href{http://inspirehep.net/record/1103032}{HEP entry}
%144 citations counted in INSPIRE as of 02 Jul 2015


%\cite{Chatrchyan:2012pb}
\item%{Chatrchyan:2012pb}
{\bf ``Ratios of dijet production cross sections as a function of the absolute difference in rapidity between jets in proton-proton collisions at $\sqrt{s}=7$ TeV''}, 
  S.~Chatrchyan {\it et al.}  [CMS Collaboration], 
  %arXiv:1204.0696 [hep-ex]
    %{}10.1140/epjc/s10052-012-2216-6
Eur.\ Phys.\ J.\ C {\bf 72}2216 (2012) %(Apr 2012)
%\href{http://inspirehep.net/record/1102908}{HEP entry}
%29 citations counted in INSPIRE as of 02 Jul 2015


%\cite{Chatrchyan:2012vs}
\item%{Chatrchyan:2012vs}
{\bf ``Measurement of the top quark pair production cross section in $pp$ collisions at $\sqrt{s} = 7$ TeV in dilepton final states containing a $\tau$''}, 
  S.~Chatrchyan {\it et al.}  [CMS Collaboration], 
  %arXiv:1203.6810 [hep-ex]
    %{}10.1103/PhysRevD.85.112007
Phys.\ Rev.\ D {\bf 85}112007 (2012) %(Mar 2012)
%\href{http://inspirehep.net/record/1095503}{HEP entry}
%55 citations counted in INSPIRE as of 02 Jul 2015


%\cite{CMS:2012ab}
\item%{CMS:2012ab}
{\bf ``Search for heavytop-like quark pair production in the dilepton final state in $pp$ collisions at $\sqrt{s} = 7$ TeV''}, 
  S.~Chatrchyan {\it et al.}  [CMS Collaboration], 
  %arXiv:1203.5410 [hep-ex]
    %{}10.1016/j.physletb.2012.07.059
Phys.\ Lett.\ B {\bf 716}103 (2012) %(Mar 2012)
%\href{http://inspirehep.net/record/1094855}{HEP entry}
%85 citations counted in INSPIRE as of 02 Jul 2015


%\cite{Chatrchyan:2012rga}
\item%{Chatrchyan:2012rga}
{\bf ``Search for $B^0_s \to \mu^+ \mu^-$ and $B^0 \to \mu^+ \mu^-$ decays''}, 
  S.~Chatrchyan {\it et al.}  [CMS Collaboration], 
  %arXiv:1203.3976 [hep-ex]
    %{}10.1007/JHEP04(2012)033
JHEP {\bf 1204}033 (2012) %(Mar 2012)
%\href{http://inspirehep.net/record/1094164}{HEP entry}
%117 citations counted in INSPIRE as of 02 Jul 2015


%\cite{Chatrchyan:2012hw}
\item%{Chatrchyan:2012hw}
{\bf ``Measurement of the cross section for production of $b b^-$ bar $X$decaying to muons in $pp$ collisions at $\sqrt{s}=7$ TeV''}, 
  S.~Chatrchyan {\it et al.}  [CMS Collaboration], 
  %arXiv:1203.3458 [hep-ex]
    %{}10.1007/JHEP06(2012)110
JHEP {\bf 1206}110 (2012) %(Mar 2012)
%\href{http://inspirehep.net/record/1093951}{HEP entry}
%25 citations counted in INSPIRE as of 02 Jul 2015


%\cite{Chatrchyan:2012taa}
\item%{Chatrchyan:2012taa}
{\bf ``Search for microscopic black holes in $pp$ collisions at $\sqrt{s}=7$ TeV''}, 
  S.~Chatrchyan {\it et al.}  [CMS Collaboration], 
  %arXiv:1202.6396 [hep-ex]
    %{}10.1007/JHEP04(2012)061
JHEP {\bf 1204}061 (2012) %(Feb 2012)
%\href{http://inspirehep.net/record/1091050}{HEP entry}
%53 citations counted in INSPIRE as of 02 Jul 2015


%\cite{Chatrchyan:2012bf}
\item%{Chatrchyan:2012bf}
{\bf ``Search for quark compositeness in dijet angular distributions from $pp$ collisions at $\sqrt{s}=7$ TeV''}, 
  S.~Chatrchyan {\it et al.}  [CMS Collaboration], 
  %arXiv:1202.5535 [hep-ex]
    %{}10.1007/JHEP05(2012)055
JHEP {\bf 1205}055 (2012) %(Feb 2012)
%\href{http://inspirehep.net/record/1090423}{HEP entry}
%51 citations counted in INSPIRE as of 02 Jul 2015


%\cite{Chatrchyan:2012nia}
\item%{Chatrchyan:2012nia}
{\bf ``Jet momentum dependence of jet quenching in PbPb collisions at $\sqrt{s_{NN}}=2.76$ TeV''}, 
  S.~Chatrchyan {\it et al.}  [CMS Collaboration], 
  %arXiv:1202.5022 [nucl-ex]
    %{}10.1016/j.physletb.2012.04.058
Phys.\ Lett.\ B {\bf 712}176 (2012) %(Feb 2012)
%\href{http://inspirehep.net/record/1090064}{HEP entry}
%118 citations counted in INSPIRE as of 02 Jul 2015


%\cite{Chatrchyan:2012dk}
\item%{Chatrchyan:2012dk}
{\bf ``Inclusive $b$-jet production in $pp$ collisions at $\sqrt{s}=7$ TeV''}, 
  S.~Chatrchyan {\it et al.}  [CMS Collaboration], 
  %arXiv:1202.4617 [hep-ex]
    %{}10.1007/JHEP04(2012)084
JHEP {\bf 1204}084 (2012) %(Feb 2012)
%\href{http://inspirehep.net/record/1089835}{HEP entry}
%45 citations counted in INSPIRE as of 02 Jul 2015


%\cite{Chatrchyan:2012ww}
\item%{Chatrchyan:2012ww}
{\bf ``Search for the standard model Higgs boson decaying to bottom quarks in $pp$ collisions at $\sqrt{s}=7$ TeV''}, 
  S.~Chatrchyan {\it et al.}  [CMS Collaboration], 
  %arXiv:1202.4195 [hep-ex]
    %{}10.1016/j.physletb.2012.02.085
Phys.\ Lett.\ B {\bf 710}284 (2012) %(Feb 2012)
%\href{http://inspirehep.net/record/1089700}{HEP entry}
%91 citations counted in INSPIRE as of 02 Jul 2015


%\cite{Chatrchyan:2012vp}
\item%{Chatrchyan:2012vp}
{\bf ``Search for neutral Higgs bosons decaying to $\tau$ pairs in $pp$ collisions at $\sqrt{s}=7$ TeV''}, 
  S.~Chatrchyan {\it et al.}  [CMS Collaboration], 
  %arXiv:1202.4083 [hep-ex]
    %{}10.1016/j.physletb.2012.05.028
Phys.\ Lett.\ B {\bf 713}68 (2012) %(Feb 2012)
%\href{http://inspirehep.net/record/1089661}{HEP entry}
%197 citations counted in INSPIRE as of 02 Jul 2015


%\cite{Chatrchyan:2012kc}
\item%{Chatrchyan:2012kc}
{\bf ``Search for large extra dimensions in dimuon and dielectron events in pp collisions at $\sqrt{s} = 7$ TeV''}, 
  S.~Chatrchyan {\it et al.}  [CMS Collaboration], 
  %arXiv:1202.3827 [hep-ex]
    %{}10.1016/j.physletb.2012.03.029
Phys.\ Lett.\ B {\bf 711}15 (2012) %(Feb 2012)
%\href{http://inspirehep.net/record/1089399}{HEP entry}
%34 citations counted in INSPIRE as of 02 Jul 2015


%\cite{Chatrchyan:2012ft}
\item%{Chatrchyan:2012ft}
{\bf ``Search for the standard model Higgs boson in the $H$ to $Z Z$ to 2 $\ell 2 \nu$ channel in $pp$ collisions at $\sqrt{s}=7$ TeV''}, 
  S.~Chatrchyan {\it et al.}  [CMS Collaboration], 
  %arXiv:1202.3478 [hep-ex]
    %{}10.1007/JHEP03(2012)040
JHEP {\bf 1203}040 (2012) %(Feb 2012)
%\href{http://inspirehep.net/record/1089334}{HEP entry}
%46 citations counted in INSPIRE as of 02 Jul 2015


%\cite{Chatrchyan:2012hr}
\item%{Chatrchyan:2012hr}
{\bf ``Search for the standard model Higgs boson in the $H$ to $Z Z$ to $\ell \ell \tau \tau$ decay channel in $pp$ collisions at $\sqrt{s}=7$ TeV''}, 
  S.~Chatrchyan {\it et al.}  [CMS Collaboration], 
  %arXiv:1202.3617 [hep-ex]
    %{}10.1007/JHEP03(2012)081
JHEP {\bf 1203}081 (2012) %(Feb 2012)
%\href{http://inspirehep.net/record/1089288}{HEP entry}
%28 citations counted in INSPIRE as of 02 Jul 2015


%\cite{CMS:2012aa}
\item%{CMS:2012aa}
{\bf ``Study of high-pT charged particle suppression in PbPb compared to $pp$ collisions at $\sqrt{s_{NN}}=2.76$ TeV''}, 
  S.~Chatrchyan {\it et al.}  [CMS Collaboration], 
  %arXiv:1202.2554 [nucl-ex]
    %{}10.1140/epjc/s10052-012-1945-x
Eur.\ Phys.\ J.\ C {\bf 72}1945 (2012) %(Feb 2012)
%\href{http://inspirehep.net/record/1088823}{HEP entry}
%216 citations counted in INSPIRE as of 02 Jul 2015


%\cite{Chatrchyan:2012dg}
\item%{Chatrchyan:2012dg}
{\bf ``Search for the standard model Higgs boson in the decay channel $H$ to $Z Z$ to 4 leptons in $pp$ collisions at $\sqrt{s}=7$ TeV''}, 
  S.~Chatrchyan {\it et al.}  [CMS Collaboration], 
  %arXiv:1202.1997 [hep-ex]
    %{}10.1103/PhysRevLett.108.111804
Phys.\ Rev.\ Lett.\  {\bf 108}111804 (2012) %(Feb 2012)
%\href{http://inspirehep.net/record/1088604}{HEP entry}
%130 citations counted in INSPIRE as of 02 Jul 2015


%\cite{Chatrchyan:2012ty}
\item%{Chatrchyan:2012ty}
{\bf ``Search for the standard model Higgs boson decaying to a $W$ pair in the fully leptonic final state in $pp$ collisions at $\sqrt{s}=7$ TeV''}, 
  S.~Chatrchyan {\it et al.}  [CMS Collaboration], 
  %arXiv:1202.1489 [hep-ex]
    %{}10.1016/j.physletb.2012.02.076
Phys.\ Lett.\ B {\bf 710}91 (2012) %(Feb 2012)
%\href{http://inspirehep.net/record/1088232}{HEP entry}
%144 citations counted in INSPIRE as of 02 Jul 2015


%\cite{Chatrchyan:2012tx}
\item%{Chatrchyan:2012tx}
{\bf ``Combined results of searches for the standard model Higgs boson in $pp$ collisions at $\sqrt{s}=7$ TeV''}, 
  S.~Chatrchyan {\it et al.}  [CMS Collaboration], 
  %arXiv:1202.1488 [hep-ex]
    %{}10.1016/j.physletb.2012.02.064
Phys.\ Lett.\ B {\bf 710}26 (2012) %(Feb 2012)
%\href{http://inspirehep.net/record/1088231}{HEP entry}
%689 citations counted in INSPIRE as of 02 Jul 2015


%\cite{Chatrchyan:2012twa}
\item%{Chatrchyan:2012twa}
{\bf ``Search for the standard model Higgs boson decaying into two photons in $pp$ collisions at $\sqrt{s}=7$ TeV''}, 
  S.~Chatrchyan {\it et al.}  [CMS Collaboration], 
  %arXiv:1202.1487 [hep-ex]
    %{}10.1016/j.physletb.2012.03.003
Phys.\ Lett.\ B {\bf 710}403 (2012) %(Feb 2012)
%\href{http://inspirehep.net/record/1088230}{HEP entry}
%209 citations counted in INSPIRE as of 02 Jul 2015


%\cite{Chatrchyan:2012sn}
\item%{Chatrchyan:2012sn}
{\bf ``Search for a Higgs boson in the decay channel $H$ to ZZ(*) to $q$ qbar $\ell^-$ l+ in $pp$ collisions at $\sqrt{s}=7$ TeV''}, 
  S.~Chatrchyan {\it et al.}  [CMS Collaboration], 
  %arXiv:1202.1416 [hep-ex]
    %{}10.1007/JHEP04(2012)036
JHEP {\bf 1204}036 (2012) %(Feb 2012)
%\href{http://inspirehep.net/record/1088226}{HEP entry}
%89 citations counted in INSPIRE as of 02 Jul 2015


%\cite{Chatrchyan:2012gwa}
\item%{Chatrchyan:2012gwa}
{\bf ``Measurement of the inclusive production cross sections for forward jets and for dijet events with one forward and one central jet in $pp$ collisions at $\sqrt{s}=7$ TeV''}, 
  S.~Chatrchyan {\it et al.}  [CMS Collaboration], 
  %arXiv:1202.0704 [hep-ex]
    %{}10.1007/JHEP06(2012)036
JHEP {\bf 1206}036 (2012) %(Feb 2012)
%\href{http://inspirehep.net/record/1087342}{HEP entry}
%46 citations counted in INSPIRE as of 02 Jul 2015


%\cite{Chatrchyan:2012np}
\item%{Chatrchyan:2012np}
{\bf ``Suppression of non-prompt $J/\psi$prompt $J/\psi$and Y(1S) in PbPb collisions at $\sqrt{s_{NN}}=2.76$ TeV''}, 
  S.~Chatrchyan {\it et al.}  [CMS Collaboration], 
  %arXiv:1201.5069 [nucl-ex]
    %{}10.1007/JHEP05(2012)063
JHEP {\bf 1205}063 (2012) %(Jan 2012)
%\href{http://inspirehep.net/record/1085651}{HEP entry}
%217 citations counted in INSPIRE as of 02 Jul 2015


%\cite{Chatrchyan:2012wg}
\item%{Chatrchyan:2012wg}
{\bf ``Centrality dependence of dihadron correlations and azimuthal anisotropy harmonics in PbPb collisions at $\sqrt{s_{NN}}=2.76$ TeV''}, 
  S.~Chatrchyan {\it et al.}  [CMS Collaboration], 
  %arXiv:1201.3158 [nucl-ex]
    %{}10.1140/epjc/s10052-012-2012-3
Eur.\ Phys.\ J.\ C {\bf 72}2012 (2012) %(Jan 2012)
%\href{http://inspirehep.net/record/1084730}{HEP entry}
%129 citations counted in INSPIRE as of 02 Jul 2015


%\cite{Chatrchyan:2012vq}
\item%{Chatrchyan:2012vq}
{\bf ``Measurement of isolated photon production in $pp$ and PbPb collisions at $\sqrt{s_{NN}}=2.76$ TeV''}, 
  S.~Chatrchyan {\it et al.}  [CMS Collaboration], 
  %arXiv:1201.3093 [nucl-ex]
    %{}10.1016/j.physletb.2012.02.077
Phys.\ Lett.\ B {\bf 710}256 (2012) %(Jan 2012)
%\href{http://inspirehep.net/record/1084729}{HEP entry}
%82 citations counted in INSPIRE as of 02 Jul 2015


%\cite{Chatrchyan:2011hk}
\item%{Chatrchyan:2011hk}
{\bf ``Measurement of the charge asymmetry in top-quark pair production in proton-proton collisions at $\sqrt{s}=7$ TeV''}, 
  S.~Chatrchyan {\it et al.}  [CMS Collaboration], 
  %arXiv:1112.5100 [hep-ex]
    %{}10.1016/j.physletb.2012.01.078
Phys.\ Lett.\ B {\bf 709}28 (2012) %(Dec 2011)
%\href{http://inspirehep.net/record/1082456}{HEP entry}
%84 citations counted in INSPIRE as of 02 Jul 2015


%\cite{Chatrchyan:2011fq}
\item%{Chatrchyan:2011fq}
{\bf ``Search for signatures of extra dimensions in the diphoton mass spectrum at the Large Hadron Collider''}, 
  S.~Chatrchyan {\it et al.}  [CMS Collaboration], 
  %arXiv:1112.0688 [hep-ex]
    %{}10.1103/PhysRevLett.108.111801
Phys.\ Rev.\ Lett.\  {\bf 108}111801 (2012) %(Dec 2011)
%\href{http://inspirehep.net/record/1079910}{HEP entry}
%87 citations counted in INSPIRE as of 02 Jul 2015


%\cite{Chatrchyan:2011ci}
\item%{Chatrchyan:2011ci}
{\bf ``Exclusive photon-photon production of muon pairs in proton-proton collisions at $\sqrt{s}=7$ TeV''}, 
  S.~Chatrchyan {\it et al.}  [CMS Collaboration], 
  %arXiv:1111.5536 [hep-ex]
    %{}10.1007/JHEP01(2012)052
JHEP {\bf 1201}052 (2012) %(Nov 2011)
%\href{http://inspirehep.net/record/954992}{HEP entry}
%41 citations counted in INSPIRE as of 02 Jul 2015


%\cite{Chatrchyan:2011kc}
\item%{Chatrchyan:2011kc}
{\bf ``$J/\psi$ and $\psi_{2S}$ production in $pp$ collisions at $\sqrt{s}=7$ TeV''}, 
  S.~Chatrchyan {\it et al.}  [CMS Collaboration], 
  %arXiv:1111.1557 [hep-ex]
    %{}10.1007/JHEP02(2012)011
JHEP {\bf 1202}011 (2012) %(Nov 2011)
%\href{http://inspirehep.net/record/944755}{HEP entry}
%98 citations counted in INSPIRE as of 02 Jul 2015


%\cite{Chatrchyan:2011qt}
\item%{Chatrchyan:2011qt}
{\bf ``Measurement of the Production Cross Section for Pairs of Isolated Photons in $pp$ collisions at $\sqrt{s}=7$ TeV''}, 
  S.~Chatrchyan {\it et al.}  [CMS Collaboration], 
  %arXiv:1110.6461 [hep-ex]
    %{}10.1007/JHEP01(2012)133
JHEP {\bf 1201}133 (2012) %(Oct 2011)
%\href{http://inspirehep.net/record/943720}{HEP entry}
%38 citations counted in INSPIRE as of 02 Jul 2015


%\cite{Chatrchyan:2011wt}
\item%{Chatrchyan:2011wt}
{\bf ``Measurement of the Rapidity and Transverse Momentum Distributions of $Z$ Bosons in $pp$ Collisions at $\sqrt{s}=7$ TeV''}, 
  S.~Chatrchyan {\it et al.}  [CMS Collaboration], 
  %arXiv:1110.4973 [hep-ex]
    %{}10.1103/PhysRevD.85.032002
Phys.\ Rev.\ D {\bf 85}032002 (2012) %(Oct 2011)
%\href{http://inspirehep.net/record/941555}{HEP entry}
%94 citations counted in INSPIRE as of 02 Jul 2015


%\cite{Chatrchyan:2011ne}
\item%{Chatrchyan:2011ne}
{\bf ``Jet Production Rates in Association with $W$ and $Z$ Bosons in $pp$ Collisions at $\sqrt{s}=7$ TeV''}, 
  S.~Chatrchyan {\it et al.}  [CMS Collaboration], 
  %arXiv:1110.3226 [hep-ex]
    %{}10.1007/JHEP01(2012)010
JHEP {\bf 1201}010 (2012) %(Oct 2011)
%\href{http://inspirehep.net/record/940012}{HEP entry}
%78 citations counted in INSPIRE as of 02 Jul 2015


%\cite{Chatrchyan:2011ya}
\item%{Chatrchyan:2011ya}
{\bf ``Measurement of the weak mixing angle with the Drell-Yan process in proton-proton collisions at the LHC''}, 
  S.~Chatrchyan {\it et al.}  [CMS Collaboration], 
  %arXiv:1110.2682 [hep-ex]
    %{}10.1103/PhysRevD.84.112002
Phys.\ Rev.\ D {\bf 84}112002 (2011) %(Oct 2011)
%\href{http://inspirehep.net/record/939559}{HEP entry}
%36 citations counted in INSPIRE as of 02 Jul 2015


%\cite{Chatrchyan:2011wb}
\item%{Chatrchyan:2011wb}
{\bf ``Forward Energy FlowCentral Charged-Particle Multiplicitiesand Pseudorapidity Gaps in W and Z Boson Events from pp Collisions at $\sqrt{s}$ = 7 TeV''}, 
  S.~Chatrchyan {\it et al.}  [CMS Collaboration], 
  %arXiv:1110.0181 [hep-ex]
    %{}10.1140/epjc/s10052-011-1839-3
Eur.\ Phys.\ J.\ C {\bf 72}1839 (2012) %(Oct 2011)
%\href{http://inspirehep.net/record/930318}{HEP entry}
%15 citations counted in INSPIRE as of 02 Jul 2015


%\cite{Chatrchyan:2012zz}
\item%{Chatrchyan:2012zz}
{\bf ``Performance of tau-lepton reconstruction and identification in CMS''}, 
  S.~Chatrchyan {\it et al.}  [CMS Collaboration], 
  %arXiv:1109.6034 [physics.ins-det]
    %{}10.1088/1748-0221/7/01/P01001
JINST {\bf 7}P01001 (2012) %(Sep 2011)
%\href{http://inspirehep.net/record/929904}{HEP entry}
%141 citations counted in INSPIRE as of 02 Jul 2015


%\cite{Chatrchyan:2011ay}
\item%{Chatrchyan:2011ay}
{\bf ``Search for a Vector-like Quark with Charge 2/3 in $t$ + $Z$ Events from $pp$ Collisions at $\sqrt{s}=7$ TeV''}, 
  S.~Chatrchyan {\it et al.}  [CMS Collaboration], 
  %arXiv:1109.4985 [hep-ex]
    %{}10.1103/PhysRevLett.107.271802
Phys.\ Rev.\ Lett.\  {\bf 107}271802 (2011) %(Sep 2011)
%\href{http://inspirehep.net/record/928282}{HEP entry}
%67 citations counted in INSPIRE as of 02 Jul 2015


%\cite{Chatrchyan:2011zy}
\item%{Chatrchyan:2011zy}
{\bf ``Search for Supersymmetry at the LHC in Events with Jets and Missing Transverse Energy''}, 
  S.~Chatrchyan {\it et al.}  [CMS Collaboration], 
  %arXiv:1109.2352 [hep-ex]
    %{}10.1103/PhysRevLett.107.221804
Phys.\ Rev.\ Lett.\  {\bf 107}221804 (2011) %(Sep 2011)
%\href{http://inspirehep.net/record/927049}{HEP entry}
%283 citations counted in INSPIRE as of 02 Jul 2015


%\cite{Chatrchyan:2011ue}
\item%{Chatrchyan:2011ue}
{\bf ``Measurement of the Differential Cross Section for Isolated Prompt Photon Production in pp Collisions at 7 TeV''}, 
  S.~Chatrchyan {\it et al.}  [CMS Collaboration], 
  %arXiv:1108.2044 [hep-ex]
    %{}10.1103/PhysRevD.84.052011
Phys.\ Rev.\ D {\bf 84}052011 (2011) %(Aug 2011)
%\href{http://inspirehep.net/record/922830}{HEP entry}
%59 citations counted in INSPIRE as of 02 Jul 2015


%\cite{Chatrchyan:2011cm}
\item%{Chatrchyan:2011cm}
{\bf ``Measurement of the Drell-Yan Cross Section in $pp$ Collisions at $\sqrt{s}=7$ TeV''}, 
  S.~Chatrchyan {\it et al.}  [CMS Collaboration], 
  %arXiv:1108.0566 [hep-ex]
    %{}10.1007/JHEP10(2011)007
JHEP {\bf 1110}007 (2011) %(Aug 2011)
%\href{http://inspirehep.net/record/921788}{HEP entry}
%55 citations counted in INSPIRE as of 02 Jul 2015


%\cite{Chatrchyan:2011kr}
\item%{Chatrchyan:2011kr}
{\bf ``Search for B(s) and B to dimuon decays in pp collisions at 7 TeV''}, 
  S.~Chatrchyan {\it et al.}  [CMS Collaboration], 
  %arXiv:1107.5834 [hep-ex]
    %{}10.1103/PhysRevLett.107.191802
Phys.\ Rev.\ Lett.\  {\bf 107}191802 (2011) %(Jul 2011)
%\href{http://inspirehep.net/record/921468}{HEP entry}
%80 citations counted in INSPIRE as of 02 Jul 2015


%\cite{Chatrchyan:2011ns}
\item%{Chatrchyan:2011ns}
{\bf ``Search for Resonances in the Dijet Mass Spectrum from 7 TeV pp Collisions at CMS''}, 
  S.~Chatrchyan {\it et al.}  [CMS Collaboration], 
  %arXiv:1107.4771 [hep-ex]
    %{}10.1016/j.physletb.2011.09.015
Phys.\ Lett.\ B {\bf 704}123 (2011) %(Jul 2011)
%\href{http://inspirehep.net/record/919742}{HEP entry}
%148 citations counted in INSPIRE as of 02 Jul 2015


%\cite{Chatrchyan:2011pb}
\item%{Chatrchyan:2011pb}
{\bf ``Dependence on pseudorapidity and centrality of charged hadron production in PbPb collisions at a nucleon-nucleon centre-of-mass energy of 2.76 TeV''}, 
  S.~Chatrchyan {\it et al.}  [CMS Collaboration], 
  %arXiv:1107.4800 [nucl-ex]
    %{}10.1007/JHEP08(2011)141
JHEP {\bf 1108}141 (2011) %(Jul 2011)
%\href{http://inspirehep.net/record/919733}{HEP entry}
%95 citations counted in INSPIRE as of 02 Jul 2015


%\cite{Chatrchyan:2011ds}
\item%{Chatrchyan:2011ds}
{\bf ``Determination of Jet Energy Calibration and Transverse Momentum Resolution in CMS''}, 
  S.~Chatrchyan {\it et al.}  [CMS Collaboration], 
  %arXiv:1107.4277 [physics.ins-det]
    %{}10.1088/1748-0221/6/11/P11002
JINST {\bf 6}P11002 (2011) %(Jul 2011)
%\href{http://inspirehep.net/record/919443}{HEP entry}
%532 citations counted in INSPIRE as of 02 Jul 2015
\end{enumerate}





%\end{document}