% Chapter Template

\chapter{Conclusions} % Main chapter title

\label{Chapter8} % Change X to a consecutive number; for referencing this chapter elsewhere, use \ref{ChapterX}

\lhead{Chapter 8. \emph{Conclusions}} 

% sto je prezentirano i zasto
This thesis presents the inclusive cross section measurement for W boson production in association with two b jets in proton-proton collisions at $\sqrt{s}=$8 TeV. The analyzed data have been collected with the CMS detector during 2012, corresponding to an integrated luminosity of 19.8 fb$^{-1}$. From the theoretical point of view, this process is important as a probe into the perturbative QCD calculations and can point to the need of improvement of theoretical calculations. On the other hand, from the experimental  point of view, accurate  knowledge of Wbb process leads to more precise measurements of other processes to which this one is a background, such as the top quark or associated Higgs and W boson production, with Higgs boson  decaying to a pair of b quarks.

% koji su rezultati
\par The cross section measurement was performed in the fiducial region defined by a presence of a lepton and exactly two 2 b-tagged jets. The result is quoted separately in the muon and electron channel. The measured values in the two channels are compatible, as predicted by the standard model. The theoretical cross section was derived using MCFM . The measured values are around one standard deviation higher than the predicted theoretical values. Several measurements including a W boson and b jets in the final state were performed in the past. However the results cannot be directly compared because of different phase space and slightly different final states.

% kako popraviti rezultate
The uncertainty on the measured values is dominated by the systematic effects. The largest uncertainties are associated with the b tagging procedure, jet energy scale and jet energy resolution. Reducing these uncertainties in the future, would allow for more sensitive test of perturbative QCD calculation at next-to-leading order.

% sto ce se dogadjati u buducnosti
After the long shutdown, which started in 2013, the LHC has just restarted its operations, reaching the highest energy of $\sqrt{s}=$ 13 TeV. This marks the start of another exciting period, which is aimed at precision measurements of Higgs boson as well as various beyonf standard model searches like supersymmetry which may even answer the question of dark matter. As for the measurements of W boson produced in association with b quarks, one of the main goals for the future is to reduce the large systematic uncertainties as stated previously. Measuring differential cross section as a function of various variables, like leading jet transverse momentum, would be of great interest to theorists. Another goal would be to probe different final states, for example studying events with two B hadrons in the same jet, or using track based tagging without requiring jets. This would allow us to study collinear final state in depth.

