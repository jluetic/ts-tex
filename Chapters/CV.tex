% LaTeX Curriculum Vitae Template
%
% Copyright (C) 2004-2009 Jason Blevins <jrblevin@sdf.lonestar.org>
% http://jblevins.org/projects/cv-template/
%
% You may use use this document as a template to create your own CV
% and you may redistribute the source code freely. No attribution is
% required in any resulting documents. I do ask that you please leave
% this notice and the above URL in the source code if you choose to
% redistribute this file.

\documentclass[letterpaper,12pt]{article}

\usepackage{hyperref}
\usepackage{geometry}


% Comment the following lines to use the default Computer Modern font
% instead of the Palatino font provided by the mathpazo package.
% Remove the 'osf' bit if you don't like the old style figures.
\usepackage[T1]{fontenc}
\usepackage[utf8]{inputenc}
\usepackage[sc]{mathpazo}
\usepackage[croatian]{babel}
% Set your name here
\def\name{Jelena Luetić}

% Replace this with a link to your CV if you like, or set it empty
% (as in \def\footerlink{}) to remove the link in the footer:
\def\footerlink{}

% The following metadata will show up in the PDF properties
\hypersetup{
  colorlinks = true,
  urlcolor = black,
  pdfauthor = {\name},
  pdfkeywords = {physics, statistics, mathematics},
  pdftitle = {\name: Curriculum Vitae},
  pdfsubject = {Curriculum Vitae},
  pdfpagemode = UseNone
}

\geometry{
  body={6.5in, 8.5in},
  left=1.0in,
  top=1.25in
}

% Customize page headers
\pagestyle{myheadings}
\markright{\name}
\thispagestyle{empty}

% Custom section fonts
\usepackage{sectsty}
\sectionfont{\rmfamily\mdseries\Large}
\subsectionfont{\rmfamily\mdseries\itshape\large}

% Other possible font commands include:
% \ttfamily for teletype,
% \sffamily for sans serif,
% \bfseries for bold,
% \scshape for small caps,
% \normalsize, \large, \Large, \LARGE sizes.

% Don't indent paragraphs.
\setlength\parindent{0em}

% Make lists without bullets
%\renewenvironment{itemize}{
%  \begin{list}{}{
%    \setlength{\leftmargin}{1.5em}
%  }
%}{
%  \end{list}
%}

\begin{document}

% Place name at left
{\huge \name}

% Alternatively, print name centered and bold:
%\centerline{\huge \bf \name}

\vspace{0.25in}

%\begin{minipage}{0.45\linewidth}
%  \href{http://www.irb.hr/}{Ruđer Bošković Institute} \\
%  Department of experimental physics \\
%  Bijenička 54, 10000 Zagreb
%\end{minipage}

\section*{Personal}


  \begin{tabular}{r|l}
	Address & Supilova 7, 10000 Zagreb, Croatia \\[5pt]   
    Date of Birth & April 5, 1987 \\[5pt]
    Citizenship & Croatian \\[5pt]    
    Phone & +385 91 1480103 \\[5pt]
    Email & \href{mailto:jelena.luetic@cern.ch}{\tt jelena.luetic@cern.ch} \\[5pt]
    Languages & English, basic Italian and French, Croatian - native \\
  \end{tabular}



\section*{Education}

\begin{table}[h!]
 \centering
\begin{tabular}{r | l}
2011 - & \textbf{PhD}, Faculty of Science, Physics department \\ & \textit{Title}: 
Measurement of the cross section for associated production of a W boson \\ & and two b quarks with the CMS detector at the Large Hadron Collider \\ &  \textit{Advisor:} Prof. Vuko Brigljević, thesis defence is scheduled for July, 15th 2015. \\[5pt] 
2005 - 2010 & \textbf{MSc}, Faculty of Science, Physics department \\  & \textit{Title}: Measurement of Z boson cross section in proton-proton collisions \\ & with CMS detector at Large Hadron Collider \\ &  \textit{Advisor:} Prof. Ivica Puljak, thesis defended on November, 30th 2010. \\ 
\end{tabular}
\end{table}


\section*{Professional experience}

\textbf{CMS Experiment} - Since 2010 I've been a part of the CMS group at Ruđer Bošković Institute, working on gauge boson produced in association with jets measurements and on technical aspects of CMS detector:
\begin{itemize}
\item One of the main contributors to Wbb cross section measurement at $\sqrt{s} =$ 8 TeV. Collaboration with University of Wisconsin-Madison, University of Trieste and CERN.
\item Responsible for Lorentz angle monitoring and calibration for CMS Pixel detector.
\item Responsible for development and maintenance of technical tools used in the offline analysis shared with other members of CMS Pixel group.
\item Participated in pixel detector recommissioning during the long shutdown. 
\item Paricipated in CMS Pixel Operations
\end{itemize}

\textbf{Particle Detectors} During 2012 and 2013 I participated in several 
\begin{itemize}
\item bla
\end{itemize}

\textbf{ACE - Antimatter Cell Experiment} -  during the CERN Summer student programme, I was a part of ACE experiment, which explores the potential of using antiprotons for cancer therapies. This is an interdisciplinary project, which brings together experts from biology, physics and medicine from more than 10 countries. My contribution to the experiment was:
\begin{itemize}
\item Development of the interface for the remote control of the power supplies for the detectors used in the experiment. This was done in LabView 
\item Development of the online data analysis software
\end{itemize}

\section*{Teaching}

\begin{tabular}{r | l}
  2011 - 2015 & Faculty of Science, Zagreb, Croatia \\ & Teaching assistant in 2 courses - \textit{Introductory laboratory exercises} (2011.- 2012.) \\ & and \textit{Programming in C} (2011.-2015.) \\[5pt]
  2010 - 2011 & Faculty of Electrical Engineering, Mechanical Engineering and Naval \\ & Architecture in Split, Croatia \\ & Teaching assistant - \textit{Laboratory exercises in Modern physics} 
\end{tabular}

\section*{Schools and Conferences}

\begin{tabular}{r | l}
  2009 & CERN Summer Student Programme, Geneva, Switzerland \\[5pt]
  2010 & CERN School of Computing, London, UK \\[5pt]
  2011 & CMSDAS - Data analysis school, Pisa, Italy \\[5pt]
  2012 & Silicon Detector Workshop, Split, Croatia \\
  	   & \textit{Talk:} Lorentz angle measurement in CMS Pixel detector\\[5pt] 
  2013 & EDIT - Excellence in Detectors, Tsukuba, Japan \\
  	   & \textit{Poster}: CMS Pixel detector and Lorentz angle determination \\[5pt]
  2014 & Fermilab - CERN Hadron Collider School, Fermilab, USA\\
\end{tabular}

\section*{Computer skills}
\begin{itemize}
\item Computer languages: C, C++, Fortran, Python 
\item Software: ROOT, Mathematica, LabView
\end{itemize}

\section*{Other}

Participation in various physics and science outreach programs:
\begin{itemize}
	\item Organization of International Masterclasses: lectures and laboratory exercises for high school students covering various topics in high energy physics.
	\item Participated in several science fairs for general public demonstrating experiments. 
\end{itemize}


\section*{Publications}




\end{document}