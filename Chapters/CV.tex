% LaTeX Curriculum Vitae Template
%
% Copyright (C) 2004-2009 Jason Blevins <jrblevin@sdf.lonestar.org>
% http://jblevins.org/projects/cv-template/
%
% You may use use this document as a template to create your own CV
% and you may redistribute the source code freely. No attribution is
% required in any resulting documents. I do ask that you please leave
% this notice and the above URL in the source code if you choose to
% redistribute this file.

\documentclass[letterpaper,12pt]{article}

\usepackage{hyperref}
\usepackage{geometry}


% Comment the following lines to use the default Computer Modern font
% instead of the Palatino font provided by the mathpazo package.
% Remove the 'osf' bit if you don't like the old style figures.
\usepackage[T1]{fontenc}
\usepackage[utf8]{inputenc}
\usepackage[sc,osf]{mathpazo}
\usepackage[croatian]{babel}
% Set your name here
\def\name{Jelena Luetić}

% Replace this with a link to your CV if you like, or set it empty
% (as in \def\footerlink{}) to remove the link in the footer:
\def\footerlink{}

% The following metadata will show up in the PDF properties
\hypersetup{
  colorlinks = true,
  urlcolor = black,
  pdfauthor = {\name},
  pdfkeywords = {physics, statistics, mathematics},
  pdftitle = {\name: Curriculum Vitae},
  pdfsubject = {Curriculum Vitae},
  pdfpagemode = UseNone
}

\geometry{
  body={6.5in, 8.5in},
  left=1.0in,
  top=1.25in
}

% Customize page headers
\pagestyle{myheadings}
\markright{\name}
\thispagestyle{empty}

% Custom section fonts
\usepackage{sectsty}
\sectionfont{\rmfamily\mdseries\Large}
\subsectionfont{\rmfamily\mdseries\itshape\large}

% Other possible font commands include:
% \ttfamily for teletype,
% \sffamily for sans serif,
% \bfseries for bold,
% \scshape for small caps,
% \normalsize, \large, \Large, \LARGE sizes.

% Don't indent paragraphs.
\setlength\parindent{0em}

% Make lists without bullets
\renewenvironment{itemize}{
  \begin{list}{}{
    \setlength{\leftmargin}{1.5em}
  }
}{
  \end{list}
}

\begin{document}

% Place name at left
{\huge \name}

% Alternatively, print name centered and bold:
%\centerline{\huge \bf \name}

\vspace{0.25in}

%\begin{minipage}{0.45\linewidth}
%  \href{http://www.irb.hr/}{Ruđer Bošković Institute} \\
%  Department of experimental physics \\
%  Bijenička 54, 10000 Zagreb
%\end{minipage}

\section*{Personal}


  \begin{tabular}{r|l}
	Address & Supilova 7, 10000 Zagreb, Croatia \\   
    Date of Birth & April 5, 1987. \\
    Citizenship & Croatian \\    
    Phone & +385 91 1480103 \\
    Email & \href{mailto:jelena.luetic@cern.ch}{\tt jelena.luetic@cern.ch} \\
  \end{tabular}



\section*{Education}

\begin{table}[h!]
 \centering
\begin{tabular}{r | l}
2011. - & \textbf{PhD}, Faculty of Science, Physics department \\ & \textit{Title}: 
Measurement of the cross section for associated production of a W boson \\ & and two b quarks with the CMS detector at the Large Hadron Collider \\ &  \textit{Advisor:} Prof. Vuko Brigljević, thesis defence is scheduled for July, 15th 2015. \\ \\ 
2005. - 2010. & \textbf{MSc}, Faculty of Science, Physics department \\  & \textit{Title}: Measurement of Z boson cross section in proton-proton collisions \\ & with CMS detector at Large Hadron Collider \\ &  \textit{Advisor:} Prof. Ivica Puljak, thesis defended on November, 30th 2010. \\ 
\end{tabular}
\end{table}


\section*{Professional experience}

Since 2009. I've been a part of the CMS group at Ruđer Bošković Institute, working on gauge boson produced in association with jets measurements and on technical aspects of CMS detector:
\begin{itemize}
\item One of the main contributors to Wbb cross section measurement at $\sqrt{s} =$ 8 TeV. Collaboration with University of Wisconsin-Madison, University of Trieste and CERN.
\item Responsible for Lorentz angle monitoring and calibration for CMS Pixel detector.
\item Responsible for development and maintenance of technical tools used in the offline analysis shared with other members of CMS Pixel group.
\item Participated in pixel detector recommissioning during the long shutdown. 
\item Paricipated in CMS Pixel Operations
\end{itemize}



\section*{Teaching}

\begin{tabular}{r | l}
  2011 - 2015 & Faculty of Science, Zagreb, Croatia \\ & Teaching assistant in 2 courses - Introductory laboratory exercises (2011.- 2012.) \\ & and Programming in C (2011.-2015.) \\
  2010 - 2011 & Faculty of Electrical Engineering, Mechanical Engineering and Naval \\ & Architecture in Split, Croatia \\ & Teaching assistant - Laboratory exercises in Modern physics 
\end{tabular}

\section*{Schools and Conferences}

\begin{tabular}{r | l}
  2009 & CERN Summer Student Programme, Geneva, Switzerland \\
  2010 & CERN School of Computing, London, UK \\ 
  2011 & CMSDAS - Data analysis school, Pisa, Italy \\
  2013 & EDIT - Excellence in Detectors, Tsukuba, Japan \\
  	   & \textit{Poster}: CMS Pixel detector and Lorentz angle determination
  2014 & Fermilab - CERN Hadron Collider School, Fermilab, USA\\
\end{tabular}

\section*{Other}
\begin{itemize}
\item 
\end{itemize}


\section*{Publications}

\subsection*{Journal Articles}

\begin{itemize}
\item A General Mathematical Theory of Depreciation, 1929, {\it Journal
    of The American Statistical Association} 20, 340--353.
\item Differential Equations Subject to Error, 1927, {\it Journal of The
    American Statistical Association}.
\item Applications of the Theory of Error to the Interpretation of
  Trends (with H. Working), 1929, {\it Journal of the American
    Statistical Association}.
\end{itemize}

\subsection*{Proceedings}

\begin{itemize}
\item A generalized T-Test and measure of multivariate dispersion,
  Proc. Second Berkeley Symposium of Mathematical Statistics and
  Probability, 1951.
\end{itemize}

\bigskip

% Footer
\begin{center}
  \begin{footnotesize}
    Last updated: \today \\
    \href{\footerlink}{\texttt{\footerlink}}
  \end{footnotesize}
\end{center}

\end{document}