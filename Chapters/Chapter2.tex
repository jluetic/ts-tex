% Chapter Template

\chapter{Theoretical overview} % Main chapter title

\label{Chapter2} % Change X to a consecutive number; for referencing this chapter elsewhere, use \ref{ChapterX}

\lhead{Chapter 2. \emph{Theoretical overview}} % Change X to a consecutive number; this is for the header on each page - perhaps a shortened title

Standard model of elementary particles is a theory emerged in 1960s and 1970s describing all of the known elementary particles and interactions except gravity. The final formulation of Standard model incorporates several theories: quantum electrodynamics, Glashow-Weinberg-Salam theory of electroweak processes and quantum chromodynamics, all of them describing the relations between quarks and fermions. First steps towards formulation of Standard model occurred in 1960. when Sheldon Glashow unified electromagnetic and weak interactions. In 1967. Steven Weinberg and Abdul Salam are using Higgs mechanism in the electroweak theory explaining the origin of mass for elementary particles. After discovery of neutral currents which arise from the exchange of the neutal Z boson, electroweak theory becomes generally accepted. W and Z bosons were discovered in 1981 at CERN, and their masses were in agreement with the Standard model prediction. Theory describing strong interactions got itćs final form in 1974. when it was shown that hadrons are consisting of quarks. There are evidences which show the Standard model is not a final theory of elementary particles, but so far it's predictions were confirmed every time through numerous experimental tests. Standard model has one additional nice property, all fundamental interactions arise from one general principle, the requirement of local gauge invariance. \\
In this chapter a brief overview of the standard model particles and interactions will be shown with the emphasis on the W boson and b quarks which are the most relevant for this thesis. Cross section determination at hadron colliders will be shown.  In the last part of the chapter historical account of the development of W+b-jets theoretical calculations is described together with the existing experimental results.

%----------------------------------------------------------------------------------------
%	SECTION 1
%----------------------------------------------------------------------------------------

\section{Standard model overview}

\par Elementary particle physics is described within a framework of Standard Model. We usually imagine particles as point like objects and some forces between them. Particles, or matter, are fermions, leptons of quarks of spin $s=1/2$. There are three charged leptons, electron, muon and tau which properties are the same except for their mass. Each of the leptons has a corresponding neutrally charged neutrino which has a very small mass. There are six different types of quarks with charge either $Q=2/3$ of $Q=-1/3$. They also carry one additional quantum number which is color charge. All objects observed in nature are colorless giving raise to the concept of quark confinement. Colorless composite objects are classifies into two categories. Bayons are fermions which are made out of three  quarks, for example proton or neutron. The other category are mesons which are made of two quarks like pions. Matter is divided into three categories which are identical except for the masses of the particles. 
\par From the point of view of the quantum field theory, Standard Model is based on a gauge symmetry $SU(3)_C \times SU(2)_L \times U(1)_Y$. Strong interaction is described by $SU(3)_C$, while electroweak sector is described by $SU(2)_L \times U(1)_Y$. All interactions within Standard model are mediated by an elementary particle which is a spin 1 boson. In the case of electromagnetic interaction, mediator is massless photon thus the range of electromagnetic interaction is infinite. For weak force mediators are three massive bosons $W^{\pm}$ and $Z$ and it's range is very small ($10^{-16}$). These four bosons are the gauge bosons of $SU(2)_L \times U(1)_Y$ group. The interaction between electroweak bosons is allowed in the Standard Model in a way that charge conservation principle remains valid. Strong force is mediated by the exchange of 8 massless gluons which are gauge bosons for $SU(3)_C$. Although gluons are massless, the range of the strong force is not infinite. Because of the effect of confinement, the range of the strong force is approximately the size of the lightest hadrons ($10^-{13} cm$). 
\begin{figure}[htbp]
	\centering
		\includegraphics[width=0.5\textwidth]{Figures/Elementary_Particles.png}
		%\rule{35em}{0.5pt}
	\caption[List of Standard model elementary particles]{List of Standard model elementary particles.}
	\label{fig:SM_particles}
\end{figure} 
\par Scalar sector of the Standard Model has been experimentally confirmed only recently \cite{Aad:2012tfa,Chatrchyan:2012ufa}. The fact that weak gauge bosons are massive indicates that $SU(2)_L \times U(1)_Y$ is not a good symmetry of the vacuum. In contrast with photon being massless, $U(1)_em$ is a good symmetry of the vacuum which means that $SU(2)_L \times U(1)_Y$ electroweak symmetry is somehow spontaneously broken to $U(1)_em$ of electromagnetism. Spontaneous symmetry breaking is implemented through Higgs mechanism which gives masses to $W\pm$ and $Z$ boson, fermions and leaves photon massless. The details of the mechanism can be found elsewhere \cite{Griffiths:1987tj} but the main point is that the mechanism also predicts a new scalar and electrically neutral particle which is called Higgs boson. The search for Higgs boson lasted few decades before finally in 2012, a new particle was discovered with mass of 125 GeV. In subsequent years, properties of this new particle have been measured and at this point, all measurements agree with Standard Model predictions for Higgs boson.    
   

%-----------------------------------
%	SUBSECTION 1.1
%-----------------------------------

\subsection{Bottom quarks}

Bottom quark was first predicted by Makoto Kobayashi and Toshihide Maskawa in 1974 when extending Cabbibo weak mixing angle to take into account CP violation observed in neutral K mesons. \cite{Kobayashi:1973fv} The name "bottom" was introduced in 1975 by Haim Harari. The bottom quark was discovered in 1977 by the Fermilab E288 experiment team led by Leon M. Lederman through the observation of $\Upsilon$ resonance. \cite{PhysRevLett.39.252} Kobayashi and Maskawa won the 2008 Nobel Prize in Physics for their explanation of CP-violation. 
\par At the the LHC, the main production mechanism for b quarks is through strong interaction ($g\rightarrow bb$) and top quark decay ($t\rightarrow Wb$). Every b quark, after production, goes through the process of hadronisation, forming one of the color neutral B mesons. B meson decays electromagnetically if produced in excited state to the ground state. Lowest state B mesons decay weakly, resulting in relatively long lifetime of 1.5 ps. According to CKM matrix \ref{eq:ckmmatrix}, b quark can decay either to c quark ($\approx 95\%$ of the cases) or u quark ($\approx 95\%$ of the cases). Long lifetime of b quark makes it possible to traverse a substantial distance inside the detector. This fact is used in the creation of various b-tagging algorithms which are taking into account tracks originating from displaced verices, discussed in Section \ref{Chapter4}.
\begin{equation} 
\begin{pmatrix} 
d' \\ s' \\ b' 
\end{pmatrix} = 
\begin{pmatrix}
V_{ud} & V_{us} & V_{ub} \\
V_{cd} & V_{cs} & V_{cb} \\
V_{td} & V_{ts} & V_{tb} 
\end{pmatrix} 
\begin{pmatrix} 
d \\ s \\ b 
\end{pmatrix}
= \begin{pmatrix}
0.974 & 0.225 & 0.003 \\
0.225 & 0.973 & 0.041 \\
0.009 & 0.040 & 0.999 
\end{pmatrix} 
\begin{pmatrix} 
d \\ s \\ b 
\end{pmatrix} 
\label{eq:ckmmatrix} 
\end{equation}


%-----------------------------------
%	SUBSECTION 1.2
%-----------------------------------

\subsection{Discovery and role of W boson}

W boson is one of the massive mediators of weak interaction with a mass of $m_W=80.1$ GeV.
The theory of the weak interactions got it's final form in 1968 when Sheldon Glashow, Steven Weinberg, and Abdus Salam unified a theory of electromagnetism and weak interactions. The discovery of W and Z bosons at UA1 and UA2 experiments was one of the major successes of the CERN experimental facility. Super Proton Synchrotron was the first accelerator powerful enough to produce W and Z bosons. Both collaborations reported their findings in 1983 \cite{Arnison:1983rp,Banner:1983jy}.
W boson at the LHC is primarily produced through quark-antiquark annihilation. In majority of the cases, W boson decays to quark-antiquark pair ($66\%$). Other decay channels include creation a lepton and it's corresponding neutrino($\approx 10\%$ per lepton generation). This decay channel was the most important for W boson discovery and it's still essential for W boson detection at hadron colliders despite the large hadronic backgrounds. 
\par Study of W+jets production at hadron colliders started in 1980s motivated by the top quark searches. Additional jets either come from radiation of additional quarks or gluons. However, becuse they carry color charge, quarks and gluons undergo the process of parton shower and hadronization forming jets in the detector. Parton shower is the process in which a high energy colored particle emits a low energy colored particle while hadronization is the process in which colored particle combine to form color neutral particles. Parton shower and hadronization cannot be computed analytically, but have to be modeled using Monte Carlo simulations. As a result of these processes, in the final state there can be a number of jets that doesn't  correspond to the number of incoming partons. This becomes relevant when trying to form an inclusive W+jets sample from exclusive (W + 1 jet, W + 2 jets...) samples and some process of matching has to be performed in order to avoid double counting. Matching procedure is described in detail in \cite{Campbell:2008cr}.  
\par Many theoretical issues arise when trying to compute cross sections for W+jets processes. Divergences while calculating amplitudes come from emission of soft particles or collinear jets. These problems are solved by introducing a cut-off called factorization scale. Other divergences come from integrating higher-order loops. Usually this type of divergence is than included into renormalized coupling constant. This procedure, however introduces a certain scale dependence into the result which will be discussed further in Section \ref{subsec:2.1}. 


%----------------------------------------------------------------------------------------
%	SECTION 2
%----------------------------------------------------------------------------------------

\section{W + b jets at hadron colliders}

First theoretical computations of W boson in association with b jets were published in 1993 \cite{Mangano:1992kp}, however only recently enough luminosity has been collected at hadron colliders to be able to make cross section measurements. This process was first interesting as a background to top quark searches and measurements where top quark decays to W boson and a b quark. In past few years, with the Higgs boson discovery, an important open question is whether the new particle also couples to fermions, and in particular to bottom quarks. Determination of this coupling requires
direct measurement of the corresponding Higgs boson decay, as recently reported
by the CMS experiment in the study of Higgs decays to bottom quarks \cite{Chatrchyan:2013zna,Chatrchyan:2014vua}.
Standard model Higgs boson branching ratio for decays into a bottom quark-antiquark
pair (bb) is $\approx 58\%$. Study of this decay channel is therefore essential in determining the nature of the newly discovered boson. The measurement of the H $\rightarrow$ bb decay will be the first direct test of whether the observed boson interacts as expected with the quark sector, as the coupling to the top quark has only been tested through loop effects. However, the large backgrounds for this measurement make it essential that all the contributing processes including W+b jets are well understood.
There are also Beyond Standard Model searches where contributions from this process is substantial including some Supersymmetry searches with lepton, b jets and missing energy in the final state.


%-----------------------------------
%	SUBSECTION 2.1
%-----------------------------------

\subsection{Cross sections at hadron colliders}
\label{subsec:2.1}

	Determining cross sections for processes at hadron collides is not an easy task. With proton being a composite object consisting of partons, it is necessary to include it's internal structure as well as the diagrams for hard scattering of interest. This means soft and hard processes are occurring in the same event. Quarks and gluons within proton interact through strong force and are described using quantum chromodynamics. Two processes make it possible to perform calculations within the QCD, asymptotic freedom and factorization theorem. Since strong force coupling constant $\alpha_s$ depends on the scale, for high momentum transfers ($Q >> \Lambda_{QCD}\approx 200$MeV) it becomes sufficiently small to make perturbative expansion in $\alpha_s$ possible. This feature is called asymptotic freedom and it is used to determine the hard process cross section. Figure \ref{fig:alpha_s} shows the results of the $\alpha_s$ measurements which is in complete agreement with the QCD predictions of asymptotic freedom. \\
\begin{figure}[htbp]
	\centering
		\includegraphics{Figures/alpha_s.pdf}
		%\rule{35em}{0.5pt}
	\caption[Strong force coupling constant]{Summary of measurement of strong coupling constant $\alpha_s$ \cite{Agashe:2014kda} }
	\label{fig:alpha_s}
\end{figure}

	\par Perturbative QCD cannot be used if the momentum transfer values are small and the coupling constant becomes large. This phenomenon is called \textit{confinement} and it requires different treatment for the quarks inside the proton. Internal structure of a proton is described using parton distribution functions which  are determined through deep inelastic scattering experiments. Parton distribution functions for each of the partons inside a proton is shown in Figure \ref{fig:MSTW} made with one specific PDF function(MSTW). Using DGLAP equations, it is possible to evolve the PDFs for any momentum transfer value which is described in detail in \cite{Campbell:2006wx} 

\begin{figure}[htbp]
	\centering
		\includegraphics{Figures/MSTW.pdf}
		%\rule{35em}{0.5pt}
	\caption[Parton distribution functions for different momentum transfers]{Parton distribution functions calculated by the MSTW group for $Q=10$GeV and $Q=10^4$GeV \citep{Martin:2009iq}}
	\label{fig:MSTW}
\end{figure}

While performing perturbative QCD calculations, it is important to impose conditions to the final state in order to avoid soft and collinear divergences. Collinear divergences originate from configurations with a small opening angle between jets. Soft divergences appear when quark or gluon is irradiated at low momentum. Factorization scale is introduced as a cut-off for diagram calculation below which perturbative QCD calculation is not performed which means that hard scattering between partons is independent from the parton internal structure. The main point of the factorization theorem  is that because of energy dependence of strong coupling constant, hard and soft part of the process are happening at different time scales ans soft part is factorized inside a parton distribution function.
\begin{figure}[htbp]
	\centering
		\includegraphics{Figures/diagram.pdf}
		%\rule{35em}{0.5pt}
	\caption[Drawing of a proton-proton collision]{Drawing of a proton-proton collision.}
	\label{fig:pp_drawing}
\end{figure}
Drawing of a proton proton collision is shown in figure \ref{fig:pp_drawing}. If we want to calculate the cross section for some process where there are two protons in the initial state and some interesting final state which we call X, according to \cite{Campbell:2006wx}, necessary steps are:
\begin{enumerate}
	\item Identify the leading order partonic processes that contribute to X
	\item Calculate the corresponding hard scattering cross section
	\item Determine the appropriate PDFs for initial state partons
	\item Make a specific choices for factorization($\mu_F$) and renormalization($\mu_R$) scales
	\item Perform integration over the fraction of momentum available for a given parton(x)  
\end{enumerate}
The cross section at hadron collides is thus a convolution of the hard scattering perturbative cross section and two incoming parton distribution functions.
\begin{equation}
\sigma_{AB} = \sum\limits_{n=1}^{\infty} \alpha_{s}^{n}(\mu_{R}^2)\sum\limits_{i,j} \int dx_1 dx_2 f_{i/A}(x_1,\mu_{F}^2) f_{j/B}(x_2,\mu_{F}^2) \sigma_{ij \rightarrow X}^{(n)}(x_1 x_2s,\mu_{R}^2,\mu_{F}^2)
 \label{eq:xsec}
\end{equation} 
\par Equation \ref{eq:xsec} shows cross section perturbation series in $\alpha_s$, $n$ denotes the order of the series where $n=1$ is leading order, $n=2$ is next to leading order, etc. 
Hard process cross section between two partons $\sigma_{ij \rightarrow X}^{(n)}$ is computed in the framework of perturbative QCD and depends on $s$ which is squared center of mass energy. Two parton distribution functions are denoted with $f_{i/A}$ and $f_{j/B}$ and correspond to the probability density that parton $i$($j$) with proton momentum fraction $x_1$($x_2$) will be found inside a proton. Sum over all combinations of partons has to be computed. Integral over available phase space for proton fraction momentum $dx$ is usually carried out by simulations.\\ 
Here $\mu_F$ represents \textit{factorization scale} and $\mu_R$ is \textit{renormalization scale} for running coupling constant. They are arbitrary cut-offs used to remove nonperturbative effects and be able to make perturbative calculations. If cross section is computed in full series, $\mu_F$ and $\mu_R$ should cancel out, and scale dependence should disappear. However, since fewer orders are used and some residual scale dependence is still present. This dependency can be used to estimate the contribution of the missing orders in the series.  
\par Usually factorization and renormalization scales are chosen to be identical and close to the scale of the process in question.   
\begin{figure}[htbp]
	\centering
		\includegraphics{Figures/pp_xsec.pdf}
		%\rule{35em}{0.5pt}
	\caption[Proton-proton cross sections]{Standard model cross sections as a function of center of mass energy.\cite{Campbell:2006wx} }
	\label{fig:pp_xsec}
\end{figure}
\par Figure \ref{fig:pp_xsec} shows some interesting Standard model cross sections in proton-proton and proton-antiproton collisions as a function of a center of mass energy. All cross sections have been computed to the NLO order using the above described procedure.

 
%-----------------------------------
%	SUBSECTION 2.2
%-----------------------------------

\subsection{Contributions to Wbb cross section}

Soft and collinear divergences are naturally avoided in processes with b jets because of relatively high mass of b quark which means that the scale of the process doesn't go below $2m_b$.  
\begin{figure}[htbp]
	\centering
		\includegraphics{Figures/LO_diag.pdf}
		%\rule{35em}{0.5pt}
	\caption[Leading order Wbb Feynmann diagram]{Leading order Wbb Feynmann diagram}
	\label{fig:LO_diag}
\end{figure}

%----------------------------------------------------------------------------------------
%	SECTION 2.3
%----------------------------------------------------------------------------------------

\section{Previous measurements}
\label{sec:2.3}
	\par Previous measurements of a W boson produced in association with b quarks have been performed on different experiments. However, the final states and phase space used in these measurements were different, which means that the results cannot be directly compared, but they can be compared with theoretical predictions. This process was measured for the first time at Tevatron with D0 and CDF experiments at $\sqrt{s} =$ 1.96 TeV. The CDF collaboration published its result in 2009 and the cross-section measured is that of “jets from b-quarks produced with a W boson”\citep{Aaltonen:2009qi}. The event selection is based on reconstructing a leptonically decaying W boson, and one or two jets where at least one has to be b-tagged. Events with jets from light quarks are vetoed with a cut on the secondary vertex mass. Contribution of other background events containing a b quark in final state (e.g. events with top quark) is estimated using Monte Carlo simulations. The measured cross section is 2.8 standard deviations higher than corresponding theoretical prediction. \\
	D0 collaboration published their result in 2012. with somewhat different phase space definition\citep{D0:2012qt}. The difference with respect to the CDF measurement consists in the inclusion of the events with 3 jets and reduced pseudorapidity range in which the measurement was performed. The measurement technique is similar to that of CDF, although b-tagging algorithms were slightly different. The measured cross section was in good agreement with the Standard model prediction.\\
	First measurements at the LHC were published by the ATLAS collaboration based on 36/pb of integrated luminosity at $\sqrt{s} =$ 7 TeV. One year later they improved their measurement using $4.6/$fb \cite{Aad:2013vka}. Selected events contain one reconstructed electron or muon, significant amount of missing transverse energy and one or two jets where exactly one is b-tagged. The phase space is divided in two regions, depending on the number of jets. Events with exactly 2 b jets and events with more than 2 jets are vetoed in order to suppress background events from top quark decay. The results are shown in Figure \ref{fig:atlas_tot}. The cross section measurement in the one jet region shows an excess corresponding to 1.5 standard deviations. In the two jet region, the measured cross section is in good agreement with theoretical predictions. A differential cross section measurement as a function of leading b jet transverse momentum has been performed for the first time and shown in figure \ref{fig:atlas_diff}. The cross section measurement in the one jet region is again higher that NLO predictions but within theoretical and experimental uncertainties. The cross section measured for the events with two jets is in good agreement with the theoretical prediction.
\begin{figure}[htbp]
	\centering
		\includegraphics[width=0.7\linewidth]{Figures/atlas_total.pdf}
		%\rule{35em}{0.5pt}
	\caption[Atlas Wbb total cross section measurement]{Measured fiducial cross-sections in the electron, muon, and combined electron and muon channels. The cross-sections are given in the 1-jet, 2-jet, and 1+2-jet fiducial regions.\cite{Aad:2013vka} }
	\label{fig:atlas_tot}
\end{figure}
\begin{figure}
\centering
  \begin{subfigure}{.5\textwidth}
  	\centering
  	\includegraphics[width=\linewidth]{Figures/atlas_diff1j.pdf}
	\caption{}  
  	\label{fig:atlas_diff1j}
\end{subfigure}%
\begin{subfigure}{.5\textwidth}
  \centering
  \includegraphics[width=\linewidth]{Figures/atlas_diff.pdf}
  \caption{}
  \label{fig:atlas_diff2j}
\end{subfigure}
\caption[Measured differential W+b-jets cross-sections as a function of leading b-jet $p_{T}$]{Measured differential W+b-jets cross-sections as a function of leading b-jet $p_{T}$ in the 1-jet (\ref{fig:atlas_diff1j}) and 2-jet (\ref{fig:atlas_diff2j}) fiducial regions, obtained by combining the muon and electron channel results. \cite{Aad:2013vka}}
\label{fig:atlas_diff}
\end{figure}		
The CMS collaboration published its results corresponding to data collected during 2011. The measured events contained a muon and missing transverse energy in the final state, together with two b-tagged jets. The measured cross section is in excellent agreement with the Standard model prediction.\citep{Chatrchyan:2013uza}

\begin{figure}[htbp]
	\centering
		\includegraphics{Figures/cms_tot.pdf}
		%\rule{35em}{0.5pt}
	\caption[CMS Wbb total cross section measurement]{\cite{Chatrchyan:2013uza} }
	\label{fig:cms_total}
\end{figure}