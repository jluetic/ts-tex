% Chapter 1

\chapter{Introduction} % Main chapter title

\label{Chapter1} % For referencing the chapter elsewhere, use \ref{Chapter1} 

\lhead{Chapter 1. \emph{Introduction}} % This is for the header on each page - perhaps a shortened title

% Filozofija
The standard model of particle physics tries to give an answer to the questions what is the matter made of and how does it interact. The predictions of the standard model have been thoroughly tested many times and various precision measurements were performed at different experiments. Nevertheless every time the standard model predictions were confirmed. The only missing link, the long sought Higgs boson, was found in 2012, with its properties measured in the following years thus completing the picture of elementary particles. However, standard model still leaves some unexplained phenomena, e.g. neutrino oscillations, matter-antimatter asymmetry, the existence of dark matter and dark energy, giving rise to the physics beyond standard model. The sensitivity of such processes in the high energy physics experiments strongly depends on the precise measurements of known processes. In such measurements, poorly known yields and kinematics of the known processes would lead to high uncertainties, and thus reducing the sensitivity of the experiment. 
% Zasto b kvark - QCD
\par The production of a W boson in association with a pair of b quarks (Wbb) has
been the topic of many theoretical studies and simulations. It is however still not well described due to collinear and infrared divergences. Several theoretical
approaches, implemented in different simulation packages, have been used to describe the Wbb production mechanism. A precise measurement of Wbb production will allow to
further constrain theoretical predictions in the framework of perturbative quantum chromodynamics (pQCD) and to test the validity of different theoretical models used in simulations. On the other hand, Wbb is a background to different standard model processes as well as beyond standard model searches. It is one of the main backgrounds to top quark  and Higgs boson measurements, in particular, events in which Higgs boson is produced in association with a W boson and decays to a pair of b quarks.
% Sto je pokazano
\par This thesis is focused on the measurement of the cross section of $pp\rightarrow W+bb+X$ process using the data collected during 2012 at $\sqrt{s} = 8$ TeV. W boson is decaying either to an electron or muon, and a corresponding neutrino. The presence of a W boson is identified through the detection of an energetic, isolated lepton, i.e. lepton without any additional activity in some predefined cone around it, and a significant amount of missing energy, which indicates the presence of a neutrino. Jets in the detector are identified as a collimated spray of particles. Selected jets are required to be tagged as jets originating from b quarks. This procedure is called b-tagging and it exploits unique b quark properties to derive a single discriminator value to distinguish between b jets and jets from lighter quarks, or gluons. 
% Organizacija teze

The thesis is organized as follows. Chapter 2 shows a brief introduction to the standard model, with the emphasis on the discovery and the role of W boson and b quarks within the standard model. An overview of the phenomenology of the proton collisions at hadron colliders is shown. Steps for the theoretical calculation of the $pp\rightarrow W+bb+X$ process are given for both, single parton scattering and double parton scattering production mechanisms. The end of the chapter summarizes all previous measurements of W boson and b quarks in the final state. Chapters 3 and 4 are focused on the description of the LHC and CMS respectively. All CMS subsystems are described and their role is explained. Chapter 5 describes the procedure for reconstruction of various physics objects, including electrons, muons and jets, and estimation of missing energy. Chapter 6 lists all data and Monte Carlo samples used. Criteria for the signal selection is described and all major backgrounds are identified. Chapter 7 describes all steps in the cross section determination, including the fitting procedure used to extract final yields, and the acceptance and efficiency estimation. In the end, the results are presented, together with the comparison to theoretical predictions. Final chapter briefly summarizes the results, and shows the prospects for the future research on this topic.
