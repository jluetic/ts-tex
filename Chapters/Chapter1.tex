% Chapter 1

\chapter{Introduction} % Main chapter title

\label{Chapter1} % For referencing the chapter elsewhere, use \ref{Chapter1} 

\fancyhead[LE,RO]{Chapter 1. \emph{Introduction}} % This is for the header on each page - perhaps a shortened title

% Filozofija
The standard model of particle physics tries to give an answer to the questions what is matter made of and how does it interact. The predictions of the standard model have been thoroughly tested through various precision measurements at different experiments. Every time the standard model predictions were confirmed. The only missing link, the long sought Higgs boson, was found in 2012, and its properties were measured in the following years, thus completing the picture of elementary particles. However, the standard model still leaves some unexplained phenomena, e.g. neutrino oscillations, matter-antimatter asymmetry, the existence of dark matter and dark energy, motivating the searches for physics beyond standard model. The sensitivity to such processes in high energy physics experiments strongly depends on the precise measurements of known processes. In such measurements, poorly known yields and kinematics of known processes would lead to high uncertainties, and thus reducing the sensitivity of the experiment. 
% Zasto b kvark - QCD
\par The production of a W boson in association with a pair of b quarks (Wbb) in proton collisions has
been the topic of many theoretical calculations and simulations. It is however still not well described due to divergences which arise in theoretical calculations, in cases where b quarks are collinear or there is a low energy massless particle irradiated. Several theoretical
approaches, implemented in different simulation packages, have been used to describe the Wbb production mechanism. A precise measurement of Wbb production will allow to
further constrain theoretical predictions in the framework of perturbative quantum chromodynamics (pQCD) and to test the validity of different theoretical models used in simulations. On the other hand, Wbb is a background in the measurements of different standard model processes as well as in several searches for physics beyond standard model. It is one of the main backgrounds in top quark  and Higgs boson measurements. When Higgs boson is produced in association with a W boson and decays to a pair of b quarks, this shows the same signature in the detector as the Wbb.
% Sto je pokazano
\par The goal of this thesis is a measurement of the cross section of the $pp\rightarrow W+bb+X$ process using the data collected during 2012 at $\sqrt{s} = 8$ TeV. The data is provided by the Large Hadron Collider (LHC) accelerator at CERN, Geneva and collected by the Compact Muon Solenoid (CMS). W boson used in the analysis is decaying either to an electron or muon, and a corresponding neutrino. The presence of a W boson is identified through the detection of an energetic, isolated lepton, i.e. a lepton with low additional activity in some predefined cone around it, and a significant amount of missing energy, which indicates the presence of a neutrino. Jets in the detector are identified as collimated sprays of particles. Selected jets are required to be tagged as jets originating from b quarks. This procedure is called b-tagging and it exploits the unique properties of b quark to derive a single discriminator value to distinguish between b jets and jets from lighter quarks, or gluons. 
% Organizacija teze

The thesis is organized as follows. Chapter 2 gives a brief introduction to the standard model, including the discovery and the role of W boson and b quarks within the standard model. An overview of the phenomenology of proton collisions at hadron colliders is shown. All the steps for the theoretical calculation of the $pp\rightarrow W+bb+X$ process are given for both, single parton scattering and double parton scattering production mechanisms. The end of the chapter summarizes all previous measurements of W boson and b quarks in the final state. Chapters 3 and 4 are focused on the description of the LHC and the CMS respectively. All CMS subsystems are described and their role is explained. Chapter 5 describes the procedure for the reconstruction of various physics objects, including electrons, muons and jets, and the estimation of missing energy. Chapter 6 lists all data and Monte Carlo samples used. The criteria for the signal selection are described and all major backgrounds are identified. Chapter 7 describes all steps in the cross section determination, including the fitting procedure used to extract the final yields, and the acceptance and efficiency estimation. In the end, the results are presented, together with the comparison to theoretical predictions. Chapter 8 briefly summarizes the results, and shows the prospects for the future research on this topic.
